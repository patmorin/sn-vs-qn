\documentclass[kpfonts]{patmorin}
\usepackage{pat}
\usepackage{paralist}

\usepackage[utf8]{inputenc}

\setlength{\parskip}{1ex}


\newtheorem{property}{Property}
\newcommand{\plabel}[1]{\label{prop:#1}}
\newcommand{\pref}[1]{Property~\ref{prop:#1}}

\DeclareMathOperator{\sn}{sn}
\DeclareMathOperator{\qn}{qn}
\DeclareMathOperator{\tw}{tw}

\renewcommand{\SS}{\mathcal{S}}



\newcommand{\aref}[1]{(X\ref{a:#1})}
\newcommand{\alabel}[1]{\label{a:#1}}

\title{\MakeUppercase{Stack Number is not Queue-Number Bounded}}
\author{TBD}

\begin{document}
\maketitle

\begin{abstract}
  We describe a family of graphs in which every member has queue number at most X, but for every integer $s$, there is a member of the family whose stack number is greater than $s$.
\end{abstract}

\section{Introduction}

We will prove the following theorem.

\begin{thm}\thmlabel{family}
  There exists a family $\mathcal{F}$ of graphs for which $\qn(G)\le X$ for every $G\in\mathcal{F}$ and, for every $s\in\N$, there exists $G\in\mathcal{F}$ for which $\sn(G)>s$.
\end{thm}

\section{The Proof}

Let $S_B$ denote the star graph root $r$ and $B$ leaves denoted $v_1,\ldots,v_B$ and let $P$ denote the path on $n$ vertices $1,\ldots,n$.  We will consider the graph $G:=S_B\boxtimes P\boxtimes P$.

\begin{lem}
    $\qn(G) \le 9$.
\end{lem}

\begin{proof}
From David's email:
\begin{verbatim}
    See Theorem 7 in the attached old paper for the necessary upper bound
    on qn(G * H).
    It says qn(G * H) <= 2 sqn(G) * qn(H) + sqn(G) + qn (H) .
    Here sqn is strict queue-number, which requires edges uv and uw to be
    in different queues, whenever u < v < w.
    We can bound sqn(P^2 * P^2) directly or use
    sqn(H) <= qn(H) * ( \Delta(H) + 1 ).
\end{verbatim}
\end{proof}

Now, consider a hypothetical $s$-stack layout $(\varphi,\prec)$ of $G$.

For each node $v$ of $S_b$, we define $\pi_v$ as the permutation of $\{1,\ldots,n\}^2$ in which $(x_1,y_1)$ appears before $(x_2,y_2)$ if and only $(v,x_1,y_1)\prec (v,x_2,y_2)$.  The following lemma is an immediate consequence of the Pigeonhole Principle:

\begin{lem}\lemlabel{uniform_order}
    There exists a permutation $\pi$ of $\{1,\ldots,n\}^2$ and a set $L_1$ of leaves of $S_B$ of size $B_1\ge \lceil B/(n^2)!\rceil$ such that $\pi_{v}=\pi$ for each $v\in L_1$.
\end{lem}

Let $Q:=P\boxtimes P$ be the $n\times n$ grid with diagonals, so that $G':=S_b\boxtimes Q$.  For each leaf $v$ in $L$, consider the subgraph $Q_v$ of $G$ induced by the vertex set $\{(v,x,y):x,y\in\{1,\ldots,n\}$.  The edge colouring $\varphi$ used in the stack layout gives an edge colouring of $Q_v$ using $s$ colours.  The graph $Q_v$ is isomorphic to $Q$, so the edge colouring of $Q_v$ defines an edge colouring of $Q$.  We call this colouring of $Q$ $\varphi_v:Q\to\{1,\ldots,s\}$.  The graph $Q$ has less than $7n^2$ edges, so there are fewer than $s^{7n^2}$ edge colourings of $Q$.  Another application of the Pigeonhole Principle proves the following:

\begin{lem}\lemlabel{uniform_colour}
    There exists a subset $L_2\subseteq L_1$ of size $B_2\ge B_1/s^{7n^2}$
    and an edge colouring $\varphi_0:Q\to\{1,\ldots,s\}$ such that $\varphi_v=\varphi_0$ for each $v\in L_2$.
\end{lem}

\begin{lem}\lemlabel{forward_or_backward}
    There exists a sequence $L_3:=u_1,\ldots,u_{B_3}$ with $\{u_1,\ldots,u_{B_3}\}\subseteq L_2$ of length $B_3\ge (B_2)^{1/2^{n^2-1}}$ such that, for each $a\in V(Q)$, $(u_1,a)\prec (u_2,a)\prec\cdots\prec (u_{B_3},a)$ or $(u_1,a)\succ (u_2,a)\succ\cdots\succ (u_{B_3},a)$.
\end{lem}

\begin{proof}
    Let $x_1,\ldots,x_{n^2}$ denote the vertices of $Q$, in any order.
    Begin with the sequence $S_1:=v_{1,1},\ldots,v_{1,b}$ that contains all $b_1:=B_2$ elements of $L_2$ ordered so that $(v_{1,1},x_1)\prec\cdots(v_{1,b},x_1)$.  For each $i\in\{2,\ldots,n^2\}$, the Erd\H{o}s-Szekeres Theorem implies that, $S_{i-1}$ contains a subsequence $S_i:=v_{i,1},\ldots,v_{i,b_i}$ of length $b_i\ge \sqrt{|S_{i-1}|}$ such that $(v_{i,1},x_i)\prec\cdots\prec(v_{i,b_i},x_i)$ or $(v_{i,1},x_i)\succ\cdots\succ(v_{i,b_i},x_i)$.  It is straightforward to verify by induction that $b_i \ge B_3^{1/2^{i-1}}$ resulting in a final sequence $S_{n^2}$ of length at least $B_2^{1/2^{n^2-1}}$.
\end{proof}

Let $d=B_3$ and let $S_d$ be the the subgraph of $S_b$ induced by $\{r\}\cup\{u_1,\ldots,u_{d}\}$ where $u_1,\ldots,u_d$ is the sequence of leaves defined in \lemref{forward_or_backward}.  Consider the vertex-colouring of $Q$ obtained by colouring each vertex $a\in V(Q)$ \emph{red} if $(u_1,a)\prec\cdots\prec (u_d,a)$ and colouring $a$ \emph{blue} if $(u_1,a)\succ\cdots\succ(u_d,a)$.

\begin{lem}\lemlabel{hex_lemma}
    The graph $Q$ contains an $n$-vertex path $R$ consisting entirely of red vertices or entirely of blue vertices.
\end{lem}

\begin{proof}
    If we remove every edge in $\{(x,y)(x+1,y+1):x,y\in\{1,\ldots,n\}\}$ from $Q$ we obtain a planar graph $H$ in which all vertices not on the outer face have degree six and ever inner face is a 3-cycle.  The dual of $H$ is the board on which the game Hex is played.  The well-known \emph{Hex Lemma} states that any colouring of the vertices of $H$ with colours red and blue contains exactly one of the following \cite{hex_lemma}:
    \begin{compactenum}
        \item a path with endpoints $(x,1)$ and $(x',n)$ consisting entirely of red vertices, for some $x,x'\in\{1,\ldots,n\}$; or
        \item a path with endpoints $(1,y)$ and $(n,y')$ consisting entirely of blue vertices, for some $y,y'\in\{1,\ldots,n\}$.
    \end{compactenum}
    In either case, the path at least $n$ vertices and therefore has a $n$-vertex subpath consisting entirely of red vertices or entirely of blue vertices.
\end{proof}

Without loss of generality, assume that the path $R:=a_1,\ldots,a_n$ defined by \lemref{hex_lemma} consists entirely of red vertices, so that $(u_1,a_j)\prec\cdots\prec (u_d,a_j)$ for each $j\in\{1,\ldots,n\}$.
Recall that $(\varphi,\prec)$ is a hypothetical $s$-stack layout of $G$ and therefore it is also an $s$-stack layout of the subgraph $X:=S_d\boxtimes R$.  The following result finishes the proof by showing that this is not possible when $n> 2s$ and $d> s2^{2s+1}$.

\begin{lem}
    The graph $X$ contains a set of edges of size at least $\min\{d/2^{n},n\}/2$ that are pairwise crossing with respect to $\prec$.
\end{lem}

\begin{proof}
    We will define two sequences of nested sets $A_1\supseteq A_1\supseteq A_{n}$ of leaves of $S_d$ so that each $A_i$ satisifies the following conditions:
    \begin{compactenum}[(C1)]
        \item $A_i$ contain $d_i\ge d/2^i$ leaves of $S_d$.
        \item Each leaf $v\in A_i$ defines an $i$-element vertex set $Z_{i,v}:=\{(v,a_j):j\in\{1,\ldots,i\}\}$.  For any distinct $v,w\in A_i$, $Z_{i,v}$ and $Z_{i,w}$ are separated with respect to $\prec$.  In other words, $Z_{i,v}\prec Z_{i,w}$ or $Z_{i,v}\succ z_{i,w}$.
    \end{compactenum}

    Before defining $A_1,\ldots,A_n$ we first show how the existence of the set $A_n$ implies the lemma.  To avoid triple-subscripts, let $d':=d_n\ge d/2^n$.   The set $A_n$ defines vertex sets $Z_{n,v_1}\prec\cdots\prec Z_{n,v_{d'}}$.  Recall that $r$ is the root of $S_b$ so it is adjacent to each of $v_{1},\ldots,v_{d'}$ in $S_b$.  Therefore, for each $j\in\{1,\ldots,n\}$ and each $i\in\{1,\ldots,d'\}$, the edge $(r,a_j)(v_i,a_j)$ is in $X$. Therefore, $(r,a_j)$ is adjacent to an element of each of $Z_{n,v_1},\ldots,Z_{n,v_{d'}}$.

    Since $Z_{n,v_1},\ldots,Z_{n,v_{d'}}$ are separated with respect to $\prec$, when viewed from afar, this situation looks like a complete bipartite graph $K_{n,d'}$ with the root vertices $L:=\{(r,a_j):j\in\{1,\ldots,n\}\}$ in left part and the groups $R:=Z_{n,v_1}\cup\cdots\cup Z_{n,v_{d'}}$ in the right part.  Any linear ordering of $K_{n,d'}$ has a large set of pairwise crossing edges so, intuitively, the graph induced by $L\cup R$ should also have a large set of pairwise crossing edges. \lemref{twister}, below, formalizes this and shows that this graph has a set of at least $\min\{d',n\}/2$ pairwise crossing edges.

    All that remains is to define the sets $A_1\supseteq\cdots\supseteq A_n$.  The set $A_1$ contains all the leaves of $S_d$.  For each $i\in\{2,\ldots,n\}$, the set $A_i$ is defined as follows:  Let $Z_1,\ldots,Z_r$ denote the sets $\{\{(v,a_j):j\in\{1,\ldots,i-1\}\}:v\in A_{i-1}$ ordered so that $Z_1\prec\cdots\prec Z_r$. Label the vertices of $A_{i-1}$ $v_1,\ldots,v_r$ so that $(v_1,a_{i-1})\prec\cdots\prec (v_r,a_{i-1})$. (This is equivalent to naming them so that $(v_j,a_j)\in Z_j$ for each $j\in\{1,\ldots,r\}$.)

    Now we define the set $A_i:=\{v_{2k+1}:k\in\{0,\ldots,\lfloor(r-1)/2\rfloor\}$.  All that remains is to verify that $A_i$ satisfies (C1) and (C2).  To see that $A_i$ satisfies (c1) just observe that $|A_i|=\lceil r/2\rceil \ge r/2= |A_{i-1}|/2\ge d/2^{i}$.  All that remains is to show that $A_i$ satisfies (C2).

    For each $j\in\{i-1,i\}$, let $Q_j:=\{(v,a_j):v\in A_{i-1}\}$.
    Recall that, for each $v\in A_{i-1}$, the edge $e_v:=(v,a_{i-1})(v,a_i)$ is in $X$.  We have the following properties:
    \begin{compactenum}[(P1)]
        \item By \lemref{uniform_colour}, $\varphi(e_v)=\varphi_0(a_{i-1},a_i)$ does not depend on $v$.  In particular for distinct $v,w\in A_{i-1}$ the edges $e_v$ and $e_w$ do not cross.
        \item By the application of \lemref{hex_lemma} the order of vertices in $Q_{i-1}$ by $\prec$ is identical to the order of vertices in $Q_i$ by $\prec$.  That is $(v,a_{i-1})\prec (w,a_{i-1})$ if and only if $(v,a_{i})\prec (w,a_{i})$ for each $v,w\in A_{i-1}$.
        \item By \lemref{uniform_order}, $(v,a_{i-1})\prec (v,a_i)$ for every $v\in A_{i-1}$ or $(v,a_{i-1})\succ (v,a_i)$ for every $v\in A_{i-1}$.
    \end{compactenum}
    These three conditions imply that the vertex sets $Q_{i-1}$ and $Q_{i}$ interleave perfectly with respect to $\prec$. More precisely,
    \[
        (v_1,a_{i-1+b})\prec (v_1,a_{i-b}) \prec (v_2,a_{i-1+b}) \prec (v_2,a_{i-b}) \cdots \prec (v_r,a_{i-1+b}) \prec (v_r,a_{i-b})
    \]
    for some $b\in\{0,1\}$.  Suppose, without loss of generality that $b=0$. [TODO: Explain why (P1)--(P3) imply a perfect interleave.]

    For each odd $j\in\{1,\ldots,r-2\}$ we have $(v_j,a_i)\prec (v_{j+1},a_{i-1}) \prec Z_{j+2}$.  Therefore $Z_j\cup\{(v_j,a_i)\} \prec Z_{j+2}$.  By a symmetric argument, $Z_j\cup\{(v_j,a_i)\} \succ Z_{j-2}$ for each odd $j\in\{3,\ldots,r\}$.  Finally, since $(v_{j},a_i)\prec (v_{j+2},a_i)$ for each odd $i\in\{1,\ldots,r\}$, we have $Z_{j}\cup\{(v_j,a_i)\} \prec Z_{j+2}\cup\{(v_{j+2},a_i)\}$ for each odd $j\in\{1,\ldots,r-2\}$.  Thus $A_i$ satisifies (C2) since the sets $Z_1\cup\{(v_1,a_i)\},Z_3\cup\{(v_3,a_i)\},\ldots,Z_{2\floor (r-1)/2\rfloor+1} \cup (v_{2\floor (r-1)/2\rfloor+1},a_i)$ are precisely the sets $Z_{i,1},\ldots,Z_{i,d_i}$ determined by our choice of $A_i$.
\end{proof}

\begin{lem}\lemlabel{twister}
    Let $G$ be any graph, let $\prec$ be any linear ordering of $V(G)$,  let $Z_{1}\prec\cdots\prec Z_{2s}$ be subsets of $V(G)$, and let $r_1\prec\cdots\prec r_{2s}$ be vertices of $G$ such that, for each $i,j\in\{1,\ldots,2s\}$, $G$ contains an edge $r_iz_j$ with $z_j\in Z_j$. Then $G$ contains a set of $s$ edges that are pairwise crossing with respect to $\prec$.
\end{lem}

\begin{proof}
    At least one of the following two cases applies:
    \begin{enumerate}
        \item $Z_s\prec r_{s+1}$ in which case the graph between $r_{s+1},\ldots,r_{2s}$ and $Z_1,\ldots,Z_s$ has a set of $s$ pairwise-crossing edges.
        \item $r_{s}\prec Z_{s+1}$ in which case the graph between $r_1,\ldots,r_s$ and $Z_{s+1},\ldots,Z_{2s}$ has a set of $s$ pairwise-crossing edges. \qedhere
    \end{enumerate}
\end{proof}


\end{document}
