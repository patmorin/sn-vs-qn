\documentclass[kpfonts]{patmorin}
\usepackage{pat}
\usepackage{paralist,graphicx,float}
\usepackage{array,longtable}

\usepackage[utf8]{inputenc}
\usepackage{todonotes}

\usepackage[noabbrev,capitalise]{cleveref}
\crefname{lem}{Lemma}{Lemmas}
\crefname{thm}{Theorem}{Theorems}
\crefname{cor}{Corollary}{Corollaries}
\crefname{prop}{Proposition}{Propositions}
\crefname{conj}{Conjecture}{Conjectures}
\crefname{open}{Open Problem}{Open Problems}
\crefname{obs}{Observation}{Observations}

\crefformat{equation}{(#2#1#3)}
\Crefformat{equation}{Equation #2(#1)#3}

\usepackage[numbers,sort&compress]{natbib}
%\usepackage{hypernat}
%\makeatletter
%\def\NAT@spacechar{~}
%\makeatother

\setlength{\parskip}{1ex}
\setlength{\parindent}{0ex}

\newtheorem{property}{Property}
\newcommand{\plabel}[1]{\label{prop:#1}}
\newcommand{\pref}[1]{Property~\ref{prop:#1}}

\DeclareMathOperator{\sn}{sn}
\DeclareMathOperator{\qn}{qn}
\DeclareMathOperator{\sqn}{sqn}
\DeclareMathOperator{\dsn}{dsn}
\DeclareMathOperator{\tw}{tw}

\renewcommand{\SS}{\mathcal{S}}

\renewcommand{\le}{\leqslant}
\renewcommand{\leq}{\leqslant}
\renewcommand{\ge}{\geqslant}
\renewcommand{\geq}{\geqslant}

\newcommand{\CartProd}{\mathbin{\square}}

\title{\MakeUppercase{Stack-Number is not Bounded by Queue-Number}}

\author{%
	Vida Dujmovi\'c,\!\!%
	\thanks{School of Computer Science and Electrical Engineering,
		University of Ottawa, Ottawa, Canada (\texttt{vida.dujmovic@uottawa.ca}).
		Research supported by NSERC.}
	\,\,
	David Eppstein,\!\!%
	\thanks{Department of Computer Science, University of California, Irvine, California, USA (\texttt{eppstein@uci.edu}).}
	\,\,
	Robert Hickingbotham,\!\!%
	\thanks{School of Mathematics, Monash University, Melbourne, Australia (\texttt{robert.hickingbotham@monash.edu}).}
	\,\,
	Pat Morin,\!\!%
	\thanks{School of Computer Science, Carleton University, Ottawa, Canada (\texttt{morin@scs.carleton.ca}). Research  supported by NSERC.}
	\,\,
	David R. Wood\thanks{School of Mathematics, Monash University, Melbourne, Australia (\texttt{david.wood@monash.edu}). Research supported by the Australian Research Council.}
}

\begin{document}
\maketitle

\begin{abstract}
We describe a family of graphs with queue-number at most 4 but unbounded stack-number. This resolves open problems of Heath, Leighton and Rosenberg (1992) and Blankenship and Oporowski (1999).
\end{abstract}

\bigskip

\section{Introduction}

Stacks and queues are fundamental data structures in computer science, but which is more powerful? In 1992, Heath, Leighton and Rosenberg~\cite{HLR92,HR92} introduced an approach for answering this question by defining the graph parameters \textit{stack-number} and \textit{queue-number} (defined below), which respectively measure the power of stacks and queues for representing graphs. The following fundamental problems, implicit in \citep{HLR92,HR92}, were made explicit by \citet{DujWoo05}\footnote{A \emph{graph parameter} is a function $\alpha$ such that $\alpha(G)\in\mathbb{R}$ for every graph $G$ and such that $\alpha(G_1)=\alpha(G_2)$ for all isomorphic graphs $G_1$ and $G_2$. A graph parameter $\alpha$ is \textit{bounded} by a graph parameter $\beta$ if there exists a function $f$ such that $\alpha(G) \leq f(\beta(G))$ for every graph $G$.}:
\begin{compactitem}
	\item Is stack-number bounded by queue-number?
	\item Is queue-number bounded by stack-number?
\end{compactitem}

If stack-number is bounded by queue-number but queue-number is not bounded by stack-number, then stacks would be considered to be more powerful than queues. Similarly, if the converse holds, then queues would be considered to be more powerful than stacks. Despite extensive research on stack- and queue-numbers, these fundamental questions have remained unsolved.

%A set of $k$ pairwise nested edges is called a \emph{$k$-rainbow} and a set of $k$ pairwise crossing edges is called a \emph{$k$-twist}. A set of $k$ edges, no pair of which nest and no pair of which cross is called a \emph{hopper}.

We now formally define stack- and queue-number. Let $G$ be a graph and let $\prec$ be a total order on $V(G)$.  Two disjoint edges $vw,xy\in E(G)$ with $v\prec w$ and $x\prec y$ \emph{cross} with respect to $\prec$ if $v\prec x\prec w\prec y$ or $x\prec v\prec y\prec w$, and \emph{nest} with respect to $\prec$ if $v\prec x\prec y\prec w$ or $x\prec v\prec w\prec y$. Let $\varphi:E(G)\to\{1,\ldots,k\}$ for some integer $k\ge 1$. Then $(\prec,\varphi)$ is a \emph{$k$-stack layout} of $G$ if $vw$ and $xy$ do not cross for all edges $vw,xy\in E(G)$ with $\varphi(vw) = \varphi(xy)$. Similarly, $(\prec,\varphi)$ is a \emph{$k$-queue layout} of $G$ if $vw$ and $xy$ do not nest for all edges $vw,xy\in E(G)$ with  $\varphi(vw)=\varphi(xy)$. See \cref{layouts} for examples. The smallest integer $s$ for which $G$ has an $s$-stack layout is called the \emph{stack-number} of $G$, denoted  $\sn(G)$. The smallest integer $q$ for which $G$ has a $q$-queue layout is called the \emph{queue-number} of $G$, denoted $\qn(G)$.

\begin{figure}[H]
	\centering
	\includegraphics{figs/layouts-1.pdf} \\[2ex]
	\includegraphics{figs/layouts-2.pdf}
	\caption{A $2$-queue layout and a $2$-stack layout of the triangulated grid graph $H_4$ defined below. Edges drawn above the vertices are assigned to the first queue/stack and edges drawn below the vertices are assigned to the second queue/stack.}
	\label{layouts}
\end{figure}

Given a $k$-stack layout $(\prec,\varphi)$ of a graph $G$, for each $i\in\{1,\dots,k\}$, the set $\varphi^{-1}(i)$ behaves like a stack, in the sense that each edge $vw \in \varphi^{-1}(i)$ with $v\prec w$ corresponds to an element in a sequence of stack operations, such that if we traverse the vertices in the order of $\prec$, then $vw$ is pushed onto the stack at $v$ and popped off the stack at $w$. Similarly, each set $\varphi^{-1}(i)$ in a queue layout  behaves like a queue. In this way, the stack-number and queue-number  respectively measure the power of stacks and queues to represent graphs.

Note that stack layouts are equivalent to book embeddings (first defined by \citet{Ollmann73} in 1973), and stack-number is also known as \emph{page-number}, \emph{book-thickness} or \emph{fixed outer-thickness}. Stack and queue layouts have other applications including computational complexity~\citep{GKS89,DSW16,Bourgain09,BY13},  RNA folding~\citep{HS99}, graph drawing in two~\citep{BB04,ADFPR12,SSSV-WG94} and three dimensions~\citep{DMW05,Wood-GD01,DMW17,DPW04},
and fault-tolerant multiprocessing~\citep{CLR87,Rosenberg83a,Rosenberg86a,Rosenberg86}.
%, stack sorting~\citep{CLR87}, and traffic light control~\citep{Kainen90}.
See \citep{BK79,Blankenship-PhD03,DujWoo04,DujWoo-DCG07,DJMMUW20,DFP13,BFGMMRU19,Yannakakis89,Yann20,MBKPRU20} for bounds on the stack- and queue-number for various graph classes.

\subsection*{Is Stack-Number Bounded by Queue-Number?}

This paper considers the first of the above questions. In a positive direction, \citet{HLR92}  showed that every 1-queue graph has a $2$-stack layout. On the other hand, they described graphs that need exponentially more stacks than queues. In particular, $n$-vertex ternary hypercubes have queue-number $O(\log n)$ and stack-number $\Omega(n^{1/9-\epsilon})$ for any $\epsilon>0$.

%Note that, in an $s$-stack layout, $(\prec,\varphi)$, $\varphi$ is a proper $s$-colouring of the auxilliary graph $H$ with vertex set $V(H)=E(G)$ and in which the edge $ef$ if present if and only if $e$ and $f$ cross with respect to $\varphi$.  Any $k$-twist in $G$ with respect to $\prec$ corresponds to a $k$-clique $H$.  Since the chromatic number of any graph is bounded by its clique number, the following observation is trivial:

%\begin{obs}\obslabel{no-s-twist}
%If $(\prec,\varphi)$ is an $s$-stack layout of a graph $G$ then $E(G)$ does not contain any $k$-twist for any $k>s$.
%\end{obs}

%We now formally define stack and queue layouts of graphs. Let $G=(V,E)$ be a graph with disjoint edges $vw, xy$ and a linear ordering $\leq$ of the vertices. Without loss of generality, we may assume that $v < w$, $x <y$ and $v < x$. We say that $vw$ and $xy$ \textit{cross} if $v<x<w<y$,  \textit{nest} if $v <x <y < w$, and are \textit{disjoint} if $v <w<x<y$. \todo{DW: Do we  need ``disjoint''?} A \textit{stack} is a set of pairwise non-crossing edges and a \textit{queue} is a set of pairwise non-nesting edges. A $k$-queue layout of $G$ consist of a linear ordering $\leq$ of its vertices and a partition $E_1,E_2, \dots, E_k$, of its edges into queues with respect to $\leq$. The stack-number of a graph $G$, $\sn(G)$, is the minimum integer $k$ such that $G$ has a $k$-stack layout. Similarly, the queue-number of a graph $G$, $\qn(G)$, is the minimum integer $k$ such that $G$ has a $k$-queue layout. We note that stack layouts of graphs are related to book embeddings of graph and that stack-number is also known as \textit{page-number}, \textit{book-thickness}, or \textit{fixed outer-thickness}.

%Note that stacks and queues are closely related to breadth-first search (BFS) and depth-first search (DFS) layouts of graphs. It can be easily shown that a BFS vertex ordering of a tree admits a $1$-queue layout and a DFS vertex ordering of a tree admits a $1$-stack layout.

%\citet{HLR92} showed that every 1-queue graph has a 2-stack layout. \citet{HLR92} showed that the ternary hypercubes requires exponentially more stacks than queues. In particular, $n$-vertex ternary hypercubes have queue-number at most $2 \log_3 n$, but stack-number at least $\Omega(n^{1/9-\epsilon})$ for any $\epsilon>0$. We prove the following theorem, which shows that stack-number is not bounded by queue-number.

Our key contribution is the following theorem, which shows that stack-number is not bounded by queue-number.

\begin{thm}\label{family}
	%  There exists a family $\mathcal{F}$ of graphs for which $\qn(G)\le 4$ for every $G\in\mathcal{F}$, and for every $s\in\N$, there exists $G\in\mathcal{F}$ for which $\sn(G)>s$.
	For every $s\in \N$ there exists a graph $G$ with $\qn(G)\le 4$ and $\sn(G)>s$.
\end{thm}

This demonstrates that stacks are not more powerful than queues for representing graphs.

\subsection*{Cartesian Products}

As illustrated in \cref{graph}, the graph $G$ in \cref{family} is the cartesian product\footnote{For graphs $G_1$ and $G_2$, the \emph{cartesian product} $G_1\CartProd G_2$ is the graph with vertex set $\{(v_1,v_2): v_1 \in V(G_1), v_2 \in V(G_2)\}$, where $(v_1,v_2)(w_1,w_2)\in E(G_1\CartProd G_2)$ if $v_1=w_1$ and $v_2w_2\in E(G_2)$, or $v_1w_1\in E(G_1)$ and $v_2=w_2$. The \emph{strong product} $G_1\boxtimes G_2$ is the graph obtained from $G_1\CartProd G_2$ by adding the edge $(v_1,v_2)(w_1,w_2)$ whenever $v_1w_1\in E(G_1)$ and $v_2w_2\in E(G_2)$. Note that \citet{Pupyrev20} independently suggested using graph products to show that stack-number is not bounded by queue-number.} $S_b\CartProd H_n$, where $S_b$ is the star graph with root $r$ and $b$ leaves, and $H_n$ is the dual of the hexagonal grid, defined by
\begin{align*}
V(H_n)  :=\{1,\ldots,n\}^2 \quad \text{ and } \quad
E(H_n) & :=  \{(x,y)(x+1,y):x\in\{1,\ldots,n-1\},\,y\in\{1,\ldots,n\}\} \\
& \qquad \cup \{(x,y)(x,y+1):x\in\{1,\ldots,n\},\,y\in\{1,\ldots,n-1\}\} \\
& \qquad \cup \{(x,y)(x+1,y+1):x,y\in\{1,\ldots,n-1\}\} \enspace .
\end{align*}

\begin{figure}[H]
	\centering
	\begin{tabular}{m{.3\textwidth}m{2ex}m{.2\textwidth}m{2ex}m{.3\textwidth}}
		\includegraphics[width=.3\textwidth]{figs/s} & $\CartProd$ & \includegraphics[width=.2\textwidth]{figs/q} & $=$
		& \includegraphics[width=.3\textwidth]{figs/product}
	\end{tabular}
	\caption{$S_9 \CartProd H_4$.}
	\label{graph}
\end{figure}


We prove the following:

\begin{thm}
\label{Main}
For every $s\in\N$, if $b$ and $n$ are sufficiently large compared to $s$, then $$\sn(S_b\CartProd H_n) > s.$$
\end{thm}

%Note that the Hex graph corresponds to the board used in the game \textit{Hex} which was invented by the  Peit Hein in 1942. In this paper, we prove \Cref{family} with the graph $G:= S_b \square H_n$ where $b$ and $n$ are large compared to $s$.

We now show that $\qn(S_b\CartProd H_n)\leq 4$, which with \cref{Main} implies \cref{family}. We need the following definition due to \citet{Wood-Queue-DMTCS05}. A queue layout $(\varphi,\prec)$ is \emph{strict} if for every vertex $u\in V(G)$ and for all neighbours $v,w\in N_G(u)$, if $u\prec v,w$ or $v,w \prec u$, then $\varphi(uv)\neq \varphi(uw)$. Let $\sqn(G)$ be the minimum integer $k$ such that $G$ has a strict $k$-queue layout. To see that $\sqn(H_n) \leq 3$, order the vertices row-by-row and then left-to-right within a row, with vertical edges in one queue, horizontal edges in one queue, and diagonal edges in another queue (this construction puts the edges below the vertices in \cref{layouts} into two queues). \citet{Wood-Queue-DMTCS05} proved that for all graphs $G_1$ and $G_2$,
\begin{equation}
\label{QueueProduct}
\qn(G_1 \CartProd G_2) \leq \qn(G_1) + \sqn(G_2).
\end{equation}
Of course, $S_b$ has a 1-queue layout (since no two edges are nested for any vertex-ordering). Thus $\qn(S_b \CartProd H_n)\leq 4$.

\citet{BK79} implicitly proved a result similar to \cref{QueueProduct} for stack layouts. Let $\dsn(G)$ be the minimum integer $k$ such that $G$ has a $k$-stack layout $(\prec,\varphi)$ where $\varphi$ is a proper edge-colouring of $G$; that is, $\varphi(vx)\neq\varphi(vy)$ for any two edges $vx,vy$ with a common endpoint. Then for every graph $G_1$ and every bipartite graph $G_2$,
\begin{equation}
\label{StackProduct}
\sn(G_1 \CartProd G_2) \leq \sn(G_1) + \dsn(G_2).
\end{equation}
The key difference between \cref{QueueProduct} and \cref{StackProduct} is that $G_2$ is assumed to be bipartite in \cref{StackProduct}. \cref{Main} says that this assumption is essential, since it is easily seen that $\dsn(H_n)$ is bounded, but $\sn(S_b \CartProd H_n)$ is unbounded by \cref{Main}.

%\todo{are there any results of \citet{Pupyrev20} worth mentioning here?}

\subsection*{Subdivisions}

A noteworthy consequence of \Cref{family} is that it resolves a conjecture of \citet{BO99}. A graph $G'$ is a \textit{subdivision} of a graph $G$ if $G'$ can be obtained from $G$ by replacing the edges $vw$ of $G$ by internally disjoint paths $P_{vw}$ with endpoints $v$ and $w$. If each $P_{vw}$ has exactly $k$ internal vertices, then $G'$ is the \emph{$k$-subdivision} of $G$. If each $P_{vw}$ has at most $k$ internal vertices, then $G'$ is a \emph{$(\leq k)$-subdivision} of $G$. \citet{BO99} conjectured that the stack-number of $(\leq k)$-subdivisions ($k$ fixed)  is not much less than the stack-number of the original graph. More precisely:

\begin{conj}[\citep{BO99}]
\label{B_conj}
There exists a function $f$ such that for every graph $G$ and integer $k$, if $G'$ is any $(\leq k)$-subdivision of $G$, then $\sn(G) \leq f(\sn(G’),k)$.
\end{conj}

\citet{DujWoo05} established a connection between this conjecture and the question of whether stack-number is bounded by queue-number. In particular, they showed that if
\cref{B_conj} were true, then stack-number would be bounded by queue-number. Since \cref{family} shows that stack-number is not bounded by queue-number, \cref{B_conj} is false. The proof of \citet{DujWoo05} is based on the following key lemma: every graph $G$ has a $3$-stack subdivision with $1+2 \ceil{\log_2\qn(G)}$ division vertices per edge. Applying this result to the graph $G=S_b\CartProd H_n$ in \cref{family},
the $5$-subdivision of $S_b\CartProd H_n$ has a $3$-stack layout. If \cref{B_conj} were true, then  $\sn(S_b\CartProd H_n) \leq f( 3,5)$, contradicting \cref{family}.

%Specifically,
%\[
%    \mathcal{F} := \{ S_b\square H_n : b,n\in\N\}
%\]
%where $S_b$ denotes the star with $b$ leaves and $H_n$ is the triangulated $n\times n$ grid.

%\todo[inline]{PM: Does anyone know if there is a standard box operator that is typeset like this $S\boxtimes H$ or $S\boxdot H$ instead of like this $S\square H$ or like this $S\Box H$?  I tried square and Box. DW: I defined \texttt{CartProd} which typesets okay $A \CartProd B$	.}


%The graph $G$ in \cref{family} is obtained as follows (See \figref{graph}): Let $S_b$ denote the star graph with root $r$ and $b$ leaves.  For an even positive integer $n$, let $H_n$ be the $n\times n$ triangulated grid, defined by $V(H_n):=\{1,\ldots,n\}^2$ and
%\begin{align*}
%E(H_n) & :=\{(x,y)(x+1,y):x\in\{1,\ldots,n-1\},\,y\in\{1,\ldots,n\}\} \\
%& \qquad \cup \{(x,y)(x,y+1):x\in\{1,\ldots,n\},\,y\in\{1,\ldots,n-1\}\} \\
%& \qquad \cup \{(x,y+1)(x+1,y):x,y\in\{1,\ldots,n-1\}\} \enspace .
%\end{align*}
%We consider the graph $G:=S_b\CartProd H_n$. That is, $V(G)=V(S_b)\times V(H_n)$ where vertices $(v_1,w_1),(v_2,w_2)\in V(G)$ are adjacent whenever $v_1=w_1$ and $v_2w_2\in E(H_n)$, or $v_1w_1\in E(S_b)$ and $v_2=w_2$.


%SAY SOMETHING ABOUT THE RESULTS OF \citet{Pupyrev20}. I EXPECT WE SOLVE SOME OPEN PROBLEM HERE.\todo{PM:Not really, his open problem is about $H\boxtimes P$ where $H$ has bounded treewidth.}

\subsection*{Is Queue-Number Bounded by Stack-Number? }

It remains open whether queues are more powerful than stacks; that is, whether queue-number is bounded by stack-number. Several reults are known about this problem. \citet{HLR92} showed that every 1-stack graph has a 2-queue layout. \citet{DJMMUW20} showed that planar graphs have bounded queue-number. (Note that graph products also feature heavily in this proof.)\ Since 2-stack graphs are planar, this implies that 2-stack graphs have bounded queue-number. It is open whether 3-stack graphs have bounded queue-number. In fact, the case of three stacks is as hard as the general question. \citet{DujWoo05} proved that queue-number is bounded by stack-number if and only if 3-stack graphs have bounded queue-number. Moreover, if this is true then queue-number is bounded by a polynomial function of stack-number.

\section{Proof of \cref{Main}}

We now turn to the proof of our main result, the lower bound on $\sn(G)$, where $G:= S_b\CartProd H_n$. Consider a hypothetical $s$-stack layout $(\varphi,\prec)$ of $G$ where $n$ and $b$ are chosen sufficiently large compared to $s$ as detailed below. We begin with three lemmata that, for sufficiently large $b$, provide a large subgraph $S_d$ of $S_b$ for which the induced stack layout of $S_d\CartProd H_n$ is highly structured.

For each node $v$ of $S_b$, define $\pi_v$ as the permutation of $\{1,\ldots,n\}^2$ in which $(x_1,y_1)$ appears before $(x_2,y_2)$ if and only if $(v,(x_1,y_1))\prec (v,(x_2,y_2))$. The following lemma is an immediate consequence of the Pigeonhole Principle:

\begin{lem}\label{uniform_order}
    There exists a permutation $\pi$ of $\{1,\ldots,n\}^2$ and a set $L_1$ of leaves of $S_b$ of size $a\ge b/(n^2)!$ such that $\pi_{v}=\pi$ for each $v\in L_1$.
\end{lem}

% \todo[inline]{If we cared, we could improve this to $b/2^{cn^2}$ since we only use the weaker property (P3) in the final proof. DW: I don't care. }

For each leaf $v$ in $L_1$, let $\varphi_v$ be the edge colouring of $H_n$ defined by $\varphi_v(xy):=\varphi((v,x)(v,y))$ for each $xy\in E(H_n)$. Since $H_n$ has maximum degree $6$ and is not 6-regular, it has fewer than $3n^2$ edges.  Therefore there are fewer than $s^{3n^2}$ edge colourings of $H_n$ using $s$ colours.  Another application of the Pigeonhole Principle proves the following:

\begin{lem}\label{uniform_colour}
    There exists a subset $L_2\subseteq L_1$ of size $c\ge a/s^{3n^2}$
    and an edge colouring $\phi:E(H_n)\to\{1,\ldots,s\}$ such that $\varphi_v=\phi$ for each $v\in L_2$.
\end{lem}


Let $S_{c}$ be the subgraph of $S_b$ induced by $L_2\cup\{r\}$. The preceding two lemmata ensure that, for distinct leaves $v$ and $w$ of $S_{c}$, the stack layouts of the isomorphic graphs $G[\{(v,p):p\in V(H_n)\}]$ and $G[\{(w,p):p\in V(H_n)\}]$ are identical. The next lemma is a statement about the relationships between the stack layouts of $G[\{(v,p):v\in V(S_{c})\}]$ and $G[\{(v,q):v\in V(S_{c})\}]$ for  distinct $p,q\in V(H_n)$.  It does not assert that these two layouts are identical but it does state that they fall into one of two categories.

\begin{lem}\label{forward_or_backward}
    There exists a sequence $u_1,\ldots,u_{d}\in L_2$ of length $d\ge c^{1/2^{n^2-1}}$ such that, for each $p\in V(H_n)$, either  $(u_1,p)\prec (u_2,p)\prec\cdots\prec (u_{d},p)$ or $(u_1,p)\succ (u_2,p)\succ\cdots\succ (u_{d},p)$.
\end{lem}

\begin{proof}
    Let $p_1,\ldots,p_{n^2}$ denote the vertices of $H_n$ in any order.
    Begin with the sequence $S_1:=v_{1,1},\ldots,v_{1,c}$ that contains all $c$ elements of $L_2$ ordered so that $(v_{1,1},p_1)\prec\cdots\prec(v_{1,c},p_1)$.  For each $i\in\{2,\ldots,n^2\}$, the Erd\H{o}s-Szekeres Theorem~\citep{ES35} implies that $S_{i-1}$ contains a subsequence $S_i:=v_{i,1},\ldots,v_{i,|S_i|}$ of length $|S_i|\ge \sqrt{|S_{i-1}|}$ such that $(v_{i,1},p_i)\prec\cdots\prec(v_{i,|S_i|},p_i)$ or $(v_{i,1},p_i)\succ\cdots\succ(v_{i,|S_i|},p_i)$.  It is straightforward to verify by induction on $i$ that $|S_i| \ge c^{1/2^{i-1}}$ resulting in a final sequence $S_{n^2}=:L_3$ of length at least $c^{1/2^{n^2-1}}$.
\end{proof}


For the rest of the proof we work with the star $S_d$ whose leaves are $u_1,\ldots,u_d$ described in \cref{forward_or_backward}.  Consider the (improper) colouring of $H_n$ obtained by colouring each vertex $p\in V(H_n)$ \emph{red} if $(u_1,p)\prec\cdots\prec (u_d,p)$ and colouring $p$ \emph{blue} if $(u_1,p)\succ\cdots\succ(u_d,p)$. We need the following famous Hex Lemma~\citep{Gale79}.

\begin{lem}[\citep{Gale79}] \label{hex_lemma}
%Every red--blue vertex colouring of the graph $H_n$	contains an $n$-vertex path $R$ consisting entirely of red vertices or entirely of blue vertices.
Every vertex 2-colouring of $H_n$ contains a monochromatic path on $n$ vertices.
\end{lem}

%\begin{proof}
%    The dual of $H_n$ is the board on which the game Hex is played.  The well-known \emph{Hex Lemma} states that any colouring of the vertices of $H_n$ with colours red and blue contains exactly one of the following \cite{Gale79}:
%    \begin{compactenum}
%        \item a path with endpoints $(x,1)$ and $(x',n)$ consisting entirely of red vertices, for some $x,x'\in\{1,\ldots,n\}$; or
%        \item a path with endpoints $(1,y)$ and $(n,y')$ consisting entirely of blue vertices, for some $y,y'\in\{1,\ldots,n\}$.
%    \end{compactenum}
%    In either case, the path contains at least $n$ vertices and therefore has a $n$-vertex subpath consisting entirely of red vertices or entirely of blue vertices.
%\end{proof}

Apply \cref{hex_lemma} with the above-defined colouring of $H_n$. We obtain a path $R:=p_1,\ldots,p_n$ in $H_n$ that, without loss of generality, consists entirely of red vertices;
thus $(u_1,p_j)\prec\cdots\prec (u_d,p_j)$ for each $j\in\{1,\ldots,n\}$.  Let $X$ be the subgraph $S_d \CartProd R$ of $G$.

\begin{lem}\label{twister}
$X$ contains a set of at least $\min\{\lfloor d/2^{n}\rfloor,\lceil n/2\rceil\}$ pairwise crossing edges with respect to $\prec$.
\end{lem}

\begin{proof}
	Extend the total order $\prec$ to a partial order over subsets of $V(G)$, where for all $V,W\subseteq V(G)$, we have $V\prec W$ if and only if $v\prec w$ for each $v\in V$ and each $w\in W$.  We abuse notation slightly by using $\prec$ to compare elements of $V(G)$ and subsets of $V(G)$ so that, for $v\in V(G)$ and $V\subseteq V(G)$, $v\prec V$ denotes $\{v\}\prec V$.
    We will define sets $A_1\supseteq \cdots\supseteq A_{n}$ of leaves of $S_d$ so that each $A_i$ satisifies the following conditions:
    \begin{compactenum}[(C1)]
        \item $A_i$ contains $d_i\ge d/2^{i-1}$ leaves of $S_d$.
        \item Each leaf $v\in A_i$ defines an $i$-element vertex set $Z_{i,v}:=\{(v,p_j):j\in\{1,\ldots,i\}\}$.  For any distinct $v,w\in A_i$, the sets $Z_{i,v}$ and $Z_{i,w}$ are \emph{separated} with respect to $\prec$; that is, $Z_{i,v}\prec Z_{i,w}$ or $Z_{i,v}\succ Z_{i,w}$.
    \end{compactenum}

    Before defining $A_1,\ldots,A_n$ we first show how the existence of the set $A_n$ implies the lemma.  To avoid triple-subscripts, let $d':=d_n\ge d/2^{n-1}$.   The set $A_n$ defines vertex sets $Z_{n,v_1}\prec\cdots\prec Z_{n,v_{d'}}$ (see \cref{fig_twister}). Recall that $r$ is the root of $S_b$ so it is adjacent to each of $v_{1},\ldots,v_{d'}$ in $S_d$.  Therefore, for each $j\in\{1,\ldots,n\}$ and each $i\in\{1,\ldots,d'\}$, the edge $(r,p_j)(v_i,p_j)$ is in $X$. Therefore, $(r,p_j)$ is adjacent to an element of each of $Z_{n,v_1},\ldots,Z_{n,v_{d'}}$.
	\begin{figure}[!h]
		\centering\includegraphics{figs/twister}
		\caption{The sets $Z_{n,v_1},\ldots,Z_{n,v_{d'}}$ where $n=4$ and $d'=5$.}
		\label{fig_twister}
	\end{figure}

    Since $Z_{n,v_1},\ldots,Z_{n,v_{d'}}$ are separated with respect to $\prec$, when viewed from afar, this situation looks like a complete bipartite graph $K_{n,d'}$ with the root vertices $L:=\{(r,p_j):j\in\{1,\ldots,n\}\}$ in one part and the groups $R:=Z_{n,v_1}\cup\cdots\cup Z_{n,v_{d'}}$ in the other part.  Any linear ordering of $K_{n,d'}$ has a large set of pairwise crossing edges so, intuitively, the induced subgraph $X[L\cup R]$ should also have a large set of pairwise crossing edges. We can formalize this as follows: Label the vertices in $L$ as $r_1,\ldots,r_n$ so that $r_1\prec \cdots\prec r_{n}$.  Then at least one of the following two cases applies (see \figref{median}):
    \begin{enumerate}
        \item $Z_{n,\lfloor d'/2\rfloor}\prec r_{\lceil n/2\rceil}$ in which case the graph between $r_{\lceil n/2\rceil},\ldots,r_{n}$ and $Z_{n,1},\ldots,Z_{n,\lfloor d'/2\rfloor}$ has a set of at least $\min\{\lfloor d'/2\rfloor,\lceil n/2\rceil\}$ pairwise-crossing edges.
        \item $r_{\lceil n/2\rceil}\prec Z_{\lceil d'/2\rceil+1}$ in which case the graph between $r_1,\ldots,r_{\lceil n/2\rceil}$ and $Z_{\lceil d'/2\rceil+1},\ldots,Z_{d'}$ has a set of $\min\{\lfloor d'/2\rfloor,\lceil n/2\rceil\}$ pairwise-crossing edges.
    \end{enumerate}
	\begin{figure}[!h]
		\centering
			\includegraphics{figs/median-1} \\ 1 \\[2em]
			\includegraphics{figs/median-2} \\ 2
		\caption{The two cases in the proof of \cref{twister}.}
		\figlabel{median}
	\end{figure}
\todo{DW: I would delete the ``1'' and ``2' in the above figure.}
	Since, by (C1), $d'\ge d/2^{n-1}$, either case results in a set of pairwise-crossing edges of size at least $\min\{\lfloor d/2^n\rfloor,\lceil n/2\rceil\}$, as claimed.

    All that remains is to define the sets $A_1\supseteq\cdots\supseteq A_n$ that satisfy (C1) and (C2).  Let $A_1$ be the set of all the leaves of $S_d$.  For each $i\in\{2,\ldots,n\}$, the set $A_i$ is defined as follows:  Let $Z_1,\ldots,Z_{|A_{i-1}|}$ denote the sets $Z_{i-1, v}$ for each $v\in A_{i-1}$ ordered so that $Z_1\prec\cdots\prec Z_{|A_{i-1}|}$. By Property (C2), this is always possible.	Label the vertices of $A_{i-1}$ as $v_1,\ldots,v_{|A_{i-1}|}$ so that $(v_1,p_{i-1})\prec\cdots\prec (v_r,p_{i-1})$.   (This is equivalent to naming them so that $(v_j,p_{i-1})\in Z_j$ for each $j\in\{1,\ldots,|A_{i-1}|\}$.)  Define the set $A_i:=\{v_{2k+1}:k\in\{0,\ldots,\lfloor(|A_{i-1}|-1)/2\rfloor\}\}=\{v_{j}\in A_{i-1}:\text{$j$ is odd}\}$.  This completes the definition of $A_1,\ldots,A_n$.

	All that remains is to verify that $A_i$ satisfies (C1) and (C2) for each $i\in\{1,\ldots,n\}$.  We do this by induction on $i$. The base case $i=1$ is trivial so we assume from this point on that $i\in\{2,\ldots,n\}$.   To see that $A_i$ satisfies (C1) just observe that $|A_i|=\lceil |A_{i-1}|/2\rceil \ge |A_{i-1}|/2\ge d/2^{i-1}$, where the final inequality follows by applying the inductive hypothesis $|A_{i-1}|\ge d/2^{i-2}$.  Now all that remains is to show that $A_i$ satisfies (C2).

    Recall that, for each $v\in A_{i-1}$, the edge $e_v:=(v,p_{i-1})(v,p_i)$ is in $X$.  We have the following properties:
    \begin{compactenum}[(P1)]
        \item By \cref{uniform_colour}, $\varphi(e_v)=\phi(p_{i-1}p_i)$ for each $v\in A_{i-1}$.
        \item Since $p_{i-1}$ and $p_i$ are both red, $(v,p_{i-1})\prec (w,p_{i-1})$ if and only if $(v,p_{i})\prec (w,p_{i})$ for each $v,w\in A_{i-1}$.
        \item By \cref{uniform_order}, $(v,p_{i-1})\prec (v,p_i)$ for every $v\in A_{i-1}$ or $(v,p_{i-1})\succ (v,p_i)$ for every $v\in A_{i-1}$.
    \end{compactenum}
    We claim that these three conditions imply that the vertex sets $\{(v,p_{i-1}):v\in A_{i-1}\}$ and $\{(v,p_i):v\in A_{i-1}\}$ interleave perfectly with respect to $\prec$. More precisely:
	\begin{clm}\clmlabel{interleave} $(v_1,p_{i-1+t})\prec (v_1,p_{i-t}) \prec (v_2,p_{i-1+t}) \prec (v_2,p_{i-t}) \cdots \prec (v_r,p_{i-1+t}) \prec (v_r,p_{i-t})$ for some $t\in\{0,1\}$.
	\end{clm}
	\begin{proof}[Proof of \clmref{interleave}]
		By (P3) we may assume, without loss of generality, that $(v,p_{i-1})\prec (v,p_i)$ for each $v\in A_{i-1}$, in which case we are trying to prove the claim for $t=0$.  Therefore, it is sufficient to show that $(v_j,p_i)\prec (v_{j+1},p_{i-1})$ for each $j\in\{1,\ldots,r-1\}$.  For the sake of contradiction, suppose $(v_j,p_{i})\succ (v_{j+1},p_{i-1})$ for some $j\in\{1,\ldots,r-1\}$. By the labelling of $A_{i-1}$,  $(v_j,p_{i-1})\prec (v_{j+1},p_{i-1})$ so, by (P2),  $(v_{j},p_i) \prec (v_{j+1},p_i)$.  Therefore
		\[
			(v_j,p_{i-1})\prec (v_{j+1},p_{i-1})\prec(v_{j},p_i) \prec
		   (v_{j+1}, p_i) \enspace .
	   	\]
		Therefore the edges $e_{v_j}=(v_j,p_{i-1})(v_j,p_{i})$ and $e_{v_{j+1}}=(v_{j+1},p_{i-1})(v_{j+1},p_i)$ cross with respect to $\prec$.  But this is a contradiction since, by (P1),  $\varphi(e_{v_j}) =\varphi(e_{v_{j+1}})=\phi(p_{i-1}p_i)$.
		This contradiction completes the proof of \clmref{interleave}.
	\end{proof}

% \todo{DW: Why are these $\prec$'s red?  PM: Just me keeping track of which one was the assumption, they don't need to be red.}

We now complete the proof that $A_i$ satisfies (C2). Apply \clmref{interleave} and assume without loss of generality that $t=0$, so that
	\[
		(v_1,p_{i-1})\prec (v_1,p_{i}) \prec (v_2,p_{i-1}) \prec (v_2,p_{i}) \cdots \prec (v_r,p_{i-1}) \prec (v_r,p_{i}) \enspace .
	\]

    For each $j\in\{1,\ldots,r-2\}$, we have $(v_{j+1},p_{i-1})\in Z_{j+1}\prec Z_{j+2}$, so  $(v_j,p_i)\prec (v_{j+1},p_{i-1}) \prec Z_{j+2}$.  Therefore $Z_j\cup\{(v_j,p_i)\} \prec Z_{j+2}$.  By a symmetric argument, $Z_j\cup\{(v_j,p_i)\} \succ Z_{j-2}$ for each  $j\in\{3,\ldots,r\}$.  Finally, since $(v_{j},p_i)\prec (v_{j+2},p_i)$ for each odd $i\in\{1,\ldots,r\}$, we have $Z_{j}\cup\{(v_j,p_i)\} \prec Z_{j+2}\cup\{(v_{j+2},p_i)\}$ for each odd $j\in\{1,\ldots,r-2\}$.  Thus $A_i$ satisfies (C2) since the sets $Z_1\cup\{(v_1,p_i)\},Z_3\cup\{(v_3,p_i)\},\ldots,Z_{2\lfloor (r-1)/2\rfloor+1} \cup (v_{2\lfloor (r-1)/2\rfloor+1},p_i)$ are precisely the sets $Z_{i,1},\ldots,Z_{i,d_i}$ determined by our choice of $A_i$.
\end{proof}

% \begin{lem}\lemlabel{twister}
%     Let $G$ be any graph, let $\prec$ be any linear ordering of $V(G)$,  let $Z_{1}\prec\cdots\prec Z_{2s}$ be subsets of $V(G)$, and let $r_1\prec\cdots\prec r_{2s}$ be vertices of $G$ such that, for each $i,j\in\{1,\ldots,2s\}$, $G$ contains an edge $r_iz_j$ with $z_j\in Z_j$. Then $G$ contains a set of $s$ edges that are pairwise crossing with respect to $\prec$. \todo[inline]{I think we should not re-use $s$ in this lemma. More importantly, do we really need Lemma 6? It could be easily merged into the proof of Lemma 5 where Lemma 6 is used, and this would avoid having to translate notation. It took me a while to realise that $r_1,\dots,r_{2s}$ corresponds to $L$ in Lemma 5.}
% \end{lem}
%
% \begin{proof}
% \end{proof}
%

\begin{proof}[Proof of \cref{Main}]
Let $G := S_b \CartProd H_n$, where $n :=2s+1$ and $b := (n^2)!\, s^{3n^2}\, ((s+1)2^n)^{2^{n^2-1}} $. Suppose that $G$ has an $s$-stack layout  $(\varphi,\prec)$. In particular, there are no $s+1$ pairwise crossing edges in $G$ with respect to $\prec$. By \cref{uniform_colour,uniform_order,forward_or_backward}, we have $a\ge b/(n^2)! = s^{3n^2}\, ((s+1)2^n)^{2^{n^2-1}}$ and $c\ge a/s^{3n^2} \geq ((s+1)2^n)^{2^{n^2-1}}$ and
$d\ge c^{1/2^{n^2-1}} \ge (s+1)2^n$. By \cref{twister}, the graph $X$, which is a subgraph of $G$, contains $\min\{\lfloor d/2^{n}\rfloor,\lceil n/2\rceil\}=s+1$ pairwise crossing edges with respect to $\prec$. This contradictions shows that $\sn(G)> s$.
\end{proof}

\section{Open Problems}

Recall that every 1-queue graph has a 2-stack layout \citep{HLR92} and we proved that there are 4-queue graphs with unbounded stack-number. The following questions remain open: Do 2-queue graphs have bounded stack-number? Do 3-queue graphs have bounded stack-number? \todo{DW: Do we want to make a conjecture here? I would say ``no'' for both questions.}

Given the role of cartesian products in our proof, it is natural to ask when is $\sn(G_1\CartProd G_2)$ bounded? As illustrated in \cref{layouts}, $\sn(H_n) \leq 2$. So $\sn(G_1\CartProd G_2)$ can be unbounded even when $G_1$ is a star and $\sn(G_2)\leq 2$. Since $\sn(G_2)\leq 1$ if and only if $G_2$ is outerplanar, the following questions naturally arise: Is $\sn(S \CartProd H)$ bounded for every star $S$ and outerplanar graph $H$ with bounded degree? Is $\sn(T \CartProd H)$ bounded for every tree $T$ and outerplanar graph $H$ with bounded degree? The assumption that $H$ has bounded degree is needed since $S_n \CartProd S_n$ contains the 1-subdivision of $K_{n,n}$, which has unbounded stack-number~\citep{Blankenship-PhD03}.

Since $H_n\subseteq P \boxtimes P$ where $P$ is the $n$-vertex path, \cref{family} implies that $\sn(S\boxtimes P\boxtimes P)$ is unbounded for stars $S$ and paths $P$. It is easily seen that $\sn(S\boxtimes P)$ is bounded~\citep{Pupyrev20}. The following question naturally arises (independently asked by \citet{Pupyrev20}):
Is $\sn(T \boxtimes P)$ bounded for every tree $T$ and path $P$? We conjecture the answer is ``no''.

\let\oldthebibliography=\thebibliography
\let\endoldthebibliography=\endthebibliography
\renewenvironment{thebibliography}[1]{%
\begin{oldthebibliography}{#1}%
\setlength{\parskip}{-0.1ex}%
\setlength{\itemsep}{-0.1ex}%
}{\end{oldthebibliography}}

%\documentclass[kpfonts]{patmorin}
\usepackage{pat}
\usepackage{paralist,graphicx}
\usepackage{array,longtable}

\usepackage[utf8]{inputenc}
\usepackage{todonotes}

\usepackage[noabbrev,capitalise]{cleveref}
\crefname{lem}{Lemma}{Lemmas}
\crefname{thm}{Theorem}{Theorems}
\crefname{cor}{Corollary}{Corollaries}
\crefname{prop}{Proposition}{Propositions}
\crefname{conj}{Conjecture}{Conjectures}
\crefname{open}{Open Problem}{Open Problems}
\crefname{obs}{Observation}{Observations}

\crefformat{equation}{(#2#1#3)}
\Crefformat{equation}{Equation #2(#1)#3}

\usepackage[numbers,sort&compress]{natbib}
\usepackage{hypernat}
\makeatletter
\def\NAT@spacechar{~}
\makeatother

\setlength{\parskip}{1ex}
\setlength{\parindent}{0ex}


\newtheorem{property}{Property}
\newcommand{\plabel}[1]{\label{prop:#1}}
\newcommand{\pref}[1]{Property~\ref{prop:#1}}

\DeclareMathOperator{\sn}{sn}
\DeclareMathOperator{\qn}{qn}
\DeclareMathOperator{\sqn}{sqn}
\DeclareMathOperator{\tw}{tw}

\renewcommand{\SS}{\mathcal{S}}

\renewcommand{\le}{\leqslant}
\renewcommand{\leq}{\leqslant}
\renewcommand{\ge}{\geqslant}
\renewcommand{\geq}{\geqslant}

\newcommand{\CartProd}{\mathbin{\square}}

\title{\MakeUppercase{Stack-Number is not Bounded by Queue-Number}}

\author{%
	Vida Dujmovi\'c,\!\!%
	\thanks{School of Computer Science and Electrical Engineering,
		University of Ottawa, Ottawa, Canada (\texttt{vida.dujmovic@uottawa.ca}).
		Research supported by NSERC and the Ontario Ministry of Research and Innovation.}
	\,\,
	Robert Hickingbotham,\!\!%
	\thanks{School of Mathematics, Monash University, Melbourne, Australia (\texttt{robert.hickingbotham@monash.edu}).}
	\,\,
	Pat Morin,\!\!%
	\thanks{School of Computer Science, Carleton University, Ottawa, Canada (\texttt{morin@scs.carleton.ca}). Research  supported by NSERC and the Ontario Ministry of Research and Innovation.}
	\,\,
	David R. Wood\thanks{School of Mathematics, Monash University, Melbourne, Australia (\texttt{david.wood@monash.edu}). Research supported by the Australian Research Council.}
}

\begin{document}
\maketitle

\begin{abstract}
We describe a family of graphs with queue-number at most 4 but unbounded stack-number. This resolves open problems of Heath, Leighton and Rosenberg (1992) and Blankenship and Oporwoski (1999).
\end{abstract}

\section{Introduction}

Stacks and queues are fundamental data structures in computer science, but which is more powerful? In 1992, \citet{HLR92} introduced an approach for answering this question by defining the graph parameters \textit{stack-number} and \textit{queue-number}, which respectively measure the power of stacks and queues for representing graphs.

%We now formally define stack and queue layouts of graphs.
Let $G$ be a graph and let $\prec$ be a total order on $V(G)$.  Two disjoint edges $vw,xy\in E(G)$ with $v\prec w$ and $x\prec y$ \emph{cross} with respect to $\prec$ if $v\prec x\prec w\prec y$ or $x\prec v\prec y\prec w$, and \emph{nest} with respect to $\prec$ if $v\prec x\prec y\prec w$ or $x\prec v\prec w\prec y$.
%A set of $k$ pairwise nested edges is called a \emph{$k$-rainbow} and a set of $k$ pairwise crossing edges is called a \emph{$k$-twist}. A set of $k$ edges, no pair of which nest and no pair of which cross is called a \emph{hopper}.
Let $\varphi:E(G)\to\{1,\ldots,k\}$ for some integer $k\ge 1$.  Then $(\prec,\varphi)$ is a \emph{$k$-stack layout} of $G$ if, for every pair of edges $vw,xy\in E(G)$, if $\varphi(vw) = \varphi(xy)$ then $vw$ and $xy$ do not cross.\todo{PM: Suggestion: Replace second if then with $\varphi(vw)\neq\varphi(xy)$ or $vw$ and $xy$ do not cross.} Similarly, the pair $(\prec,\varphi)$ is a \emph{$k$-queue layout} of $G$ if, for every pair of edges $vw,xy\in E(G)$, if $\varphi(vw)=\varphi(xy)$ then  $vw$ and $xy$ do not nest. The smallest integer $s$ for which $G$ has an $s$-stack layout is called the \emph{stack-number} of $G$, denoted  $\sn(G)$. The smallest integer $q$ for which $G$ has a $q$-queue layout is called the \emph{queue-number} of $G$, denoted $\qn(G)$. Note that stack layouts are equivalent to book embeddings, and stack-number is also known as \emph{page-number}, \emph{book-thickness} or \emph{fixed outer-thickness}. \todo{add references}

Given a $k$-stack layout $(\prec,\varphi)$ of a graph $G$, for each $i\in\{1,\dots,k\}$, the set $E_i:= \{e\in E(G):\varphi(e)=i\}$ behaves like a stack, in the sense that each edge $e=vw \in E_i$ with $v\prec w$ corresponds to an element in a sequence of stack opertions, such that if we traverse the vertices in the order of $\prec$, then $e$ is pushed onto the stack at $v$ and popped off the stack at $w$. Similarly, given a $k$-queue layout $(\prec,\varphi)$, each set $E_i$ behaves like a queue. In this way, the stack-number and queue-number  respectively measure the power of stacks and queues to represent graphs.

\todo{Should the following two paragraphs be moved to prior to the definitions\newline PM: Yes}

The following key problems are implcit in the work of \citet{HLR92}, made explicit by \citet{DujWoo05}
\footnote{A \emph{graph parameter} is a function $\alpha$ such that $\alpha(G)\in\mathbb{R}$ for every graph $G$, such that $\alpha(G_1)=\alpha(G_2)$ for all isomorphic graphs $G_1$ and $G_2$. A graph parameter $\alpha$ is \textit{bounded} by a graph parameter $\beta$ if there exists a function $f$ such that for every graph $G$ we have $\alpha(G) \leq f(\beta(G))$.}:
\begin{compactitem}
	\item Is stack-number bounded by queue-number?
	\item Is queue-number bounded by stack-number?
\end{compactitem}

If stack-number is bounded by queue-number but queue-number is not bounded by stack-number, then we would consider stacks to be more powerful than queues. Similarly, if the converse holds, then we would consider queues to be more powerful than stacks. Despite extensive research on stack and queue layouts of graphs (see \citep{DujWoo04,DujWoo-DCG07,DJMMUW20} and the references therein), these fundamental questions have remained unsolved.

\subsection*{Is Stack-Number Bounded by Queue-number?}

This paper considers the first of the above questions. In a positive direction, \citet{HLR92}  showed that every 1-queue graph has a $2$-stack layout. On the other hand, they described graphs that need exponentially more stacks than queues. In particular, $n$-vertex ternary hypercubes have queue-number $O(\log n)$ and stack-number $\Omega(n^{1/9-\epsilon})$ for any $\epsilon>0$.

%Note that, in an $s$-stack layout, $(\prec,\varphi)$, $\varphi$ is a proper $s$-colouring of the auxilliary graph $H$ with vertex set $V(H)=E(G)$ and in which the edge $ef$ if present if and only if $e$ and $f$ cross with respect to $\varphi$.  Any $k$-twist in $G$ with respect to $\prec$ corresponds to a $k$-clique $H$.  Since the chromatic number of any graph is bounded by its clique number, the following observation is trivial:

%\begin{obs}\obslabel{no-s-twist}
%If $(\prec,\varphi)$ is an $s$-stack layout of a graph $G$ then $E(G)$ does not contain any $k$-twist for any $k>s$.
%\end{obs}

%We now formally define stack and queue layouts of graphs. Let $G=(V,E)$ be a graph with disjoint edges $vw, xy$ and a linear ordering $\leq$ of the vertices. Without loss of generality, we may assume that $v < w$, $x <y$ and $v < x$. We say that $vw$ and $xy$ \textit{cross} if $v<x<w<y$,  \textit{nest} if $v <x <y < w$, and are \textit{disjoint} if $v <w<x<y$. \todo{DW: Do we  need ``disjoint''?} A \textit{stack} is a set of pairwise non-crossing edges and a \textit{queue} is a set of pairwise non-nesting edges. A $k$-queue layout of $G$ consist of a linear ordering $\leq$ of its vertices and a partition $E_1,E_2, \dots, E_k$, of its edges into queues with respect to $\leq$. The stack-number of a graph $G$, $\sn(G)$, is the minimum integer $k$ such that $G$ has a $k$-stack layout. Similarly, the queue-number of a graph $G$, $\qn(G)$, is the minimum integer $k$ such that $G$ has a $k$-queue layout. We note that stack layouts of graphs are related to book embeddings of graph and that stack-number is also known as \textit{page-number}, \textit{book-thickness}, or \textit{fixed outer-thickness}.

%Note that stacks and queues are closely related to breadth-first search (BFS) and depth-first search (DFS) layouts of graphs. It can be easily shown that a BFS vertex ordering of a tree admits a $1$-queue layout and a DFS vertex ordering of a tree admits a $1$-stack layout.

%\citet{HLR92} showed that every 1-queue graph has a 2-stack layout. \citet{HLR92} showed that the ternary hypercubes requires exponentially more stacks than queues. In particular, $n$-vertex ternary hypercubes have queue-number at most $2 \log_3 n$, but stack-number at least $\Omega(n^{1/9-\epsilon})$ for any $\epsilon>0$. We prove the following theorem, which shows that stack-number is not bounded by queue-number.

Our key contribution is the following theorem, which shows that stack-number is not bounded by queue-number.
This demonstrates that stacks are not more powerful than queues in the sense discussed above.

\begin{thm}\label{family}
	%  There exists a family $\mathcal{F}$ of graphs for which $\qn(G)\le 4$ for every $G\in\mathcal{F}$, and for every $s\in\N$, there exists $G\in\mathcal{F}$ for which $\sn(G)>s$.
	For every $s\in \N$ there exists a graph $G$ with $\qn(G)\le 4$ and $\sn(G)>s$.
\end{thm}

The graph $G$ in \cref{family} is a cartesian product. For graphs $G_1$ and $G_2$, the \emph{cartesian product} $G_1\CartProd G_2$ is the graph with vertex set $\{(v_1,v_2): v_1 \in V(G_1), v_2 \in V(G_2)\}$, where $(v_1,v_2)(w_1,w_2)\in E(G_1\CartProd G_2)$ if $v_1=w_1$ and $v_2w_2\in E(G_2)$, or $v_1w_1\in E(G_1)$ and $v_2=w_2$.

Let $S_b$ be the star graph with root $r$ and $b$ leaves. For $n\in\N$, let $H_n$ be the dual of the hexagonal grid, defined by $V(H_n):=\{1,\ldots,n\}^2$ and
\begin{align*}
E(H_n) & :=  \{(x,y)(x+1,y):x\in\{1,\ldots,n-1\},\,y\in\{1,\ldots,n\}\} \\
& \qquad \cup \{(x,y)(x,y+1):x\in\{1,\ldots,n\},\,y\in\{1,\ldots,n-1\}\} \\
& \qquad \cup \{(x,y)(x+1,y+1):x,y\in\{1,\ldots,n-1\}\} \enspace .
\end{align*}

The graph $G$ in \cref{family} is $S_b \CartProd H_n$ where $b$ and $n$ are chosen to be sufficiently large compared to $s$, as illustrated in \cref{graph}.
%Note that the Hex graph corresponds to the board used in the game \textit{Hex} which was invented by the  Peit Hein in 1942. In this paper, we prove \Cref{family} with the graph $G:= S_b \square H_n$ where $b$ and $n$ are large compared to $s$.
Note that \citet{Pupyrev20} independently suggested using graph products to show that stack-number is not bounded by queue-number.

\begin{figure}
	\begin{center}
		\begin{tabular}{m{.25\textwidth}m{2ex}m{.25\textwidth}m{2ex}m{.3\textwidth}}
			\includegraphics[width=.25\textwidth]{figs/s} & $\CartProd$ & \includegraphics[width=.25\textwidth]{figs/q} & $=$
			& \includegraphics[width=.3\textwidth]{figs/product}
		\end{tabular}
	\end{center}
	% \includegraphics[width=\textwidth]{figs/figure}
	\caption{$S_9 \CartProd H_4$.}
	\label{graph}
\end{figure}


\subsection*{Subdivisions}

A noteworthy consequence of \Cref{family} is that it resolves a conjecture of \citet{BO99,BO01}. A graph $G'$ is a \textit{subdivision} of a graph $G$ if $G'$ can be obtained from $G$ by replacing the edges $vw$ of $G$ by internally disjoint paths $P_{vw}$ with endpoints $v$ and $w$. If each $P_{vw}$ has exactly $k$ internal vertices, then $G'$ is the \emph{$k$-subdivision} of $G$. If each $P_{vw}$ has at most $k$ internal vertices, then $G'$ is a \emph{$(\leq k)$-subdivision} of $G$. \citet{BO99} conjectured that the stack-number of $(\leq k)$-subdivisions ($k$ fixed)  is not much less than the stack-number of the original graph. More precisely:

\begin{conj}[\citep{BO99}]
\label{B_conj}
There exists a function $f$ such that for every graph $G$ and integer $k$, if $G'$ is any $(\leq k)$-subdivision of $G$, then $\sn(G) \leq f(\sn(G’),k)$.
\end{conj}

\citet{DujWoo05} established a connection between this conjecture and the question of whether stack-number is bounded by queue-number. In particular, they showed that if
\cref{B_conj} is true then stack-number is bounded by queue-number. Since \cref{family} shows that stack-number is not bounded by queue-number, \cref{B_conj} is false. The proof of \citet{DujWoo05} is based on the folllowing key lemma: every graph $G$ has a $3$-stack subdivision with $1+2 \ceil{\log_2\qn(G)}$ division vertices per edge. Applying this result to the graph $G=S_b\CartProd H_n$ in \cref{family},
the $5$-subdivision of $S_b\CartProd H_n$ has a $3$-stack layout. If \cref{B_conj} was true, then $\sn(S_b\CartProd H_n) \leq f( 3,5)$, contradicting \cref{family}.

%Specifically,
%\[
%    \mathcal{F} := \{ S_b\square H_n : b,n\in\N\}
%\]
%where $S_b$ denotes the star with $b$ leaves and $H_n$ is the triangulated $n\times n$ grid.

%\todo[inline]{PM: Does anyone know if there is a standard box operator that is typeset like this $S\boxtimes H$ or $S\boxdot H$ instead of like this $S\square H$ or like this $S\Box H$?  I tried square and Box. DW: I defined \texttt{CartProd} which typesets okay $A \CartProd B$	.}


%The graph $G$ in \cref{family} is obtained as follows (See \figref{graph}): Let $S_b$ denote the star graph with root $r$ and $b$ leaves.  For an even positive integer $n$, let $H_n$ be the $n\times n$ triangulated grid, defined by $V(H_n):=\{1,\ldots,n\}^2$ and
%\begin{align*}
%E(H_n) & :=\{(x,y)(x+1,y):x\in\{1,\ldots,n-1\},\,y\in\{1,\ldots,n\}\} \\
%& \qquad \cup \{(x,y)(x,y+1):x\in\{1,\ldots,n\},\,y\in\{1,\ldots,n-1\}\} \\
%& \qquad \cup \{(x,y+1)(x+1,y):x,y\in\{1,\ldots,n-1\}\} \enspace .
%\end{align*}
%We consider the graph $G:=S_b\CartProd H_n$. That is, $V(G)=V(S_b)\times V(H_n)$ where vertices $(v_1,w_1),(v_2,w_2)\in V(G)$ are adjacent whenever $v_1=w_1$ and $v_2w_2\in E(H_n)$, or $v_1w_1\in E(S_b)$ and $v_2=w_2$.


%SAY SOMETHING ABOUT THE RESULTS OF \citet{Pupyrev20}. I EXPECT WE SOLVE SOME OPEN PROBLEM HERE.\todo{PM:Not really, his open problem is about $H\boxtimes P$ where $H$ has bounded treewidth.}

\subsection*{Is Queue-number Bounded by Stack-Number? }

It remains open whether queues are more powerful than stacks; that is, whether queue-number is bounded by stack-number. Several reults are known about this problem. \citet{HLR92} showed that every 1-stack graph has a 2-queue layout. \citet{DJMMUW20} showed that planar graphs have bounded queue-number. (Note that graph products also feature heavily in this proof.)\ Since 2-stack graphs are planar, this implies that 2-stack graphs have bounded queue-number. It is open whether 3-stack graphs have bounded queue-number. In fact, the case of three stacks is as hard as the general question. \citet{DujWoo05} proved that queue-number is bounded by stack-number if and only if 3-stack graphs have bounded queue-number. Moreover, if this is true then stack-number is bounded by a polynomial function of queue-number.


\section{The Proof}

First we prove that $\qn(S_b\CartProd H_n)\leq 4$, as claimed in \cref{family}. We need the following definition due to \citet{Wood-Queue-DMTCS05}. A queue layout $(\varphi,\prec)$ is \emph{strict} if for every vertex $u\in V(G)$ and for all neighbours $v,w\in N_G(u)$, if $u\prec v,w$ or $v,w \prec u$, then $\varphi(uv)\neq \varphi(uw)$. Let $\sqn(G)$ be the minimum integer $k$ such that $G$ has a strict $k$-queue layout. To see that $\sqn(H_n) \leq 3$, order the vertices row-by-row and then left-to-right within a row, with vertical edges in one queue, horizontal edges in one queue, and diagonal edges in another queue.
\citet{Wood-Queue-DMTCS05} proved that $\qn(G \CartProd H) \leq \qn(G) + \sqn(H)$ for all graphs $G$ and $H$. Of course, $S_b$ has a 1-queue layout (since no two edges are nested for any vertex-ordering). Thus $\qn(S_b \CartProd H_n)\leq 4$.

%\bigskip 
We now turn to the proof of our main result, the lower bound on $\sn(G)$, where $G:= S_b\CartProd H_n$. Consider a hypothetical $s$-stack layout $(\varphi,\prec)$ of $G$ where $n$ and $b$ are chosen sufficiently large compared to $s$ as detailed below. We begin with three lemmata that, for sufficiently large $b$, provide a large subgraph $S_d$ of $S_b$ for which the induced stack layout of $S_d\CartProd H_n$ is highly structured.

For each node $v$ of $S_b$, define $\pi_v$ as the permutation of $\{1,\ldots,n\}^2$ in which $(x_1,y_1)$ appears before $(x_2,y_2)$ if and only $(v,x_1,y_1)\prec (v,x_2,y_2)$.  The following lemma is an immediate consequence of the Pigeonhole Principle:

\begin{lem}\lemlabel{uniform_order}
    There exists a permutation $\pi$ of $\{1,\ldots,n\}^2$ and a set $L_1$ of leaves of $S_b$ of size $b_1\ge \lceil b/(n^2)!\rceil$ such that $\pi_{v}=\pi$ for each $v\in L_1$.
\end{lem}

% \todo[inline]{If we cared, we could improve this to $b/2^{cn^2}$ since we only use the weaker property (P3) in the final proof. DW: I don't care. }

For each leaf $v$ in $L$, let $\varphi_v$ be the edge colouring of $H_n$ defined by $\varphi_v(x,y):=\varphi(v,(x,y))$. Since $H_n$ has maximum degree $6$ and is not 6-regular, it has less than $3n^2$ edges.  Therefore there are fewer than $s^{3n^2}$ edge colourings of $H_n$ using $s$ colours.  Another application of the Pigeonhole Principle proves the following:

\begin{lem}\lemlabel{uniform_colour}
    There exists a subset $L_2\subseteq L_1$ of size $b_2\ge b_1/s^{3n^2}$
    and an edge colouring $\varphi_0:H_n\to\{1,\ldots,s\}$ such that $\varphi_v=\varphi_0$ for each $v\in L_2$.
\end{lem}

The preceding two lemmata ensure that, for distinct leaves $v$ and $w$ of $S_{b_2}$, the stack layouts of the isomorphic graphs $H^v:=G[\{(v,p):p\in V(H_n)]$ and $H^w:=G[\{(w,p):p\in V(H_n)]$ are identical.  The next lemma is a statement about the relationships between the stack layouts of $S^p:=G[\{(v,p):v\in V(S_{b_2})]$ and $S^q:=G[\{(v,q):v\in V(S_{b_2})]$ for  distinct $p,q\in V(H_n)$.  It cannot assert that these two layouts are identical but it does state that they fall into one of two categories.

% \todo{DW: I suggest we replace $H_v$ by $H^v$ and replace $S_p$ by $S^p$.}

\begin{lem}\lemlabel{forward_or_backward}
    There exists a sequence $L_3:=u_1,\ldots,u_{b_3}$ with $\{u_1,\ldots,u_{b_3}\}\subseteq L_2$ of length $b_3\ge b_2^{1/2^{n^2-1}}$ such that, for each $p\in V(H_n)$, either  $(u_1,p)\prec (u_2,p)\prec\cdots\prec (u_{b_3},p)$ or $(u_1,p)\succ (u_2,p)\succ\cdots\succ (u_{b_3},p)$.
\end{lem}

\begin{proof}
    Let $p_1,\ldots,p_{n^2}$ denote the vertices of $H_n$, in any order.
    Begin with the sequence $S_1:=v_{1,1},\ldots,v_{1,d_1}$ that contains all $d_1:=b_2$ elements of $L_2$ ordered so that $(v_{1,1},p_1)\prec\cdots\prec(v_{1,d_1},p_1)$.  For each $i\in\{2,\ldots,n^2\}$, the Erd\H{o}s-Szekeres Theorem~\citep{ES35} implies that, $S_{i-1}$ contains a subsequence $S_i:=v_{i,1},\ldots,v_{i,d_i}$ of length $d_i\ge \sqrt{|S_{i-1}|}$ such that $(v_{i,1},p_i)\prec\cdots\prec(v_{i,d_i},p_i)$ or $(v_{i,1},p_i)\succ\cdots\succ(v_{i,d_i},p_i)$.  It is straightforward to verify by induction on $i$ that $d_i \ge b_2^{1/2^{i-1}}$ resulting in a final sequence $S_{n^2}=:L_3$ of length at least $b_2^{1/2^{n^2-1}}$.
\end{proof}

Let $d:=b_3$ and let $S_d$ be the the subgraph of $S_b$ induced by $\{r\}\cup\{u_1,\ldots,u_{d}\}$ where $u_1,\ldots,u_d$ is the sequence of leaves defined in \lemref{forward_or_backward}.  Consider the (improper) vertex colouring of $H_n$ obtained by colouring each vertex $p\in V(H_n)$ \emph{red} if $(u_1,p)\prec\cdots\prec (u_d,p)$ and colouring $p$ \emph{blue} if $(u_1,p)\succ\cdots\succ(u_d,p)$. We need the following famous Hex Lemma~\citep{Gale79}.

\begin{lem}\lemlabel{hex_lemma}
Every red--blue vertex colouring of the graph $H_n$	contains an $n$-vertex path $R$ consisting entirely of red vertices or entirely of blue vertices.
\end{lem}

%\begin{proof}
%    The dual of $H_n$ is the board on which the game Hex is played.  The well-known \emph{Hex Lemma} states that any colouring of the vertices of $H_n$ with colours red and blue contains exactly one of the following \cite{Gale79}:
%    \begin{compactenum}
%        \item a path with endpoints $(x,1)$ and $(x',n)$ consisting entirely of red vertices, for some $x,x'\in\{1,\ldots,n\}$; or
%        \item a path with endpoints $(1,y)$ and $(n,y')$ consisting entirely of blue vertices, for some $y,y'\in\{1,\ldots,n\}$.
%    \end{compactenum}
%    In either case, the path contains at least $n$ vertices and therefore has a $n$-vertex subpath consisting entirely of red vertices or entirely of blue vertices.
%\end{proof}

Without loss of generality, assume that the path $R:=p_1,\ldots,p_n$ defined by \lemref{hex_lemma} consists entirely of red vertices, so that $(u_1,p_j)\prec\cdots\prec (u_d,p_j)$ for each $j\in\{1,\ldots,n\}$.
Recall that $(\varphi,\prec)$ is a hypothetical $s$-stack layout of $G$ and therefore it is also an $s$-stack layout of the subgraph $X:=S_d\CartProd R$.  The following result finishes the proof by showing that this is not possible when $n> 2s$ and $d> s2^{n}$.

\begin{lem}
    The graph $X$ contains a set of edges of size at least $\min\{d/2^{n},n/2\}$ that are pairwise crossing with respect to $\prec$.
\end{lem}

\begin{proof}
    We will define two sequences of nested sets $A_1\supseteq \cdots\supseteq A_{n}$ of leaves of $S_d$ so that each $A_i$ satisifies the following conditions:
    \begin{compactenum}[(C1)]
        \item $A_i$ contain $d_i\ge d/2^{i-1}$ leaves of $S_d$.
        \item Each leaf $v\in A_i$ defines an $i$-element vertex set $Z_{i,v}:=\{(v,p_j):j\in\{1,\ldots,i\}\}$.  For any distinct $v,w\in A_i$, $Z_{i,v}$ and $Z_{i,w}$ are \emph{separated} with respect to $\prec$, i.e., $Z_{i,v}\prec Z_{i,w}$ or $Z_{i,v}\succ Z_{i,w}$.
    \end{compactenum}

    Before defining $A_1,\ldots,A_n$ we first show how the existence of the set $A_n$ implies the lemma.  To avoid triple-subscripts, let $d':=d_n\ge d/2^n$.   The set $A_n$ defines vertex sets $Z_{n,v_1}\prec\cdots\prec Z_{n,v_{d'}}$.  Refer to \figref{twister}. Recall that $r$ is the root of $S_b$ so it is adjacent to each of $v_{1},\ldots,v_{d'}$ in $S_b$.  Therefore, for each $j\in\{1,\ldots,n\}$ and each $i\in\{1,\ldots,d'\}$, the edge $(r,p_j)(v_i,p_j)$ is in $X$. Therefore, $(r,p_j)$ is adjacent to an element of each of $Z_{n,v_1},\ldots,Z_{n,v_{d'}}$.
	\begin{figure}
		\begin{center}
			\includegraphics{figs/twister}
		\end{center}
		\caption{The sets $Z_{n,1},\ldots,Z_{n,d'}$ ($n=4$, $d'=5$).}
		\figlabel{twister}
	\end{figure}

    Since $Z_{n,v_1},\ldots,Z_{n,v_{d'}}$ are separated with respect to $\prec$, when viewed from afar, this situation looks like a complete bipartite graph $K_{n,d'}$ with the root vertices $L:=\{(r,p_j):j\in\{1,\ldots,n\}\}$ in one part and the groups $R:=Z_{n,v_1}\cup\cdots\cup Z_{n,v_{d'}}$ in the other part.  Any linear ordering of $K_{n,d'}$ has a large set of pairwise crossing edges so, intuitively, the graph induced by $L\cup R$ should also have a large set of pairwise crossing edges. \lemref{twister}, below, formalizes this and shows that this graph has a set of at least $\min\{d',n\}/2$ pairwise crossing edges.

    All that remains is to define the sets $A_1\supseteq\cdots\supseteq A_n$ that satisfy (C1) and (C2).  The set $A_1$ contains all the leaves of $S_d$.  For each $i\in\{2,\ldots,n\}$, the set $A_i$ is defined as follows:  Let $Z_1,\ldots,Z_r$ denote the sets $\{(v,p_j):j\in\{1,\ldots,i-1\}\}$ for $v\in A_{i-1}$ ordered so that $Z_1\prec\cdots\prec Z_r$.  By Property (C2), this is always possible.	Label the vertices of $A_{i-1}$ as $v_1,\ldots,v_r$ so that $(v_1,p_{i-1})\prec\cdots\prec (v_r,p_{i-1})$.   (This is equivalent to naming them so that $(v_j,p_{i-1})\in Z_j$ for each $j\in\{1,\ldots,r\}$.)  We define the set $A_i:=\{v_{2k+1}:k\in\{0,\ldots,\lfloor(r-1)/2\rfloor\}\}=\{v_{j}\in A_{i-1}:\text{$j$ is odd}\}$.  This completes the definition of $A_1,\ldots,A_n$.

	All that remains is to verify that $A_i$ satisfies (C1) and (C2).  We do this by induction on $i$. The base case $i=1$ is trivial so we assume from this point on that $i\in\{2,\ldots,n\}$.   To see that $A_i$ satisfies (C1) just observe that $|A_i|=\lceil r/2\rceil \ge r/2= |A_{i-1}|/2\ge d/2^{i-1}$.  All that remains is to show that $A_i$ satisfies (C2).

    For each $j\in\{i-1,i\}$, let $H^j:=H_n[\{(v,p_j):v\in A_{i-1}\}]$.
    Recall that, for each $v\in A_{i-1}$, the edge $e_v:=(v,p_{i-1})(v,p_i)$ is in $X$.  We have the following properties:
    \begin{compactenum}[(P1)]
        \item By \lemref{uniform_colour}, $\varphi(e_v)=\varphi_0(p_{i-1},p_i)$ for each $v\in A_{i-1}$.
        \item By the application of \lemref{hex_lemma}, $(v,p_{i-1})\prec (w,p_{i-1})$ if and only if $(v,p_{i})\prec (w,p_{i})$ for each $v,w\in A_{i-1}$.
        \item By \lemref{uniform_order}, $(v,p_{i-1})\prec (v,p_i)$ for every $v\in A_{i-1}$ or $(v,p_{i-1})\succ (v,p_i)$ for every $v\in A_{i-1}$.
    \end{compactenum}
    We claim that these three conditions imply that the vertex sets of $H^{i-1}$ and $H^{i}$ interleave perfectly with respect to $\prec$. More precisely:
	\begin{clm}\clmlabel{interleave} $(v_1,p_{i-1+t})\prec (v_1,p_{i-t}) \prec (v_2,p_{i-1+t}) \prec (v_2,p_{i-t}) \cdots \prec (v_r,p_{i-1+t}) \prec (v_r,p_{i-t})$ for some $t\in\{0,1\}$.
	\end{clm}
	\begin{proof}[Proof of \clmref{interleave}]
		By (P3) we may assume, without loss of generality, that $(v,p_{i-1})\prec (v,p_i)$ for each $v\in A_{i-1}$, in which case we are trying to prove the claim for $t=0$.  It is sufficient, therefore to show that $(v_j,p_i)\prec (v_{j+1},p_{i-1})$ for each $j\in\{1,\ldots,r-1\}$.  For the sake of contradiction, suppose $(v_j,p_{i})\succ (v_{j+1},p_{i-1})$ for some $j\in\{1,\ldots,r-1\}$. By definition $(v_j,p_{i-1})\prec (v_{j+1},p_{i-1})$ so, by (P2)  $(v_{j},p_i) \prec (v_{j+1},p_i)$.  Therefore
		\[
			(v_j,p_{i-1})\prec (v_{j+1},p_{i-1})\prec(v_{j},p_i) \prec
		   (v_{j+1}, p_i) \enspace .
	   	\]
		Therefore the edges $(v_j,p_{i-1})(v_j,p_{i})$ and $(v_{j+1},p_{i-1})(v_{j+1},p_i)$ cross with repect to $\prec$.  But this is a contradiction since, by (P1),  $\varphi((v_j,p_{i-1})(v_j,p_{i})) =\varphi((v_{j+1},p_{i-1})(v_{j+1},p_i))=\varphi_0(p_{i-1}p_i)$.
		This contradiction completes the proof of \clmref{interleave}.
	\end{proof}

% \todo{DW: Why are these $\prec$'s red?  PM: Just me keeping track of which one was the assumption, they don't need to be red.}

	Now, apply \clmref{interleave} and assume without loss of generality that $t=0$, so that
	\[
		(v_1,p_{i-1})\prec (v_1,p_{i}) \prec (v_2,p_{i-1}) \prec (v_2,p_{i}) \cdots \prec (v_r,p_{i-1}) \prec (v_r,p_{i}) \enspace .
	\]

    For each $j\in\{1,\ldots,r-2\}$, $(v_{j+1},p_{i-1})\in Z_{j+1}\prec Z_{j+2}$, so  $(v_j,p_i)\prec (v_{j+1},p_{i-1}) \prec Z_{j+2}$.  Therefore $Z_j\cup\{(v_j,p_i)\} \prec Z_{j+2}$.  By a symmetric argument, $Z_j\cup\{(v_j,p_i)\} \succ Z_{j-2}$ for each  $j\in\{2,\ldots,r\}$.  Finally, since $(v_{j},p_i)\prec (v_{j+2},p_i)$ for each odd $i\in\{1,\ldots,r\}$, we have $Z_{j}\cup\{(v_j,p_i)\} \prec Z_{j+2}\cup\{(v_{j+2},p_i)\}$ for each odd $j\in\{1,\ldots,r-2\}$.  Thus $A_i$ satisifies (C2) since the sets $Z_1\cup\{(v_1,p_i)\},Z_3\cup\{(v_3,p_i)\},\ldots,Z_{2\floor (r-1)/2\rfloor+1} \cup (v_{2\floor (r-1)/2\rfloor+1},p_i)$ are precisely the sets $Z_{i,1},\ldots,Z_{i,d_i}$ determined by our choice of $A_i$.
\end{proof}

\begin{lem}\lemlabel{twister}
    Let $G$ be any graph, let $\prec$ be any linear ordering of $V(G)$,  let $Z_{1}\prec\cdots\prec Z_{2s}$ be subsets of $V(G)$, and let $r_1\prec\cdots\prec r_{2s}$ be vertices of $G$ such that, for each $i,j\in\{1,\ldots,2s\}$, $G$ contains an edge $r_iz_j$ with $z_j\in Z_j$. Then $G$ contains a set of $s$ edges that are pairwise crossing with respect to $\prec$.
\end{lem}

\begin{proof}
    At least one of the following two cases applies (see \figref{median}):
    \begin{enumerate}
        \item $Z_s\prec r_{s+1}$ in which case the graph between $r_{s+1},\ldots,r_{2s}$ and $Z_1,\ldots,Z_s$ has a set of $s$ pairwise-crossing edges.
        \item $r_{s}\prec Z_{s+1}$ in which case the graph between $r_1,\ldots,r_s$ and $Z_{s+1},\ldots,Z_{2s}$ has a set of $s$ pairwise-crossing edges. \qedhere
    \end{enumerate}
\end{proof}
\begin{figure}
	\begin{center}
		\includegraphics{figs/median-1} \\ 1 \\[2em]
		\includegraphics{figs/median-2} \\ 2
	\end{center}
	\caption{The two cases in the proof of \lemref{twister}.}
	\figlabel{median}
\end{figure}

\section{Open Problems}

Recall that every 1-queue graph has a 2-stack layout \citep{HLR92} and we proved that there are 4-queue graphs with unbounded stack-number. The following questions remain open: Do 2-queue graphs have bounded stack-number? Do 3-queue graphs have bounded stack-number?

Given the role of cartesian products in our proof, it is natural to ask when is $\sn(G_1\CartProd G_2)$ bounded? Note that $H_n$ is a subgraph of a planar Hamiltonian graphs (namely, $H_{2n}$), so $\sn(H_n) \leq 2$. So $\sn(G_1\CartProd G_2)$ can be unbounded even when $G_1$ is a star and $\sn(G_2)\leq 2$.
Since $\sn(G_2)\leq 1$ if and only $G_2$ is outerplanar, the following question naturally arises: Is $\sn(S \CartProd H)$ bounded for every star $S$ and outerplanar graph $H$ with bounded degree? Is $\sn(T \CartProd H)$ bounded for every tree $T$ and outerplanar graph $H$ with bounded degree? The assumption that $H$ has bounded degree is needed since $S_n \CartProd S_n$ contain the 1-subdivision of $K_{n,n}$, which has unbounded stack-number~\citep{Blankenship-PhD03}.

MENTION RESULTS OF \citet{Pupyrev20} about bipartite graphs.

Since $H_n\subseteq P \boxtimes P$ where $P$ is the $n$-vertex path, \cref{family} implies that $\sn(S\boxtimes P\boxtimes P)$ is unbounded for stars $S$ and paths $P$. It is easily seen that $\sn(S\boxtimes P)$ is bounded~\citep{Pupyrev20}. The following question naturally arises (independently asked by \citet{Pupyrev20}):
Is $\sn(T \boxtimes P)$ bounded for every tree $T$ and path $P$? We conjecture the answer is ``no''.

%\documentclass[kpfonts]{patmorin}
\usepackage{pat}
\usepackage{paralist,graphicx}
\usepackage{array,longtable}

\usepackage[utf8]{inputenc}
\usepackage{todonotes}

\usepackage[noabbrev,capitalise]{cleveref}
\crefname{lem}{Lemma}{Lemmas}
\crefname{thm}{Theorem}{Theorems}
\crefname{cor}{Corollary}{Corollaries}
\crefname{prop}{Proposition}{Propositions}
\crefname{conj}{Conjecture}{Conjectures}
\crefname{open}{Open Problem}{Open Problems}
\crefname{obs}{Observation}{Observations}

\crefformat{equation}{(#2#1#3)}
\Crefformat{equation}{Equation #2(#1)#3}

\usepackage[numbers,sort&compress]{natbib}
\usepackage{hypernat}
\makeatletter
\def\NAT@spacechar{~}
\makeatother

\setlength{\parskip}{1ex}
\setlength{\parindent}{0ex}


\newtheorem{property}{Property}
\newcommand{\plabel}[1]{\label{prop:#1}}
\newcommand{\pref}[1]{Property~\ref{prop:#1}}

\DeclareMathOperator{\sn}{sn}
\DeclareMathOperator{\qn}{qn}
\DeclareMathOperator{\sqn}{sqn}
\DeclareMathOperator{\tw}{tw}

\renewcommand{\SS}{\mathcal{S}}

\renewcommand{\le}{\leqslant}
\renewcommand{\leq}{\leqslant}
\renewcommand{\ge}{\geqslant}
\renewcommand{\geq}{\geqslant}

\newcommand{\CartProd}{\mathbin{\square}}

\title{\MakeUppercase{Stack-Number is not Bounded by Queue-Number}}

\author{%
	Vida Dujmovi\'c,\!\!%
	\thanks{School of Computer Science and Electrical Engineering,
		University of Ottawa, Ottawa, Canada (\texttt{vida.dujmovic@uottawa.ca}).
		Research supported by NSERC and the Ontario Ministry of Research and Innovation.}
	\,\,
	Robert Hickingbotham,\!\!%
	\thanks{School of Mathematics, Monash University, Melbourne, Australia (\texttt{robert.hickingbotham@monash.edu}).}
	\,\,
	Pat Morin,\!\!%
	\thanks{School of Computer Science, Carleton University, Ottawa, Canada (\texttt{morin@scs.carleton.ca}). Research  supported by NSERC and the Ontario Ministry of Research and Innovation.}
	\,\,
	David R. Wood\thanks{School of Mathematics, Monash University, Melbourne, Australia (\texttt{david.wood@monash.edu}). Research supported by the Australian Research Council.}
}

\begin{document}
\maketitle

\begin{abstract}
We describe a family of graphs with queue-number at most 4 but unbounded stack-number. This resolves open problems of Heath, Leighton and Rosenberg (1992) and Blankenship and Oporwoski (1999).
\end{abstract}

\section{Introduction}

Stacks and queues are fundamental data structures in computer science, but which is more powerful? In 1992, \citet{HLR92} introduced an approach for answering this question by defining the graph parameters \textit{stack-number} and \textit{queue-number}, which respectively measure the power of stacks and queues for representing graphs.

%We now formally define stack and queue layouts of graphs.
Let $G$ be a graph and let $\prec$ be a total order on $V(G)$.  Two disjoint edges $vw,xy\in E(G)$ with $v\prec w$ and $x\prec y$ \emph{cross} with respect to $\prec$ if $v\prec x\prec w\prec y$ or $x\prec v\prec y\prec w$, and \emph{nest} with respect to $\prec$ if $v\prec x\prec y\prec w$ or $x\prec v\prec w\prec y$.
%A set of $k$ pairwise nested edges is called a \emph{$k$-rainbow} and a set of $k$ pairwise crossing edges is called a \emph{$k$-twist}. A set of $k$ edges, no pair of which nest and no pair of which cross is called a \emph{hopper}.
Let $\varphi:E(G)\to\{1,\ldots,k\}$ for some integer $k\ge 1$.  Then $(\prec,\varphi)$ is a \emph{$k$-stack layout} of $G$ if, for every pair of edges $vw,xy\in E(G)$, if $\varphi(vw) = \varphi(xy)$ then $vw$ and $xy$ do not cross.\todo{PM: Suggestion: Replace second if then with $\varphi(vw)\neq\varphi(xy)$ or $vw$ and $xy$ do not cross.} Similarly, the pair $(\prec,\varphi)$ is a \emph{$k$-queue layout} of $G$ if, for every pair of edges $vw,xy\in E(G)$, if $\varphi(vw)=\varphi(xy)$ then  $vw$ and $xy$ do not nest. The smallest integer $s$ for which $G$ has an $s$-stack layout is called the \emph{stack-number} of $G$, denoted  $\sn(G)$. The smallest integer $q$ for which $G$ has a $q$-queue layout is called the \emph{queue-number} of $G$, denoted $\qn(G)$. Note that stack layouts are equivalent to book embeddings, and stack-number is also known as \emph{page-number}, \emph{book-thickness} or \emph{fixed outer-thickness}. \todo{add references}

Given a $k$-stack layout $(\prec,\varphi)$ of a graph $G$, for each $i\in\{1,\dots,k\}$, the set $E_i:= \{e\in E(G):\varphi(e)=i\}$ behaves like a stack, in the sense that each edge $e=vw \in E_i$ with $v\prec w$ corresponds to an element in a sequence of stack opertions, such that if we traverse the vertices in the order of $\prec$, then $e$ is pushed onto the stack at $v$ and popped off the stack at $w$. Similarly, given a $k$-queue layout $(\prec,\varphi)$, each set $E_i$ behaves like a queue. In this way, the stack-number and queue-number  respectively measure the power of stacks and queues to represent graphs.

\todo{Should the following two paragraphs be moved to prior to the definitions\newline PM: Yes}

The following key problems are implcit in the work of \citet{HLR92}, made explicit by \citet{DujWoo05}
\footnote{A \emph{graph parameter} is a function $\alpha$ such that $\alpha(G)\in\mathbb{R}$ for every graph $G$, such that $\alpha(G_1)=\alpha(G_2)$ for all isomorphic graphs $G_1$ and $G_2$. A graph parameter $\alpha$ is \textit{bounded} by a graph parameter $\beta$ if there exists a function $f$ such that for every graph $G$ we have $\alpha(G) \leq f(\beta(G))$.}:
\begin{compactitem}
	\item Is stack-number bounded by queue-number?
	\item Is queue-number bounded by stack-number?
\end{compactitem}

If stack-number is bounded by queue-number but queue-number is not bounded by stack-number, then we would consider stacks to be more powerful than queues. Similarly, if the converse holds, then we would consider queues to be more powerful than stacks. Despite extensive research on stack and queue layouts of graphs (see \citep{DujWoo04,DujWoo-DCG07,DJMMUW20} and the references therein), these fundamental questions have remained unsolved.

\subsection*{Is Stack-Number Bounded by Queue-number?}

This paper considers the first of the above questions. In a positive direction, \citet{HLR92}  showed that every 1-queue graph has a $2$-stack layout. On the other hand, they described graphs that need exponentially more stacks than queues. In particular, $n$-vertex ternary hypercubes have queue-number $O(\log n)$ and stack-number $\Omega(n^{1/9-\epsilon})$ for any $\epsilon>0$.

%Note that, in an $s$-stack layout, $(\prec,\varphi)$, $\varphi$ is a proper $s$-colouring of the auxilliary graph $H$ with vertex set $V(H)=E(G)$ and in which the edge $ef$ if present if and only if $e$ and $f$ cross with respect to $\varphi$.  Any $k$-twist in $G$ with respect to $\prec$ corresponds to a $k$-clique $H$.  Since the chromatic number of any graph is bounded by its clique number, the following observation is trivial:

%\begin{obs}\obslabel{no-s-twist}
%If $(\prec,\varphi)$ is an $s$-stack layout of a graph $G$ then $E(G)$ does not contain any $k$-twist for any $k>s$.
%\end{obs}

%We now formally define stack and queue layouts of graphs. Let $G=(V,E)$ be a graph with disjoint edges $vw, xy$ and a linear ordering $\leq$ of the vertices. Without loss of generality, we may assume that $v < w$, $x <y$ and $v < x$. We say that $vw$ and $xy$ \textit{cross} if $v<x<w<y$,  \textit{nest} if $v <x <y < w$, and are \textit{disjoint} if $v <w<x<y$. \todo{DW: Do we  need ``disjoint''?} A \textit{stack} is a set of pairwise non-crossing edges and a \textit{queue} is a set of pairwise non-nesting edges. A $k$-queue layout of $G$ consist of a linear ordering $\leq$ of its vertices and a partition $E_1,E_2, \dots, E_k$, of its edges into queues with respect to $\leq$. The stack-number of a graph $G$, $\sn(G)$, is the minimum integer $k$ such that $G$ has a $k$-stack layout. Similarly, the queue-number of a graph $G$, $\qn(G)$, is the minimum integer $k$ such that $G$ has a $k$-queue layout. We note that stack layouts of graphs are related to book embeddings of graph and that stack-number is also known as \textit{page-number}, \textit{book-thickness}, or \textit{fixed outer-thickness}.

%Note that stacks and queues are closely related to breadth-first search (BFS) and depth-first search (DFS) layouts of graphs. It can be easily shown that a BFS vertex ordering of a tree admits a $1$-queue layout and a DFS vertex ordering of a tree admits a $1$-stack layout.

%\citet{HLR92} showed that every 1-queue graph has a 2-stack layout. \citet{HLR92} showed that the ternary hypercubes requires exponentially more stacks than queues. In particular, $n$-vertex ternary hypercubes have queue-number at most $2 \log_3 n$, but stack-number at least $\Omega(n^{1/9-\epsilon})$ for any $\epsilon>0$. We prove the following theorem, which shows that stack-number is not bounded by queue-number.

Our key contribution is the following theorem, which shows that stack-number is not bounded by queue-number.
This demonstrates that stacks are not more powerful than queues in the sense discussed above.

\begin{thm}\label{family}
	%  There exists a family $\mathcal{F}$ of graphs for which $\qn(G)\le 4$ for every $G\in\mathcal{F}$, and for every $s\in\N$, there exists $G\in\mathcal{F}$ for which $\sn(G)>s$.
	For every $s\in \N$ there exists a graph $G$ with $\qn(G)\le 4$ and $\sn(G)>s$.
\end{thm}

The graph $G$ in \cref{family} is a cartesian product. For graphs $G_1$ and $G_2$, the \emph{cartesian product} $G_1\CartProd G_2$ is the graph with vertex set $\{(v_1,v_2): v_1 \in V(G_1), v_2 \in V(G_2)\}$, where $(v_1,v_2)(w_1,w_2)\in E(G_1\CartProd G_2)$ if $v_1=w_1$ and $v_2w_2\in E(G_2)$, or $v_1w_1\in E(G_1)$ and $v_2=w_2$.

Let $S_b$ be the star graph with root $r$ and $b$ leaves. For $n\in\N$, let $H_n$ be the dual of the hexagonal grid, defined by $V(H_n):=\{1,\ldots,n\}^2$ and
\begin{align*}
E(H_n) & :=  \{(x,y)(x+1,y):x\in\{1,\ldots,n-1\},\,y\in\{1,\ldots,n\}\} \\
& \qquad \cup \{(x,y)(x,y+1):x\in\{1,\ldots,n\},\,y\in\{1,\ldots,n-1\}\} \\
& \qquad \cup \{(x,y)(x+1,y+1):x,y\in\{1,\ldots,n-1\}\} \enspace .
\end{align*}

The graph $G$ in \cref{family} is $S_b \CartProd H_n$ where $b$ and $n$ are chosen to be sufficiently large compared to $s$, as illustrated in \cref{graph}.
%Note that the Hex graph corresponds to the board used in the game \textit{Hex} which was invented by the  Peit Hein in 1942. In this paper, we prove \Cref{family} with the graph $G:= S_b \square H_n$ where $b$ and $n$ are large compared to $s$.
Note that \citet{Pupyrev20} independently suggested using graph products to show that stack-number is not bounded by queue-number.

\begin{figure}
	\begin{center}
		\begin{tabular}{m{.25\textwidth}m{2ex}m{.25\textwidth}m{2ex}m{.3\textwidth}}
			\includegraphics[width=.25\textwidth]{figs/s} & $\CartProd$ & \includegraphics[width=.25\textwidth]{figs/q} & $=$
			& \includegraphics[width=.3\textwidth]{figs/product}
		\end{tabular}
	\end{center}
	% \includegraphics[width=\textwidth]{figs/figure}
	\caption{$S_9 \CartProd H_4$.}
	\label{graph}
\end{figure}


\subsection*{Subdivisions}

A noteworthy consequence of \Cref{family} is that it resolves a conjecture of \citet{BO99,BO01}. A graph $G'$ is a \textit{subdivision} of a graph $G$ if $G'$ can be obtained from $G$ by replacing the edges $vw$ of $G$ by internally disjoint paths $P_{vw}$ with endpoints $v$ and $w$. If each $P_{vw}$ has exactly $k$ internal vertices, then $G'$ is the \emph{$k$-subdivision} of $G$. If each $P_{vw}$ has at most $k$ internal vertices, then $G'$ is a \emph{$(\leq k)$-subdivision} of $G$. \citet{BO99} conjectured that the stack-number of $(\leq k)$-subdivisions ($k$ fixed)  is not much less than the stack-number of the original graph. More precisely:

\begin{conj}[\citep{BO99}]
\label{B_conj}
There exists a function $f$ such that for every graph $G$ and integer $k$, if $G'$ is any $(\leq k)$-subdivision of $G$, then $\sn(G) \leq f(\sn(G’),k)$.
\end{conj}

\citet{DujWoo05} established a connection between this conjecture and the question of whether stack-number is bounded by queue-number. In particular, they showed that if
\cref{B_conj} is true then stack-number is bounded by queue-number. Since \cref{family} shows that stack-number is not bounded by queue-number, \cref{B_conj} is false. The proof of \citet{DujWoo05} is based on the folllowing key lemma: every graph $G$ has a $3$-stack subdivision with $1+2 \ceil{\log_2\qn(G)}$ division vertices per edge. Applying this result to the graph $G=S_b\CartProd H_n$ in \cref{family},
the $5$-subdivision of $S_b\CartProd H_n$ has a $3$-stack layout. If \cref{B_conj} was true, then $\sn(S_b\CartProd H_n) \leq f( 3,5)$, contradicting \cref{family}.

%Specifically,
%\[
%    \mathcal{F} := \{ S_b\square H_n : b,n\in\N\}
%\]
%where $S_b$ denotes the star with $b$ leaves and $H_n$ is the triangulated $n\times n$ grid.

%\todo[inline]{PM: Does anyone know if there is a standard box operator that is typeset like this $S\boxtimes H$ or $S\boxdot H$ instead of like this $S\square H$ or like this $S\Box H$?  I tried square and Box. DW: I defined \texttt{CartProd} which typesets okay $A \CartProd B$	.}


%The graph $G$ in \cref{family} is obtained as follows (See \figref{graph}): Let $S_b$ denote the star graph with root $r$ and $b$ leaves.  For an even positive integer $n$, let $H_n$ be the $n\times n$ triangulated grid, defined by $V(H_n):=\{1,\ldots,n\}^2$ and
%\begin{align*}
%E(H_n) & :=\{(x,y)(x+1,y):x\in\{1,\ldots,n-1\},\,y\in\{1,\ldots,n\}\} \\
%& \qquad \cup \{(x,y)(x,y+1):x\in\{1,\ldots,n\},\,y\in\{1,\ldots,n-1\}\} \\
%& \qquad \cup \{(x,y+1)(x+1,y):x,y\in\{1,\ldots,n-1\}\} \enspace .
%\end{align*}
%We consider the graph $G:=S_b\CartProd H_n$. That is, $V(G)=V(S_b)\times V(H_n)$ where vertices $(v_1,w_1),(v_2,w_2)\in V(G)$ are adjacent whenever $v_1=w_1$ and $v_2w_2\in E(H_n)$, or $v_1w_1\in E(S_b)$ and $v_2=w_2$.


%SAY SOMETHING ABOUT THE RESULTS OF \citet{Pupyrev20}. I EXPECT WE SOLVE SOME OPEN PROBLEM HERE.\todo{PM:Not really, his open problem is about $H\boxtimes P$ where $H$ has bounded treewidth.}

\subsection*{Is Queue-number Bounded by Stack-Number? }

It remains open whether queues are more powerful than stacks; that is, whether queue-number is bounded by stack-number. Several reults are known about this problem. \citet{HLR92} showed that every 1-stack graph has a 2-queue layout. \citet{DJMMUW20} showed that planar graphs have bounded queue-number. (Note that graph products also feature heavily in this proof.)\ Since 2-stack graphs are planar, this implies that 2-stack graphs have bounded queue-number. It is open whether 3-stack graphs have bounded queue-number. In fact, the case of three stacks is as hard as the general question. \citet{DujWoo05} proved that queue-number is bounded by stack-number if and only if 3-stack graphs have bounded queue-number. Moreover, if this is true then stack-number is bounded by a polynomial function of queue-number.


\section{The Proof}

First we prove that $\qn(S_b\CartProd H_n)\leq 4$, as claimed in \cref{family}. We need the following definition due to \citet{Wood-Queue-DMTCS05}. A queue layout $(\varphi,\prec)$ is \emph{strict} if for every vertex $u\in V(G)$ and for all neighbours $v,w\in N_G(u)$, if $u\prec v,w$ or $v,w \prec u$, then $\varphi(uv)\neq \varphi(uw)$. Let $\sqn(G)$ be the minimum integer $k$ such that $G$ has a strict $k$-queue layout. To see that $\sqn(H_n) \leq 3$, order the vertices row-by-row and then left-to-right within a row, with vertical edges in one queue, horizontal edges in one queue, and diagonal edges in another queue.
\citet{Wood-Queue-DMTCS05} proved that $\qn(G \CartProd H) \leq \qn(G) + \sqn(H)$ for all graphs $G$ and $H$. Of course, $S_b$ has a 1-queue layout (since no two edges are nested for any vertex-ordering). Thus $\qn(S_b \CartProd H_n)\leq 4$.

%\bigskip 
We now turn to the proof of our main result, the lower bound on $\sn(G)$, where $G:= S_b\CartProd H_n$. Consider a hypothetical $s$-stack layout $(\varphi,\prec)$ of $G$ where $n$ and $b$ are chosen sufficiently large compared to $s$ as detailed below. We begin with three lemmata that, for sufficiently large $b$, provide a large subgraph $S_d$ of $S_b$ for which the induced stack layout of $S_d\CartProd H_n$ is highly structured.

For each node $v$ of $S_b$, define $\pi_v$ as the permutation of $\{1,\ldots,n\}^2$ in which $(x_1,y_1)$ appears before $(x_2,y_2)$ if and only $(v,x_1,y_1)\prec (v,x_2,y_2)$.  The following lemma is an immediate consequence of the Pigeonhole Principle:

\begin{lem}\lemlabel{uniform_order}
    There exists a permutation $\pi$ of $\{1,\ldots,n\}^2$ and a set $L_1$ of leaves of $S_b$ of size $b_1\ge \lceil b/(n^2)!\rceil$ such that $\pi_{v}=\pi$ for each $v\in L_1$.
\end{lem}

% \todo[inline]{If we cared, we could improve this to $b/2^{cn^2}$ since we only use the weaker property (P3) in the final proof. DW: I don't care. }

For each leaf $v$ in $L$, let $\varphi_v$ be the edge colouring of $H_n$ defined by $\varphi_v(x,y):=\varphi(v,(x,y))$. Since $H_n$ has maximum degree $6$ and is not 6-regular, it has less than $3n^2$ edges.  Therefore there are fewer than $s^{3n^2}$ edge colourings of $H_n$ using $s$ colours.  Another application of the Pigeonhole Principle proves the following:

\begin{lem}\lemlabel{uniform_colour}
    There exists a subset $L_2\subseteq L_1$ of size $b_2\ge b_1/s^{3n^2}$
    and an edge colouring $\varphi_0:H_n\to\{1,\ldots,s\}$ such that $\varphi_v=\varphi_0$ for each $v\in L_2$.
\end{lem}

The preceding two lemmata ensure that, for distinct leaves $v$ and $w$ of $S_{b_2}$, the stack layouts of the isomorphic graphs $H^v:=G[\{(v,p):p\in V(H_n)]$ and $H^w:=G[\{(w,p):p\in V(H_n)]$ are identical.  The next lemma is a statement about the relationships between the stack layouts of $S^p:=G[\{(v,p):v\in V(S_{b_2})]$ and $S^q:=G[\{(v,q):v\in V(S_{b_2})]$ for  distinct $p,q\in V(H_n)$.  It cannot assert that these two layouts are identical but it does state that they fall into one of two categories.

% \todo{DW: I suggest we replace $H_v$ by $H^v$ and replace $S_p$ by $S^p$.}

\begin{lem}\lemlabel{forward_or_backward}
    There exists a sequence $L_3:=u_1,\ldots,u_{b_3}$ with $\{u_1,\ldots,u_{b_3}\}\subseteq L_2$ of length $b_3\ge b_2^{1/2^{n^2-1}}$ such that, for each $p\in V(H_n)$, either  $(u_1,p)\prec (u_2,p)\prec\cdots\prec (u_{b_3},p)$ or $(u_1,p)\succ (u_2,p)\succ\cdots\succ (u_{b_3},p)$.
\end{lem}

\begin{proof}
    Let $p_1,\ldots,p_{n^2}$ denote the vertices of $H_n$, in any order.
    Begin with the sequence $S_1:=v_{1,1},\ldots,v_{1,d_1}$ that contains all $d_1:=b_2$ elements of $L_2$ ordered so that $(v_{1,1},p_1)\prec\cdots\prec(v_{1,d_1},p_1)$.  For each $i\in\{2,\ldots,n^2\}$, the Erd\H{o}s-Szekeres Theorem~\citep{ES35} implies that, $S_{i-1}$ contains a subsequence $S_i:=v_{i,1},\ldots,v_{i,d_i}$ of length $d_i\ge \sqrt{|S_{i-1}|}$ such that $(v_{i,1},p_i)\prec\cdots\prec(v_{i,d_i},p_i)$ or $(v_{i,1},p_i)\succ\cdots\succ(v_{i,d_i},p_i)$.  It is straightforward to verify by induction on $i$ that $d_i \ge b_2^{1/2^{i-1}}$ resulting in a final sequence $S_{n^2}=:L_3$ of length at least $b_2^{1/2^{n^2-1}}$.
\end{proof}

Let $d:=b_3$ and let $S_d$ be the the subgraph of $S_b$ induced by $\{r\}\cup\{u_1,\ldots,u_{d}\}$ where $u_1,\ldots,u_d$ is the sequence of leaves defined in \lemref{forward_or_backward}.  Consider the (improper) vertex colouring of $H_n$ obtained by colouring each vertex $p\in V(H_n)$ \emph{red} if $(u_1,p)\prec\cdots\prec (u_d,p)$ and colouring $p$ \emph{blue} if $(u_1,p)\succ\cdots\succ(u_d,p)$. We need the following famous Hex Lemma~\citep{Gale79}.

\begin{lem}\lemlabel{hex_lemma}
Every red--blue vertex colouring of the graph $H_n$	contains an $n$-vertex path $R$ consisting entirely of red vertices or entirely of blue vertices.
\end{lem}

%\begin{proof}
%    The dual of $H_n$ is the board on which the game Hex is played.  The well-known \emph{Hex Lemma} states that any colouring of the vertices of $H_n$ with colours red and blue contains exactly one of the following \cite{Gale79}:
%    \begin{compactenum}
%        \item a path with endpoints $(x,1)$ and $(x',n)$ consisting entirely of red vertices, for some $x,x'\in\{1,\ldots,n\}$; or
%        \item a path with endpoints $(1,y)$ and $(n,y')$ consisting entirely of blue vertices, for some $y,y'\in\{1,\ldots,n\}$.
%    \end{compactenum}
%    In either case, the path contains at least $n$ vertices and therefore has a $n$-vertex subpath consisting entirely of red vertices or entirely of blue vertices.
%\end{proof}

Without loss of generality, assume that the path $R:=p_1,\ldots,p_n$ defined by \lemref{hex_lemma} consists entirely of red vertices, so that $(u_1,p_j)\prec\cdots\prec (u_d,p_j)$ for each $j\in\{1,\ldots,n\}$.
Recall that $(\varphi,\prec)$ is a hypothetical $s$-stack layout of $G$ and therefore it is also an $s$-stack layout of the subgraph $X:=S_d\CartProd R$.  The following result finishes the proof by showing that this is not possible when $n> 2s$ and $d> s2^{n}$.

\begin{lem}
    The graph $X$ contains a set of edges of size at least $\min\{d/2^{n},n/2\}$ that are pairwise crossing with respect to $\prec$.
\end{lem}

\begin{proof}
    We will define two sequences of nested sets $A_1\supseteq \cdots\supseteq A_{n}$ of leaves of $S_d$ so that each $A_i$ satisifies the following conditions:
    \begin{compactenum}[(C1)]
        \item $A_i$ contain $d_i\ge d/2^{i-1}$ leaves of $S_d$.
        \item Each leaf $v\in A_i$ defines an $i$-element vertex set $Z_{i,v}:=\{(v,p_j):j\in\{1,\ldots,i\}\}$.  For any distinct $v,w\in A_i$, $Z_{i,v}$ and $Z_{i,w}$ are \emph{separated} with respect to $\prec$, i.e., $Z_{i,v}\prec Z_{i,w}$ or $Z_{i,v}\succ Z_{i,w}$.
    \end{compactenum}

    Before defining $A_1,\ldots,A_n$ we first show how the existence of the set $A_n$ implies the lemma.  To avoid triple-subscripts, let $d':=d_n\ge d/2^n$.   The set $A_n$ defines vertex sets $Z_{n,v_1}\prec\cdots\prec Z_{n,v_{d'}}$.  Refer to \figref{twister}. Recall that $r$ is the root of $S_b$ so it is adjacent to each of $v_{1},\ldots,v_{d'}$ in $S_b$.  Therefore, for each $j\in\{1,\ldots,n\}$ and each $i\in\{1,\ldots,d'\}$, the edge $(r,p_j)(v_i,p_j)$ is in $X$. Therefore, $(r,p_j)$ is adjacent to an element of each of $Z_{n,v_1},\ldots,Z_{n,v_{d'}}$.
	\begin{figure}
		\begin{center}
			\includegraphics{figs/twister}
		\end{center}
		\caption{The sets $Z_{n,1},\ldots,Z_{n,d'}$ ($n=4$, $d'=5$).}
		\figlabel{twister}
	\end{figure}

    Since $Z_{n,v_1},\ldots,Z_{n,v_{d'}}$ are separated with respect to $\prec$, when viewed from afar, this situation looks like a complete bipartite graph $K_{n,d'}$ with the root vertices $L:=\{(r,p_j):j\in\{1,\ldots,n\}\}$ in one part and the groups $R:=Z_{n,v_1}\cup\cdots\cup Z_{n,v_{d'}}$ in the other part.  Any linear ordering of $K_{n,d'}$ has a large set of pairwise crossing edges so, intuitively, the graph induced by $L\cup R$ should also have a large set of pairwise crossing edges. \lemref{twister}, below, formalizes this and shows that this graph has a set of at least $\min\{d',n\}/2$ pairwise crossing edges.

    All that remains is to define the sets $A_1\supseteq\cdots\supseteq A_n$ that satisfy (C1) and (C2).  The set $A_1$ contains all the leaves of $S_d$.  For each $i\in\{2,\ldots,n\}$, the set $A_i$ is defined as follows:  Let $Z_1,\ldots,Z_r$ denote the sets $\{(v,p_j):j\in\{1,\ldots,i-1\}\}$ for $v\in A_{i-1}$ ordered so that $Z_1\prec\cdots\prec Z_r$.  By Property (C2), this is always possible.	Label the vertices of $A_{i-1}$ as $v_1,\ldots,v_r$ so that $(v_1,p_{i-1})\prec\cdots\prec (v_r,p_{i-1})$.   (This is equivalent to naming them so that $(v_j,p_{i-1})\in Z_j$ for each $j\in\{1,\ldots,r\}$.)  We define the set $A_i:=\{v_{2k+1}:k\in\{0,\ldots,\lfloor(r-1)/2\rfloor\}\}=\{v_{j}\in A_{i-1}:\text{$j$ is odd}\}$.  This completes the definition of $A_1,\ldots,A_n$.

	All that remains is to verify that $A_i$ satisfies (C1) and (C2).  We do this by induction on $i$. The base case $i=1$ is trivial so we assume from this point on that $i\in\{2,\ldots,n\}$.   To see that $A_i$ satisfies (C1) just observe that $|A_i|=\lceil r/2\rceil \ge r/2= |A_{i-1}|/2\ge d/2^{i-1}$.  All that remains is to show that $A_i$ satisfies (C2).

    For each $j\in\{i-1,i\}$, let $H^j:=H_n[\{(v,p_j):v\in A_{i-1}\}]$.
    Recall that, for each $v\in A_{i-1}$, the edge $e_v:=(v,p_{i-1})(v,p_i)$ is in $X$.  We have the following properties:
    \begin{compactenum}[(P1)]
        \item By \lemref{uniform_colour}, $\varphi(e_v)=\varphi_0(p_{i-1},p_i)$ for each $v\in A_{i-1}$.
        \item By the application of \lemref{hex_lemma}, $(v,p_{i-1})\prec (w,p_{i-1})$ if and only if $(v,p_{i})\prec (w,p_{i})$ for each $v,w\in A_{i-1}$.
        \item By \lemref{uniform_order}, $(v,p_{i-1})\prec (v,p_i)$ for every $v\in A_{i-1}$ or $(v,p_{i-1})\succ (v,p_i)$ for every $v\in A_{i-1}$.
    \end{compactenum}
    We claim that these three conditions imply that the vertex sets of $H^{i-1}$ and $H^{i}$ interleave perfectly with respect to $\prec$. More precisely:
	\begin{clm}\clmlabel{interleave} $(v_1,p_{i-1+t})\prec (v_1,p_{i-t}) \prec (v_2,p_{i-1+t}) \prec (v_2,p_{i-t}) \cdots \prec (v_r,p_{i-1+t}) \prec (v_r,p_{i-t})$ for some $t\in\{0,1\}$.
	\end{clm}
	\begin{proof}[Proof of \clmref{interleave}]
		By (P3) we may assume, without loss of generality, that $(v,p_{i-1})\prec (v,p_i)$ for each $v\in A_{i-1}$, in which case we are trying to prove the claim for $t=0$.  It is sufficient, therefore to show that $(v_j,p_i)\prec (v_{j+1},p_{i-1})$ for each $j\in\{1,\ldots,r-1\}$.  For the sake of contradiction, suppose $(v_j,p_{i})\succ (v_{j+1},p_{i-1})$ for some $j\in\{1,\ldots,r-1\}$. By definition $(v_j,p_{i-1})\prec (v_{j+1},p_{i-1})$ so, by (P2)  $(v_{j},p_i) \prec (v_{j+1},p_i)$.  Therefore
		\[
			(v_j,p_{i-1})\prec (v_{j+1},p_{i-1})\prec(v_{j},p_i) \prec
		   (v_{j+1}, p_i) \enspace .
	   	\]
		Therefore the edges $(v_j,p_{i-1})(v_j,p_{i})$ and $(v_{j+1},p_{i-1})(v_{j+1},p_i)$ cross with repect to $\prec$.  But this is a contradiction since, by (P1),  $\varphi((v_j,p_{i-1})(v_j,p_{i})) =\varphi((v_{j+1},p_{i-1})(v_{j+1},p_i))=\varphi_0(p_{i-1}p_i)$.
		This contradiction completes the proof of \clmref{interleave}.
	\end{proof}

% \todo{DW: Why are these $\prec$'s red?  PM: Just me keeping track of which one was the assumption, they don't need to be red.}

	Now, apply \clmref{interleave} and assume without loss of generality that $t=0$, so that
	\[
		(v_1,p_{i-1})\prec (v_1,p_{i}) \prec (v_2,p_{i-1}) \prec (v_2,p_{i}) \cdots \prec (v_r,p_{i-1}) \prec (v_r,p_{i}) \enspace .
	\]

    For each $j\in\{1,\ldots,r-2\}$, $(v_{j+1},p_{i-1})\in Z_{j+1}\prec Z_{j+2}$, so  $(v_j,p_i)\prec (v_{j+1},p_{i-1}) \prec Z_{j+2}$.  Therefore $Z_j\cup\{(v_j,p_i)\} \prec Z_{j+2}$.  By a symmetric argument, $Z_j\cup\{(v_j,p_i)\} \succ Z_{j-2}$ for each  $j\in\{2,\ldots,r\}$.  Finally, since $(v_{j},p_i)\prec (v_{j+2},p_i)$ for each odd $i\in\{1,\ldots,r\}$, we have $Z_{j}\cup\{(v_j,p_i)\} \prec Z_{j+2}\cup\{(v_{j+2},p_i)\}$ for each odd $j\in\{1,\ldots,r-2\}$.  Thus $A_i$ satisifies (C2) since the sets $Z_1\cup\{(v_1,p_i)\},Z_3\cup\{(v_3,p_i)\},\ldots,Z_{2\floor (r-1)/2\rfloor+1} \cup (v_{2\floor (r-1)/2\rfloor+1},p_i)$ are precisely the sets $Z_{i,1},\ldots,Z_{i,d_i}$ determined by our choice of $A_i$.
\end{proof}

\begin{lem}\lemlabel{twister}
    Let $G$ be any graph, let $\prec$ be any linear ordering of $V(G)$,  let $Z_{1}\prec\cdots\prec Z_{2s}$ be subsets of $V(G)$, and let $r_1\prec\cdots\prec r_{2s}$ be vertices of $G$ such that, for each $i,j\in\{1,\ldots,2s\}$, $G$ contains an edge $r_iz_j$ with $z_j\in Z_j$. Then $G$ contains a set of $s$ edges that are pairwise crossing with respect to $\prec$.
\end{lem}

\begin{proof}
    At least one of the following two cases applies (see \figref{median}):
    \begin{enumerate}
        \item $Z_s\prec r_{s+1}$ in which case the graph between $r_{s+1},\ldots,r_{2s}$ and $Z_1,\ldots,Z_s$ has a set of $s$ pairwise-crossing edges.
        \item $r_{s}\prec Z_{s+1}$ in which case the graph between $r_1,\ldots,r_s$ and $Z_{s+1},\ldots,Z_{2s}$ has a set of $s$ pairwise-crossing edges. \qedhere
    \end{enumerate}
\end{proof}
\begin{figure}
	\begin{center}
		\includegraphics{figs/median-1} \\ 1 \\[2em]
		\includegraphics{figs/median-2} \\ 2
	\end{center}
	\caption{The two cases in the proof of \lemref{twister}.}
	\figlabel{median}
\end{figure}

\section{Open Problems}

Recall that every 1-queue graph has a 2-stack layout \citep{HLR92} and we proved that there are 4-queue graphs with unbounded stack-number. The following questions remain open: Do 2-queue graphs have bounded stack-number? Do 3-queue graphs have bounded stack-number?

Given the role of cartesian products in our proof, it is natural to ask when is $\sn(G_1\CartProd G_2)$ bounded? Note that $H_n$ is a subgraph of a planar Hamiltonian graphs (namely, $H_{2n}$), so $\sn(H_n) \leq 2$. So $\sn(G_1\CartProd G_2)$ can be unbounded even when $G_1$ is a star and $\sn(G_2)\leq 2$.
Since $\sn(G_2)\leq 1$ if and only $G_2$ is outerplanar, the following question naturally arises: Is $\sn(S \CartProd H)$ bounded for every star $S$ and outerplanar graph $H$ with bounded degree? Is $\sn(T \CartProd H)$ bounded for every tree $T$ and outerplanar graph $H$ with bounded degree? The assumption that $H$ has bounded degree is needed since $S_n \CartProd S_n$ contain the 1-subdivision of $K_{n,n}$, which has unbounded stack-number~\citep{Blankenship-PhD03}.

MENTION RESULTS OF \citet{Pupyrev20} about bipartite graphs.

Since $H_n\subseteq P \boxtimes P$ where $P$ is the $n$-vertex path, \cref{family} implies that $\sn(S\boxtimes P\boxtimes P)$ is unbounded for stars $S$ and paths $P$. It is easily seen that $\sn(S\boxtimes P)$ is bounded~\citep{Pupyrev20}. The following question naturally arises (independently asked by \citet{Pupyrev20}):
Is $\sn(T \boxtimes P)$ bounded for every tree $T$ and path $P$? We conjecture the answer is ``no''.

%\documentclass[kpfonts]{patmorin}
\usepackage{pat}
\usepackage{paralist,graphicx}
\usepackage{array,longtable}

\usepackage[utf8]{inputenc}
\usepackage{todonotes}

\usepackage[noabbrev,capitalise]{cleveref}
\crefname{lem}{Lemma}{Lemmas}
\crefname{thm}{Theorem}{Theorems}
\crefname{cor}{Corollary}{Corollaries}
\crefname{prop}{Proposition}{Propositions}
\crefname{conj}{Conjecture}{Conjectures}
\crefname{open}{Open Problem}{Open Problems}
\crefname{obs}{Observation}{Observations}

\crefformat{equation}{(#2#1#3)}
\Crefformat{equation}{Equation #2(#1)#3}

\usepackage[numbers,sort&compress]{natbib}
\usepackage{hypernat}
\makeatletter
\def\NAT@spacechar{~}
\makeatother

\setlength{\parskip}{1ex}
\setlength{\parindent}{0ex}


\newtheorem{property}{Property}
\newcommand{\plabel}[1]{\label{prop:#1}}
\newcommand{\pref}[1]{Property~\ref{prop:#1}}

\DeclareMathOperator{\sn}{sn}
\DeclareMathOperator{\qn}{qn}
\DeclareMathOperator{\sqn}{sqn}
\DeclareMathOperator{\tw}{tw}

\renewcommand{\SS}{\mathcal{S}}

\renewcommand{\le}{\leqslant}
\renewcommand{\leq}{\leqslant}
\renewcommand{\ge}{\geqslant}
\renewcommand{\geq}{\geqslant}

\newcommand{\CartProd}{\mathbin{\square}}

\title{\MakeUppercase{Stack-Number is not Bounded by Queue-Number}}

\author{%
	Vida Dujmovi\'c,\!\!%
	\thanks{School of Computer Science and Electrical Engineering,
		University of Ottawa, Ottawa, Canada (\texttt{vida.dujmovic@uottawa.ca}).
		Research supported by NSERC and the Ontario Ministry of Research and Innovation.}
	\,\,
	Robert Hickingbotham,\!\!%
	\thanks{School of Mathematics, Monash University, Melbourne, Australia (\texttt{robert.hickingbotham@monash.edu}).}
	\,\,
	Pat Morin,\!\!%
	\thanks{School of Computer Science, Carleton University, Ottawa, Canada (\texttt{morin@scs.carleton.ca}). Research  supported by NSERC and the Ontario Ministry of Research and Innovation.}
	\,\,
	David R. Wood\thanks{School of Mathematics, Monash University, Melbourne, Australia (\texttt{david.wood@monash.edu}). Research supported by the Australian Research Council.}
}

\begin{document}
\maketitle

\begin{abstract}
We describe a family of graphs with queue-number at most 4 but unbounded stack-number. This resolves open problems of Heath, Leighton and Rosenberg (1992) and Blankenship and Oporwoski (1999).
\end{abstract}

\section{Introduction}

Stacks and queues are fundamental data structures in computer science, but which is more powerful? In 1992, \citet{HLR92} introduced an approach for answering this question by defining the graph parameters \textit{stack-number} and \textit{queue-number}, which respectively measure the power of stacks and queues for representing graphs.

%We now formally define stack and queue layouts of graphs.
Let $G$ be a graph and let $\prec$ be a total order on $V(G)$.  Two disjoint edges $vw,xy\in E(G)$ with $v\prec w$ and $x\prec y$ \emph{cross} with respect to $\prec$ if $v\prec x\prec w\prec y$ or $x\prec v\prec y\prec w$, and \emph{nest} with respect to $\prec$ if $v\prec x\prec y\prec w$ or $x\prec v\prec w\prec y$.
%A set of $k$ pairwise nested edges is called a \emph{$k$-rainbow} and a set of $k$ pairwise crossing edges is called a \emph{$k$-twist}. A set of $k$ edges, no pair of which nest and no pair of which cross is called a \emph{hopper}.
Let $\varphi:E(G)\to\{1,\ldots,k\}$ for some integer $k\ge 1$.  Then $(\prec,\varphi)$ is a \emph{$k$-stack layout} of $G$ if, for every pair of edges $vw,xy\in E(G)$, if $\varphi(vw) = \varphi(xy)$ then $vw$ and $xy$ do not cross.\todo{PM: Suggestion: Replace second if then with $\varphi(vw)\neq\varphi(xy)$ or $vw$ and $xy$ do not cross.} Similarly, the pair $(\prec,\varphi)$ is a \emph{$k$-queue layout} of $G$ if, for every pair of edges $vw,xy\in E(G)$, if $\varphi(vw)=\varphi(xy)$ then  $vw$ and $xy$ do not nest. The smallest integer $s$ for which $G$ has an $s$-stack layout is called the \emph{stack-number} of $G$, denoted  $\sn(G)$. The smallest integer $q$ for which $G$ has a $q$-queue layout is called the \emph{queue-number} of $G$, denoted $\qn(G)$. Note that stack layouts are equivalent to book embeddings, and stack-number is also known as \emph{page-number}, \emph{book-thickness} or \emph{fixed outer-thickness}. \todo{add references}

Given a $k$-stack layout $(\prec,\varphi)$ of a graph $G$, for each $i\in\{1,\dots,k\}$, the set $E_i:= \{e\in E(G):\varphi(e)=i\}$ behaves like a stack, in the sense that each edge $e=vw \in E_i$ with $v\prec w$ corresponds to an element in a sequence of stack opertions, such that if we traverse the vertices in the order of $\prec$, then $e$ is pushed onto the stack at $v$ and popped off the stack at $w$. Similarly, given a $k$-queue layout $(\prec,\varphi)$, each set $E_i$ behaves like a queue. In this way, the stack-number and queue-number  respectively measure the power of stacks and queues to represent graphs.

\todo{Should the following two paragraphs be moved to prior to the definitions\newline PM: Yes}

The following key problems are implcit in the work of \citet{HLR92}, made explicit by \citet{DujWoo05}
\footnote{A \emph{graph parameter} is a function $\alpha$ such that $\alpha(G)\in\mathbb{R}$ for every graph $G$, such that $\alpha(G_1)=\alpha(G_2)$ for all isomorphic graphs $G_1$ and $G_2$. A graph parameter $\alpha$ is \textit{bounded} by a graph parameter $\beta$ if there exists a function $f$ such that for every graph $G$ we have $\alpha(G) \leq f(\beta(G))$.}:
\begin{compactitem}
	\item Is stack-number bounded by queue-number?
	\item Is queue-number bounded by stack-number?
\end{compactitem}

If stack-number is bounded by queue-number but queue-number is not bounded by stack-number, then we would consider stacks to be more powerful than queues. Similarly, if the converse holds, then we would consider queues to be more powerful than stacks. Despite extensive research on stack and queue layouts of graphs (see \citep{DujWoo04,DujWoo-DCG07,DJMMUW20} and the references therein), these fundamental questions have remained unsolved.

\subsection*{Is Stack-Number Bounded by Queue-number?}

This paper considers the first of the above questions. In a positive direction, \citet{HLR92}  showed that every 1-queue graph has a $2$-stack layout. On the other hand, they described graphs that need exponentially more stacks than queues. In particular, $n$-vertex ternary hypercubes have queue-number $O(\log n)$ and stack-number $\Omega(n^{1/9-\epsilon})$ for any $\epsilon>0$.

%Note that, in an $s$-stack layout, $(\prec,\varphi)$, $\varphi$ is a proper $s$-colouring of the auxilliary graph $H$ with vertex set $V(H)=E(G)$ and in which the edge $ef$ if present if and only if $e$ and $f$ cross with respect to $\varphi$.  Any $k$-twist in $G$ with respect to $\prec$ corresponds to a $k$-clique $H$.  Since the chromatic number of any graph is bounded by its clique number, the following observation is trivial:

%\begin{obs}\obslabel{no-s-twist}
%If $(\prec,\varphi)$ is an $s$-stack layout of a graph $G$ then $E(G)$ does not contain any $k$-twist for any $k>s$.
%\end{obs}

%We now formally define stack and queue layouts of graphs. Let $G=(V,E)$ be a graph with disjoint edges $vw, xy$ and a linear ordering $\leq$ of the vertices. Without loss of generality, we may assume that $v < w$, $x <y$ and $v < x$. We say that $vw$ and $xy$ \textit{cross} if $v<x<w<y$,  \textit{nest} if $v <x <y < w$, and are \textit{disjoint} if $v <w<x<y$. \todo{DW: Do we  need ``disjoint''?} A \textit{stack} is a set of pairwise non-crossing edges and a \textit{queue} is a set of pairwise non-nesting edges. A $k$-queue layout of $G$ consist of a linear ordering $\leq$ of its vertices and a partition $E_1,E_2, \dots, E_k$, of its edges into queues with respect to $\leq$. The stack-number of a graph $G$, $\sn(G)$, is the minimum integer $k$ such that $G$ has a $k$-stack layout. Similarly, the queue-number of a graph $G$, $\qn(G)$, is the minimum integer $k$ such that $G$ has a $k$-queue layout. We note that stack layouts of graphs are related to book embeddings of graph and that stack-number is also known as \textit{page-number}, \textit{book-thickness}, or \textit{fixed outer-thickness}.

%Note that stacks and queues are closely related to breadth-first search (BFS) and depth-first search (DFS) layouts of graphs. It can be easily shown that a BFS vertex ordering of a tree admits a $1$-queue layout and a DFS vertex ordering of a tree admits a $1$-stack layout.

%\citet{HLR92} showed that every 1-queue graph has a 2-stack layout. \citet{HLR92} showed that the ternary hypercubes requires exponentially more stacks than queues. In particular, $n$-vertex ternary hypercubes have queue-number at most $2 \log_3 n$, but stack-number at least $\Omega(n^{1/9-\epsilon})$ for any $\epsilon>0$. We prove the following theorem, which shows that stack-number is not bounded by queue-number.

Our key contribution is the following theorem, which shows that stack-number is not bounded by queue-number.
This demonstrates that stacks are not more powerful than queues in the sense discussed above.

\begin{thm}\label{family}
	%  There exists a family $\mathcal{F}$ of graphs for which $\qn(G)\le 4$ for every $G\in\mathcal{F}$, and for every $s\in\N$, there exists $G\in\mathcal{F}$ for which $\sn(G)>s$.
	For every $s\in \N$ there exists a graph $G$ with $\qn(G)\le 4$ and $\sn(G)>s$.
\end{thm}

The graph $G$ in \cref{family} is a cartesian product. For graphs $G_1$ and $G_2$, the \emph{cartesian product} $G_1\CartProd G_2$ is the graph with vertex set $\{(v_1,v_2): v_1 \in V(G_1), v_2 \in V(G_2)\}$, where $(v_1,v_2)(w_1,w_2)\in E(G_1\CartProd G_2)$ if $v_1=w_1$ and $v_2w_2\in E(G_2)$, or $v_1w_1\in E(G_1)$ and $v_2=w_2$.

Let $S_b$ be the star graph with root $r$ and $b$ leaves. For $n\in\N$, let $H_n$ be the dual of the hexagonal grid, defined by $V(H_n):=\{1,\ldots,n\}^2$ and
\begin{align*}
E(H_n) & :=  \{(x,y)(x+1,y):x\in\{1,\ldots,n-1\},\,y\in\{1,\ldots,n\}\} \\
& \qquad \cup \{(x,y)(x,y+1):x\in\{1,\ldots,n\},\,y\in\{1,\ldots,n-1\}\} \\
& \qquad \cup \{(x,y)(x+1,y+1):x,y\in\{1,\ldots,n-1\}\} \enspace .
\end{align*}

The graph $G$ in \cref{family} is $S_b \CartProd H_n$ where $b$ and $n$ are chosen to be sufficiently large compared to $s$, as illustrated in \cref{graph}.
%Note that the Hex graph corresponds to the board used in the game \textit{Hex} which was invented by the  Peit Hein in 1942. In this paper, we prove \Cref{family} with the graph $G:= S_b \square H_n$ where $b$ and $n$ are large compared to $s$.
Note that \citet{Pupyrev20} independently suggested using graph products to show that stack-number is not bounded by queue-number.

\begin{figure}
	\begin{center}
		\begin{tabular}{m{.25\textwidth}m{2ex}m{.25\textwidth}m{2ex}m{.3\textwidth}}
			\includegraphics[width=.25\textwidth]{figs/s} & $\CartProd$ & \includegraphics[width=.25\textwidth]{figs/q} & $=$
			& \includegraphics[width=.3\textwidth]{figs/product}
		\end{tabular}
	\end{center}
	% \includegraphics[width=\textwidth]{figs/figure}
	\caption{$S_9 \CartProd H_4$.}
	\label{graph}
\end{figure}


\subsection*{Subdivisions}

A noteworthy consequence of \Cref{family} is that it resolves a conjecture of \citet{BO99,BO01}. A graph $G'$ is a \textit{subdivision} of a graph $G$ if $G'$ can be obtained from $G$ by replacing the edges $vw$ of $G$ by internally disjoint paths $P_{vw}$ with endpoints $v$ and $w$. If each $P_{vw}$ has exactly $k$ internal vertices, then $G'$ is the \emph{$k$-subdivision} of $G$. If each $P_{vw}$ has at most $k$ internal vertices, then $G'$ is a \emph{$(\leq k)$-subdivision} of $G$. \citet{BO99} conjectured that the stack-number of $(\leq k)$-subdivisions ($k$ fixed)  is not much less than the stack-number of the original graph. More precisely:

\begin{conj}[\citep{BO99}]
\label{B_conj}
There exists a function $f$ such that for every graph $G$ and integer $k$, if $G'$ is any $(\leq k)$-subdivision of $G$, then $\sn(G) \leq f(\sn(G’),k)$.
\end{conj}

\citet{DujWoo05} established a connection between this conjecture and the question of whether stack-number is bounded by queue-number. In particular, they showed that if
\cref{B_conj} is true then stack-number is bounded by queue-number. Since \cref{family} shows that stack-number is not bounded by queue-number, \cref{B_conj} is false. The proof of \citet{DujWoo05} is based on the folllowing key lemma: every graph $G$ has a $3$-stack subdivision with $1+2 \ceil{\log_2\qn(G)}$ division vertices per edge. Applying this result to the graph $G=S_b\CartProd H_n$ in \cref{family},
the $5$-subdivision of $S_b\CartProd H_n$ has a $3$-stack layout. If \cref{B_conj} was true, then $\sn(S_b\CartProd H_n) \leq f( 3,5)$, contradicting \cref{family}.

%Specifically,
%\[
%    \mathcal{F} := \{ S_b\square H_n : b,n\in\N\}
%\]
%where $S_b$ denotes the star with $b$ leaves and $H_n$ is the triangulated $n\times n$ grid.

%\todo[inline]{PM: Does anyone know if there is a standard box operator that is typeset like this $S\boxtimes H$ or $S\boxdot H$ instead of like this $S\square H$ or like this $S\Box H$?  I tried square and Box. DW: I defined \texttt{CartProd} which typesets okay $A \CartProd B$	.}


%The graph $G$ in \cref{family} is obtained as follows (See \figref{graph}): Let $S_b$ denote the star graph with root $r$ and $b$ leaves.  For an even positive integer $n$, let $H_n$ be the $n\times n$ triangulated grid, defined by $V(H_n):=\{1,\ldots,n\}^2$ and
%\begin{align*}
%E(H_n) & :=\{(x,y)(x+1,y):x\in\{1,\ldots,n-1\},\,y\in\{1,\ldots,n\}\} \\
%& \qquad \cup \{(x,y)(x,y+1):x\in\{1,\ldots,n\},\,y\in\{1,\ldots,n-1\}\} \\
%& \qquad \cup \{(x,y+1)(x+1,y):x,y\in\{1,\ldots,n-1\}\} \enspace .
%\end{align*}
%We consider the graph $G:=S_b\CartProd H_n$. That is, $V(G)=V(S_b)\times V(H_n)$ where vertices $(v_1,w_1),(v_2,w_2)\in V(G)$ are adjacent whenever $v_1=w_1$ and $v_2w_2\in E(H_n)$, or $v_1w_1\in E(S_b)$ and $v_2=w_2$.


%SAY SOMETHING ABOUT THE RESULTS OF \citet{Pupyrev20}. I EXPECT WE SOLVE SOME OPEN PROBLEM HERE.\todo{PM:Not really, his open problem is about $H\boxtimes P$ where $H$ has bounded treewidth.}

\subsection*{Is Queue-number Bounded by Stack-Number? }

It remains open whether queues are more powerful than stacks; that is, whether queue-number is bounded by stack-number. Several reults are known about this problem. \citet{HLR92} showed that every 1-stack graph has a 2-queue layout. \citet{DJMMUW20} showed that planar graphs have bounded queue-number. (Note that graph products also feature heavily in this proof.)\ Since 2-stack graphs are planar, this implies that 2-stack graphs have bounded queue-number. It is open whether 3-stack graphs have bounded queue-number. In fact, the case of three stacks is as hard as the general question. \citet{DujWoo05} proved that queue-number is bounded by stack-number if and only if 3-stack graphs have bounded queue-number. Moreover, if this is true then stack-number is bounded by a polynomial function of queue-number.


\section{The Proof}

First we prove that $\qn(S_b\CartProd H_n)\leq 4$, as claimed in \cref{family}. We need the following definition due to \citet{Wood-Queue-DMTCS05}. A queue layout $(\varphi,\prec)$ is \emph{strict} if for every vertex $u\in V(G)$ and for all neighbours $v,w\in N_G(u)$, if $u\prec v,w$ or $v,w \prec u$, then $\varphi(uv)\neq \varphi(uw)$. Let $\sqn(G)$ be the minimum integer $k$ such that $G$ has a strict $k$-queue layout. To see that $\sqn(H_n) \leq 3$, order the vertices row-by-row and then left-to-right within a row, with vertical edges in one queue, horizontal edges in one queue, and diagonal edges in another queue.
\citet{Wood-Queue-DMTCS05} proved that $\qn(G \CartProd H) \leq \qn(G) + \sqn(H)$ for all graphs $G$ and $H$. Of course, $S_b$ has a 1-queue layout (since no two edges are nested for any vertex-ordering). Thus $\qn(S_b \CartProd H_n)\leq 4$.

%\bigskip 
We now turn to the proof of our main result, the lower bound on $\sn(G)$, where $G:= S_b\CartProd H_n$. Consider a hypothetical $s$-stack layout $(\varphi,\prec)$ of $G$ where $n$ and $b$ are chosen sufficiently large compared to $s$ as detailed below. We begin with three lemmata that, for sufficiently large $b$, provide a large subgraph $S_d$ of $S_b$ for which the induced stack layout of $S_d\CartProd H_n$ is highly structured.

For each node $v$ of $S_b$, define $\pi_v$ as the permutation of $\{1,\ldots,n\}^2$ in which $(x_1,y_1)$ appears before $(x_2,y_2)$ if and only $(v,x_1,y_1)\prec (v,x_2,y_2)$.  The following lemma is an immediate consequence of the Pigeonhole Principle:

\begin{lem}\lemlabel{uniform_order}
    There exists a permutation $\pi$ of $\{1,\ldots,n\}^2$ and a set $L_1$ of leaves of $S_b$ of size $b_1\ge \lceil b/(n^2)!\rceil$ such that $\pi_{v}=\pi$ for each $v\in L_1$.
\end{lem}

% \todo[inline]{If we cared, we could improve this to $b/2^{cn^2}$ since we only use the weaker property (P3) in the final proof. DW: I don't care. }

For each leaf $v$ in $L$, let $\varphi_v$ be the edge colouring of $H_n$ defined by $\varphi_v(x,y):=\varphi(v,(x,y))$. Since $H_n$ has maximum degree $6$ and is not 6-regular, it has less than $3n^2$ edges.  Therefore there are fewer than $s^{3n^2}$ edge colourings of $H_n$ using $s$ colours.  Another application of the Pigeonhole Principle proves the following:

\begin{lem}\lemlabel{uniform_colour}
    There exists a subset $L_2\subseteq L_1$ of size $b_2\ge b_1/s^{3n^2}$
    and an edge colouring $\varphi_0:H_n\to\{1,\ldots,s\}$ such that $\varphi_v=\varphi_0$ for each $v\in L_2$.
\end{lem}

The preceding two lemmata ensure that, for distinct leaves $v$ and $w$ of $S_{b_2}$, the stack layouts of the isomorphic graphs $H^v:=G[\{(v,p):p\in V(H_n)]$ and $H^w:=G[\{(w,p):p\in V(H_n)]$ are identical.  The next lemma is a statement about the relationships between the stack layouts of $S^p:=G[\{(v,p):v\in V(S_{b_2})]$ and $S^q:=G[\{(v,q):v\in V(S_{b_2})]$ for  distinct $p,q\in V(H_n)$.  It cannot assert that these two layouts are identical but it does state that they fall into one of two categories.

% \todo{DW: I suggest we replace $H_v$ by $H^v$ and replace $S_p$ by $S^p$.}

\begin{lem}\lemlabel{forward_or_backward}
    There exists a sequence $L_3:=u_1,\ldots,u_{b_3}$ with $\{u_1,\ldots,u_{b_3}\}\subseteq L_2$ of length $b_3\ge b_2^{1/2^{n^2-1}}$ such that, for each $p\in V(H_n)$, either  $(u_1,p)\prec (u_2,p)\prec\cdots\prec (u_{b_3},p)$ or $(u_1,p)\succ (u_2,p)\succ\cdots\succ (u_{b_3},p)$.
\end{lem}

\begin{proof}
    Let $p_1,\ldots,p_{n^2}$ denote the vertices of $H_n$, in any order.
    Begin with the sequence $S_1:=v_{1,1},\ldots,v_{1,d_1}$ that contains all $d_1:=b_2$ elements of $L_2$ ordered so that $(v_{1,1},p_1)\prec\cdots\prec(v_{1,d_1},p_1)$.  For each $i\in\{2,\ldots,n^2\}$, the Erd\H{o}s-Szekeres Theorem~\citep{ES35} implies that, $S_{i-1}$ contains a subsequence $S_i:=v_{i,1},\ldots,v_{i,d_i}$ of length $d_i\ge \sqrt{|S_{i-1}|}$ such that $(v_{i,1},p_i)\prec\cdots\prec(v_{i,d_i},p_i)$ or $(v_{i,1},p_i)\succ\cdots\succ(v_{i,d_i},p_i)$.  It is straightforward to verify by induction on $i$ that $d_i \ge b_2^{1/2^{i-1}}$ resulting in a final sequence $S_{n^2}=:L_3$ of length at least $b_2^{1/2^{n^2-1}}$.
\end{proof}

Let $d:=b_3$ and let $S_d$ be the the subgraph of $S_b$ induced by $\{r\}\cup\{u_1,\ldots,u_{d}\}$ where $u_1,\ldots,u_d$ is the sequence of leaves defined in \lemref{forward_or_backward}.  Consider the (improper) vertex colouring of $H_n$ obtained by colouring each vertex $p\in V(H_n)$ \emph{red} if $(u_1,p)\prec\cdots\prec (u_d,p)$ and colouring $p$ \emph{blue} if $(u_1,p)\succ\cdots\succ(u_d,p)$. We need the following famous Hex Lemma~\citep{Gale79}.

\begin{lem}\lemlabel{hex_lemma}
Every red--blue vertex colouring of the graph $H_n$	contains an $n$-vertex path $R$ consisting entirely of red vertices or entirely of blue vertices.
\end{lem}

%\begin{proof}
%    The dual of $H_n$ is the board on which the game Hex is played.  The well-known \emph{Hex Lemma} states that any colouring of the vertices of $H_n$ with colours red and blue contains exactly one of the following \cite{Gale79}:
%    \begin{compactenum}
%        \item a path with endpoints $(x,1)$ and $(x',n)$ consisting entirely of red vertices, for some $x,x'\in\{1,\ldots,n\}$; or
%        \item a path with endpoints $(1,y)$ and $(n,y')$ consisting entirely of blue vertices, for some $y,y'\in\{1,\ldots,n\}$.
%    \end{compactenum}
%    In either case, the path contains at least $n$ vertices and therefore has a $n$-vertex subpath consisting entirely of red vertices or entirely of blue vertices.
%\end{proof}

Without loss of generality, assume that the path $R:=p_1,\ldots,p_n$ defined by \lemref{hex_lemma} consists entirely of red vertices, so that $(u_1,p_j)\prec\cdots\prec (u_d,p_j)$ for each $j\in\{1,\ldots,n\}$.
Recall that $(\varphi,\prec)$ is a hypothetical $s$-stack layout of $G$ and therefore it is also an $s$-stack layout of the subgraph $X:=S_d\CartProd R$.  The following result finishes the proof by showing that this is not possible when $n> 2s$ and $d> s2^{n}$.

\begin{lem}
    The graph $X$ contains a set of edges of size at least $\min\{d/2^{n},n/2\}$ that are pairwise crossing with respect to $\prec$.
\end{lem}

\begin{proof}
    We will define two sequences of nested sets $A_1\supseteq \cdots\supseteq A_{n}$ of leaves of $S_d$ so that each $A_i$ satisifies the following conditions:
    \begin{compactenum}[(C1)]
        \item $A_i$ contain $d_i\ge d/2^{i-1}$ leaves of $S_d$.
        \item Each leaf $v\in A_i$ defines an $i$-element vertex set $Z_{i,v}:=\{(v,p_j):j\in\{1,\ldots,i\}\}$.  For any distinct $v,w\in A_i$, $Z_{i,v}$ and $Z_{i,w}$ are \emph{separated} with respect to $\prec$, i.e., $Z_{i,v}\prec Z_{i,w}$ or $Z_{i,v}\succ Z_{i,w}$.
    \end{compactenum}

    Before defining $A_1,\ldots,A_n$ we first show how the existence of the set $A_n$ implies the lemma.  To avoid triple-subscripts, let $d':=d_n\ge d/2^n$.   The set $A_n$ defines vertex sets $Z_{n,v_1}\prec\cdots\prec Z_{n,v_{d'}}$.  Refer to \figref{twister}. Recall that $r$ is the root of $S_b$ so it is adjacent to each of $v_{1},\ldots,v_{d'}$ in $S_b$.  Therefore, for each $j\in\{1,\ldots,n\}$ and each $i\in\{1,\ldots,d'\}$, the edge $(r,p_j)(v_i,p_j)$ is in $X$. Therefore, $(r,p_j)$ is adjacent to an element of each of $Z_{n,v_1},\ldots,Z_{n,v_{d'}}$.
	\begin{figure}
		\begin{center}
			\includegraphics{figs/twister}
		\end{center}
		\caption{The sets $Z_{n,1},\ldots,Z_{n,d'}$ ($n=4$, $d'=5$).}
		\figlabel{twister}
	\end{figure}

    Since $Z_{n,v_1},\ldots,Z_{n,v_{d'}}$ are separated with respect to $\prec$, when viewed from afar, this situation looks like a complete bipartite graph $K_{n,d'}$ with the root vertices $L:=\{(r,p_j):j\in\{1,\ldots,n\}\}$ in one part and the groups $R:=Z_{n,v_1}\cup\cdots\cup Z_{n,v_{d'}}$ in the other part.  Any linear ordering of $K_{n,d'}$ has a large set of pairwise crossing edges so, intuitively, the graph induced by $L\cup R$ should also have a large set of pairwise crossing edges. \lemref{twister}, below, formalizes this and shows that this graph has a set of at least $\min\{d',n\}/2$ pairwise crossing edges.

    All that remains is to define the sets $A_1\supseteq\cdots\supseteq A_n$ that satisfy (C1) and (C2).  The set $A_1$ contains all the leaves of $S_d$.  For each $i\in\{2,\ldots,n\}$, the set $A_i$ is defined as follows:  Let $Z_1,\ldots,Z_r$ denote the sets $\{(v,p_j):j\in\{1,\ldots,i-1\}\}$ for $v\in A_{i-1}$ ordered so that $Z_1\prec\cdots\prec Z_r$.  By Property (C2), this is always possible.	Label the vertices of $A_{i-1}$ as $v_1,\ldots,v_r$ so that $(v_1,p_{i-1})\prec\cdots\prec (v_r,p_{i-1})$.   (This is equivalent to naming them so that $(v_j,p_{i-1})\in Z_j$ for each $j\in\{1,\ldots,r\}$.)  We define the set $A_i:=\{v_{2k+1}:k\in\{0,\ldots,\lfloor(r-1)/2\rfloor\}\}=\{v_{j}\in A_{i-1}:\text{$j$ is odd}\}$.  This completes the definition of $A_1,\ldots,A_n$.

	All that remains is to verify that $A_i$ satisfies (C1) and (C2).  We do this by induction on $i$. The base case $i=1$ is trivial so we assume from this point on that $i\in\{2,\ldots,n\}$.   To see that $A_i$ satisfies (C1) just observe that $|A_i|=\lceil r/2\rceil \ge r/2= |A_{i-1}|/2\ge d/2^{i-1}$.  All that remains is to show that $A_i$ satisfies (C2).

    For each $j\in\{i-1,i\}$, let $H^j:=H_n[\{(v,p_j):v\in A_{i-1}\}]$.
    Recall that, for each $v\in A_{i-1}$, the edge $e_v:=(v,p_{i-1})(v,p_i)$ is in $X$.  We have the following properties:
    \begin{compactenum}[(P1)]
        \item By \lemref{uniform_colour}, $\varphi(e_v)=\varphi_0(p_{i-1},p_i)$ for each $v\in A_{i-1}$.
        \item By the application of \lemref{hex_lemma}, $(v,p_{i-1})\prec (w,p_{i-1})$ if and only if $(v,p_{i})\prec (w,p_{i})$ for each $v,w\in A_{i-1}$.
        \item By \lemref{uniform_order}, $(v,p_{i-1})\prec (v,p_i)$ for every $v\in A_{i-1}$ or $(v,p_{i-1})\succ (v,p_i)$ for every $v\in A_{i-1}$.
    \end{compactenum}
    We claim that these three conditions imply that the vertex sets of $H^{i-1}$ and $H^{i}$ interleave perfectly with respect to $\prec$. More precisely:
	\begin{clm}\clmlabel{interleave} $(v_1,p_{i-1+t})\prec (v_1,p_{i-t}) \prec (v_2,p_{i-1+t}) \prec (v_2,p_{i-t}) \cdots \prec (v_r,p_{i-1+t}) \prec (v_r,p_{i-t})$ for some $t\in\{0,1\}$.
	\end{clm}
	\begin{proof}[Proof of \clmref{interleave}]
		By (P3) we may assume, without loss of generality, that $(v,p_{i-1})\prec (v,p_i)$ for each $v\in A_{i-1}$, in which case we are trying to prove the claim for $t=0$.  It is sufficient, therefore to show that $(v_j,p_i)\prec (v_{j+1},p_{i-1})$ for each $j\in\{1,\ldots,r-1\}$.  For the sake of contradiction, suppose $(v_j,p_{i})\succ (v_{j+1},p_{i-1})$ for some $j\in\{1,\ldots,r-1\}$. By definition $(v_j,p_{i-1})\prec (v_{j+1},p_{i-1})$ so, by (P2)  $(v_{j},p_i) \prec (v_{j+1},p_i)$.  Therefore
		\[
			(v_j,p_{i-1})\prec (v_{j+1},p_{i-1})\prec(v_{j},p_i) \prec
		   (v_{j+1}, p_i) \enspace .
	   	\]
		Therefore the edges $(v_j,p_{i-1})(v_j,p_{i})$ and $(v_{j+1},p_{i-1})(v_{j+1},p_i)$ cross with repect to $\prec$.  But this is a contradiction since, by (P1),  $\varphi((v_j,p_{i-1})(v_j,p_{i})) =\varphi((v_{j+1},p_{i-1})(v_{j+1},p_i))=\varphi_0(p_{i-1}p_i)$.
		This contradiction completes the proof of \clmref{interleave}.
	\end{proof}

% \todo{DW: Why are these $\prec$'s red?  PM: Just me keeping track of which one was the assumption, they don't need to be red.}

	Now, apply \clmref{interleave} and assume without loss of generality that $t=0$, so that
	\[
		(v_1,p_{i-1})\prec (v_1,p_{i}) \prec (v_2,p_{i-1}) \prec (v_2,p_{i}) \cdots \prec (v_r,p_{i-1}) \prec (v_r,p_{i}) \enspace .
	\]

    For each $j\in\{1,\ldots,r-2\}$, $(v_{j+1},p_{i-1})\in Z_{j+1}\prec Z_{j+2}$, so  $(v_j,p_i)\prec (v_{j+1},p_{i-1}) \prec Z_{j+2}$.  Therefore $Z_j\cup\{(v_j,p_i)\} \prec Z_{j+2}$.  By a symmetric argument, $Z_j\cup\{(v_j,p_i)\} \succ Z_{j-2}$ for each  $j\in\{2,\ldots,r\}$.  Finally, since $(v_{j},p_i)\prec (v_{j+2},p_i)$ for each odd $i\in\{1,\ldots,r\}$, we have $Z_{j}\cup\{(v_j,p_i)\} \prec Z_{j+2}\cup\{(v_{j+2},p_i)\}$ for each odd $j\in\{1,\ldots,r-2\}$.  Thus $A_i$ satisifies (C2) since the sets $Z_1\cup\{(v_1,p_i)\},Z_3\cup\{(v_3,p_i)\},\ldots,Z_{2\floor (r-1)/2\rfloor+1} \cup (v_{2\floor (r-1)/2\rfloor+1},p_i)$ are precisely the sets $Z_{i,1},\ldots,Z_{i,d_i}$ determined by our choice of $A_i$.
\end{proof}

\begin{lem}\lemlabel{twister}
    Let $G$ be any graph, let $\prec$ be any linear ordering of $V(G)$,  let $Z_{1}\prec\cdots\prec Z_{2s}$ be subsets of $V(G)$, and let $r_1\prec\cdots\prec r_{2s}$ be vertices of $G$ such that, for each $i,j\in\{1,\ldots,2s\}$, $G$ contains an edge $r_iz_j$ with $z_j\in Z_j$. Then $G$ contains a set of $s$ edges that are pairwise crossing with respect to $\prec$.
\end{lem}

\begin{proof}
    At least one of the following two cases applies (see \figref{median}):
    \begin{enumerate}
        \item $Z_s\prec r_{s+1}$ in which case the graph between $r_{s+1},\ldots,r_{2s}$ and $Z_1,\ldots,Z_s$ has a set of $s$ pairwise-crossing edges.
        \item $r_{s}\prec Z_{s+1}$ in which case the graph between $r_1,\ldots,r_s$ and $Z_{s+1},\ldots,Z_{2s}$ has a set of $s$ pairwise-crossing edges. \qedhere
    \end{enumerate}
\end{proof}
\begin{figure}
	\begin{center}
		\includegraphics{figs/median-1} \\ 1 \\[2em]
		\includegraphics{figs/median-2} \\ 2
	\end{center}
	\caption{The two cases in the proof of \lemref{twister}.}
	\figlabel{median}
\end{figure}

\section{Open Problems}

Recall that every 1-queue graph has a 2-stack layout \citep{HLR92} and we proved that there are 4-queue graphs with unbounded stack-number. The following questions remain open: Do 2-queue graphs have bounded stack-number? Do 3-queue graphs have bounded stack-number?

Given the role of cartesian products in our proof, it is natural to ask when is $\sn(G_1\CartProd G_2)$ bounded? Note that $H_n$ is a subgraph of a planar Hamiltonian graphs (namely, $H_{2n}$), so $\sn(H_n) \leq 2$. So $\sn(G_1\CartProd G_2)$ can be unbounded even when $G_1$ is a star and $\sn(G_2)\leq 2$.
Since $\sn(G_2)\leq 1$ if and only $G_2$ is outerplanar, the following question naturally arises: Is $\sn(S \CartProd H)$ bounded for every star $S$ and outerplanar graph $H$ with bounded degree? Is $\sn(T \CartProd H)$ bounded for every tree $T$ and outerplanar graph $H$ with bounded degree? The assumption that $H$ has bounded degree is needed since $S_n \CartProd S_n$ contain the 1-subdivision of $K_{n,n}$, which has unbounded stack-number~\citep{Blankenship-PhD03}.

MENTION RESULTS OF \citet{Pupyrev20} about bipartite graphs.

Since $H_n\subseteq P \boxtimes P$ where $P$ is the $n$-vertex path, \cref{family} implies that $\sn(S\boxtimes P\boxtimes P)$ is unbounded for stars $S$ and paths $P$. It is easily seen that $\sn(S\boxtimes P)$ is bounded~\citep{Pupyrev20}. The following question naturally arises (independently asked by \citet{Pupyrev20}):
Is $\sn(T \boxtimes P)$ bounded for every tree $T$ and path $P$? We conjecture the answer is ``no''.

%\input{sn-vs-qn.bbl}

%%%  Squashing the bibliography 
\let\oldthebibliography=\thebibliography
\let\endoldthebibliography=\endthebibliography
\renewenvironment{thebibliography}[1]{%
	\begin{oldthebibliography}{#1}%
		\setlength{\parskip}{0.1ex}%
		\setlength{\itemsep}{0.1ex}%
	}{\end{oldthebibliography}}
\bibliographystyle{DavidNatbibStyle}
\bibliography{../../BibTeX/myBibliography}

\end{document}


%%%  Squashing the bibliography 
\let\oldthebibliography=\thebibliography
\let\endoldthebibliography=\endthebibliography
\renewenvironment{thebibliography}[1]{%
	\begin{oldthebibliography}{#1}%
		\setlength{\parskip}{0.1ex}%
		\setlength{\itemsep}{0.1ex}%
	}{\end{oldthebibliography}}
\bibliographystyle{DavidNatbibStyle}
\bibliography{../../BibTeX/myBibliography}

\end{document}


%%%  Squashing the bibliography 
\let\oldthebibliography=\thebibliography
\let\endoldthebibliography=\endthebibliography
\renewenvironment{thebibliography}[1]{%
	\begin{oldthebibliography}{#1}%
		\setlength{\parskip}{0.1ex}%
		\setlength{\itemsep}{0.1ex}%
	}{\end{oldthebibliography}}
\bibliographystyle{DavidNatbibStyle}
\bibliography{../../BibTeX/myBibliography}

\end{document}


% OR

\bibliographystyle{DavidNatbibStyle}
\bibliography{../../BibTeX/myBibliography}

\end{document}
