\documentclass[kpfonts]{patmorin}
\usepackage{pat}
\usepackage{paralist}

\usepackage[utf8]{inputenc}

\setlength{\parskip}{1ex}


\DeclareMathOperator{\sn}{sn}
\DeclareMathOperator{\qn}{qn}

\renewcommand{\SS}{\mathcal{S}}


\newcommand{\aref}[1]{(X\ref{a:#1})}
\newcommand{\alabel}[1]{\label{a:#1}}

\title{\MakeUppercase{Stack Number is not Queue Number Bounded}}
\author{TBD}

\begin{document}
\maketitle

\begin{abstract}
  We describe a family of graphs in which every member has queue number at most 4, but for every integer $s$, there is a member of the family whose stack number is greater than $s$.
\end{abstract}

\section{Introduction}


\section{Basic Definitions}


\subsection{Total Orders: Separation and Interleaving}
For any total order $\prec$ on a set $Z$ and any $x,y\in Z$, we use the interval notation $[x,y]_\prec:=\{v\in Z: x\preceq v\preceq y\}$.  For any subset $X\subseteq Z$, $\min_\prec(X)$ is the unique value in $\ell\in X$ such that $\ell\prec x$ for every $x\in X\setminus\{\ell\}$ and $\max_\prec(X)$ is the unique value in $r\in X$ such that $x\prec r$ for every $x\in X\setminus\{r\}$.  In the remainder of the paper we will only be working one partial order $\prec$ that is used in a hypothetical $s$-stack layout, so we will omit the $\prec$ subscript on $\min$, $\max$ and $[x,y]$.

Let $X\cup Y$ be a partition of $Z$.  We say that $X$ and $Y$ are \emph{separated} (with respect to $\prec$) if $[\min(X),\max(X)]\cap [\min(Y),\max(Y)]=\emptyset$.  We say that $X$ and $Y$ \emph{interleave} (with respect to $\prec$) if, for some integers $\ell,c\ge 1$, $|X|=\ell$, $|Y|=c\ell$, $Y$ can be partitioned into sets $Y_1,\ldots,Y_{\ell}$, each of size $c$, and the elements of $X$ can be ordered as $x_1,\ldots,x_\ell$ so that 
\[  
  x_1\prec \min(Y_1) \preceq \max(Y_1) \prec x_2 \prec \cdots \prec x_{\ell} \prec \min(Y_\ell) 
\]
or 
\[  
  \max(Y_1) \prec x_1 \prec\min(Y_2)\preceq \max(Y_2)\prec x_2 \prec \cdots \prec x_{\ell-1} \prec \min(Y_\ell) \preceq \max(Y_\ell) \prec x_\ell \enspace . 
\]

\subsection{Linear Layouts of Graphs: Stacks and Queues}

Let $G$ be a graph and let $\prec$ be a total order on $V(G)$.  Two edges $vw,xy\in E(G)$ with $v\prec w$ and $x\prec y$ \emph{cross} with respect to $\prec$ if $v\prec x\prec w\prec y$ or $x\prec v\prec y\prec w$. Two edges $vw,xy\in E(G)$ with $v\prec w$ and $x\prec y$ \emph{nest} with respect to $\prec$ if $v\prec x\prec y\prec w$ or $x\prec v\prec w\prec y$.  Two edges of $G$ are \emph{disjoint} with respect to $\prec$ if they do not cross and they do not nest. A set of $k$ pairwise nesting edges is called a \emph{$k$-rainbow} and a set of pairwise crossing edges is called a \emph{$k$-twist}.

Let $\varphi:E(G)\to\{1,\ldots,k\}$ for some integer $k\ge 1$.  Then $(\prec,\varphi)$ is a \emph{$k$-stack layout} of $G$ if, for every pair of edges $vw,xy\in E(G)$, $vw$ and $xy$ don't cross or $\varphi(vw)\neq\varphi(xy)$. The pair $(\prec,\varphi)$ is a $k$-queue layout of $G$ if, for every pair of edges $vw,xy\in E(G)$, $vw$ and $xy$ don't nest or $\varphi(vw)\neq\varphi(xy)$.  The smallest integer $s$ for which $G$ has an $s$-stack layout is called the \emph{stack number} of $G$ and is denoted by $\sn(G)$.  The smallest integer $q$ for which $G$ has a $q$-queue layout is called the \emph{queue number} of $G$ and is denoted by $\qn(G)$.  

Note that, in an $s$-stack layout, $(\prec,\varphi)$, $\varphi$ is a proper $s$-colouring of the auxilliary graph $H$ with vertex set $V(H)=E(G)$ and in which the edge $ef$ if present if and only if $e$ and $f$ cross with respect to $\varphi$.  Any $k$-twist in $G$ with respect to $\prec$ corresponds to a $k$-clique $H$.  Since the chromatic number of any graph is bounded by its clique number, the following observation is trivial:

\begin{obs}\obslabel{no-s-twist}
  If $(\prec,\varphi)$ is an $s$-stack layout of a graph $G$ then $E(G)$ does not contain any $k$-twist for any $k>s$.
\end{obs}


\subsection{Strong Products: Graphs with Paths}

The \emph{strong product} of two graphs $G$ and $H$, denoted $G\boxtimes H$ is the graph with vertex set $V(G\boxtimes H)=V(G)\times V(H)$ and in which the edge $(v,x)(w,y)$ is in $E(G\boxtimes H)$ if and only if
\begin{enumerate}
  \item $vw\in E(G)$ and $x=y$;
  \item $v=w$ and $xy\in E(H)$; or
  \item $vw\in E(G)$ and $xy\in E(H)$.
\end{enumerate}
More concisely: If $d_G(v,w)$ denotes the distance from $v$ to $w$ in $G$, then $(v,x)(w,y)\in E(G\boxtimes H)$ if and only if $d_G(v,w)\le 1$ and $d_H(x,y)\le 1$.

The $n$-vertex path $P_n$ (of length $n-1$) is an important part of this paper and use the convention that the vertices of $P_n$ are the integers $1,\ldots,n$ in the order they appear on the path. That is is, $P_n$ has vertex set $V(P_n)=\{1,\ldots,n\}$ and edge set $E(P_n)=\{(i,i+1):i\in\{1,\ldots,n-1\}\}$.

The following is a special case of a more general result of Dujmović \etal~\cite[Lemma~8]{XXX}:

\begin{cor}\corlabel{queue-product}
  For any graph $H$ and any integer $n\ge 1$, $\qn(H\boxtimes P_n)\le 3\cdot\qn(G)+1$. 
\end{cor}

\section{Overview of the Proof}

Our strategy, is to show that, for any $s\in\N$, there exists $b,h,n\in\N$ such that $G:=G_{b,h,n}:=T\boxtimes P_n$ has stack-number greater than $s$, where $T$ is a complete $b$-ary tree of height $h$. For any tree $T$, $\qn(T)=1$.  Therefore, by \corref{queue-product}, $\qn(G)\le 4$.  In the following, most quantities are increasing functions of $s$ and some of them are increasing functions of $h$ or $n$.  None of the quantities is an increasing function of $b$.  More specifically, in many places we treat $s,h,n$ as fixed and choose $b$ arbitrarily large.

\paragraph{The special subtree $T'$ and the mapping $f$:}
We first fix a hypothetical $s$-stack layout $(\prec,\varphi)$ of $G$ and, using this, we describe a complete $b'$-ary subtree $T'$ of $T$ (also of height $h$) that has many special properties.  (Here $b'\in\omega_b(1)$ grows with $b$.)  

We now describe some of the special properties of $T'$.  For each $i\in\{1,\ldots,n\}$, let $V'_i=\{(v,i):v\in V(T')\}$, and for each $d\in\{0,\ldots,h\}$, let $V'_{i,d}\subseteq V'_i$ consist of the depth-$d$ nodes in $V'_i$. By carefully choosing $T'$ we are able to define a mapping $f:\{1,\ldots,n\}\times \{1,\ldots,h\}\to \{1,\ldots,nh\}$ with the following properties:
\begin{enumerate}
  \item If $f(i,d)\neq f(j,e)$ then the sets $V'_{i,d}$ and $V'_{j,e}$ are separated with respect to $\prec$.
  \item If $f(i,e)= f(j,e)$ then the sets $V'_{i,d}$ and $V'_{j,e}$ interleave\footnote{Define what it means for two sets of unequal size to interleave.} with respect to $\prec$.
\end{enumerate}
During this process, we will make extensive use of the ability to choose $b$ arbitrarily large. 

\paragraph{Properties of $f$:}

Next we will show that $f$ must be at most $s'$-to-one for some $s'$ dependent on $s$.\footnote{Watch out here: $s'$ can't depend on $h$, or we're sunk.}  That is, for every $x\in\{1,\ldots,nh\}$, there are at most $s'$ pairs $(i,d)$ such that $f(i,d)=x$.  Indeed, otherwise, we have $s'$ sets $V'_{i,d}$ that pairwise interleave.  Unsurprisingly, this implies that $G$ contains a twist of size $\omega_1(s')$.  In particular, for large enough $s'$, $G$ contains a twist of size greater than $s$. By \obsref{no-s-twist}, this contradicts the assumption that $(\prec,\varphi)$ is an $s$-stack layout.

\paragraph{The grid $Q$ and derived graph $Q_f$:}
Next we will show that the structure of the $\boxtimes$ operator defines a graph $Q$ with vertex set $V(Q)=\{(i,d):i\in\{1,\ldots,n\},\, d\in\{1,\ldots,h\}\}$ and that contains every edge $(i,j)(i',j')$ such that $\max\{|i-i'|,|j-j'|\}\le 1$. In other words, $Q:=P_n\boxtimes P_h$ has the structure of an $n\times h$ grid, including diagonals. Note that $Q$ is very far from being bipartite: It contains many 3-cycles.

The graph $Q$ and the function $f$ defines a graph $Q_f$ with vertex set $V(Q_f):=\{1,\ldots,nh\}$ and in which the edge $pq$ is present if and only if there exists an edge $ab\in E(Q_f)$ with $f(a)=p$ and $f(b)=q$.\footnote{This has a name: $f$ is a homomorphism of $Q$ onto the graph $Q_f$ (where $Q_f$ has self-loops at each vertex.}  We will show that the graph $Q_f$ must be bipartite.  The final step in the proof is to show that there is no $s'$-to-one function $f$ that makes $Q_f$ bipartite.  Roughly:  Any such function $f$ must take $Q$ onto a graph $Q_f$ that is path of length at most $n$.  But this implies that some vertex of $Q_f$ is the image of at least $|V(Q)|/n=h$ vertices of $Q$. Choosing $h\ge s'$ contradicts the requirement that $f$ is $s'$-to-one thus completing the proof.
% 
% 
% We need:
% \begin{align*}
%     s'& =f(s) \\ 
%     h/n&\ge s',\enspace
% \end{align*}

The remainder of the paper works through these steps in the proof.


\subsection{Getting Started}

\begin{lem}
  Let $S$ be a $b$-ary star with root $r$ and leaf set $L=\{v_1,\ldots,v_b\}$ and let $(\prec,\varphi)$ be an $s$-stack layout of $G:=S\boxtimes P_2$.  Then, there exists $L'\subseteq L$ such that $|L'|\ge \tfrac{1}{2}\sqrt{b/s}$ and the two sets $L'_1:=\{(v,1):v\in L\}$ and $L'_2=\{(v,2):v\in L\}$ interleave or are separated with respect to $\prec$.  
\end{lem}

\begin{proof}
  Select the edge-colour $c\in\{1,\ldots,s\}$ that maximizes the size of the set $L_c:=\{ v\in L: \varphi((v,1)(v,2))=c\}$.  By the Pigeonhole Principle, $|L_c|\ge b/s$.  The subgraph $G[L_c]$ is a matching in which no two edges cross with respect to $\prec$.  By Dilwerth's Theorem, $E(G[L_c])$ contains a subset $M$ size at least $\sqrt{b/s}$ that is a rainbow or a matching with respect to $\prec$.

  The edge set $M$ can be partitioned into two sets $M_1$ and $M_2$ such that $M_1$ contains each edge $(v,1)(v,2)\in M$ such that $(v,1)\prec (v,2)$ and $M_2$ contains each edge such that $(v,2)\prec (v,1)$.  For some $a\in\{1,2\}$, $M_a$ has size at least $\tfrac{1}{2}\sqrt{b/s}$. 
    
  Take $L':=\{v: (v,1)(v,2)\in M_a\}$.  If $M_a$ forms a rainbow with respect to $\prec$ then $L'_1$ and $L'_2$ that are separated and therefore satisfy the conclusion of the lemma.  If $M_a$ forms a set that is disjoint with respect to $\prec$, then $L'_1$ and $L'_2$ interleave and therefore also satisfy the conclusion of the lemma.
\end{proof}


\begin{lem}\lemlabel{large-complete-subtree}
  Let $T$ be a complete $b$-ary tree of height $h$ and let $L$ be a set of at most $(b-r)^h$ leaves of $T$.  Then $T-L$ contains a complete $r$-ary tree of height $h$. 
\end{lem}

\begin{proof}
  The proof is by induction on $h$.  For the case $h=1$, $T$ is a $b$-ary star and $L$ contains at most $b-r$ leaves of $T$.  Therefore, $T-L$ is a star with at least $r$ leaves.
  
  For $h\ge 1$, consider the set of $b^{h-1}$ nodes at depth $h-1$.  Say that such a node $v$ is \emph{bad} if more than $b-r$ of $v$'s children are in $L$ and \emph{good} otherwise.  The set $L'$ of bad nodes has size at most
  \[  |L'|\le \frac{|L|}{b-r} \le \frac{(b-r)^h}{b-r} = (b-r)^{h-1} \]
  By applying the inductive hypothesis with $h-1$ and $L'$ and we find a complete $r$-ary subtree $T'$ of height $h-1$ whose leaves are all good nodes.  By definition, each good node has at least $r$ children not in $L$, so $T-L$ contains a a complete $r$-ary subtree of height $h$.
\end{proof}  


\begin{lem}\lemlabel{layer-colouring}
  For any $r,s\in\N$, there exists $b_0\in\N$ such that the following is true for every integer $b\ge b_0$.  Let $T$ be a complete $b$-ary tree of height $h$ and let $\varphi:E(T)\to\{1,\ldots,s\}$.  Then $T$ contain a complete $r$-ary subtree $T$ of height $h$ such that, for any $d\in\{0,\ldots,h-1\}$ and any two edges $e,f\in E(T')$ whose four endpoints each have depths in $\{d,d+1\}$, $\varphi(e)=\varphi(f)$. 
\end{lem}

\begin{proof}
  For each root to leaf path $v_0,\ldots,v_h$ in $T$, define the colour sequence
  \[  c(v_h):=(\varphi(v_0v_1),\varphi(v_1v_2),\ldots,\varphi(v_{h-1}v_h)) \enspace .
  \]  
  These sequences partition the leaves of $T$ into $s^h$ classes.  At least one of these classes has size at least $(b/s)^h$.  The set $L$ of leaves not belonging to this class has size at most $b^h-(b/s)^h=(1-(1/s)^h)b^h$.  By \lemref{large-complete-subtree}, $T-L$ contains a complete $r$-ary subtree of height $h$ provided that $|L|\le (b-r)^h=(1-r/b)^hb^h$, which is true provided that 
  \begin{equation}\eqlabel{fatty}
     1-(1/s)^h \le (1-r/b)^h \enspace .
  \end{equation}  
  For any positive integers $s$, $h$, and $r$, $1-(1/s)^h<1$ and $\lim_{b\to\infty} (1-r/b)^h=1$.  Therefore, any sufficiently large integer $b$ satisifies \eqref{fatty}. 
\end{proof}

Let $\prec$ be a total order on $V(T)$ for some tree $T$, let $W$ be the set of depth-$d$ nodes in $T$ and let $Y$ be the set of depth-$d'$ nodes in $T$, for some $d'>d$.  We say that $\prec$ \emph{reverses} $W$ and $Y$ if, for every $x,y\in Y$ with respective ancestors $v,w\in W$, $x\prec y$ if and only if $w\prec v$.  We say that $\prec$ is \emph{consistent} for $W$ and $Y$ if $x\prec y$ if and only if $v\prec w$.  We say that 

\begin{lem}\lemlabel{consistent-interleave}
  Let $T$ be a complete $b$-ary tree of height $h$, let $\prec$ be a total order on $V(T)$ and let $T'$ be a complete $r$-ary subtree of $T$ of height $h$.  If $\prec$ is consistent for the set $V_i$ of depth-$i$ nodes of $T$ and the set $V_j$ of depth-$j$ nodes of $T$ and $V_i$ and $V_j$ interleave with respect to $\prec$, then the sets $V_i'$ of depth-$i$ nodes in $T'$ and the set $V_j'$ of depth-$j$ nodes in $T'$ also interleave with respect to $\prec$.
\end{lem}

\begin{proof}
  Indeed, interleaving implies that between two consecutive nodes of $V_i$ there are exactly $r^{j-i}$ nodes of $V_j$, beginning with a single node $v\in V_i$ followed by a block $x_1,\ldots,x_{b^{j-i}}$ of $b^{j-i}$ nodes of $V_j$ or vice-versa.  In either case, consistency implies that $v$ is an ancestor of $x_1,\ldots,x_{b^{j-i}}$.  If $v\in T'$ then exactly $r^{j-i}$ elements of $x_1,\ldots,x_{b^{j-i}}$ are in $T'$.  If $v\not\in T'$ then none of  $x_1,\ldots,x_{b^{j-i}}$.  Thus, $\prec$ separates any two consecutive elements of $V'_i$ with exactly $r^{j-i}$ elements of $V'_j$, i.e., $V'_i$ and $V'_j$ interleave.
\end{proof}

At this point, it is convenient to treat the vertices of $T$ (and later of $G$) as real numbers in $[0,1)$ chosen to respect $\prec$, so that for any $v\in V(T)$, $v\prec w$ if and only if $v < w$.  We use this view from this point onward.

\begin{lem}
  For every $r,h\in\N$, there exists $b_0\in\N$ such that the following is true for every integer $b\ge b_0$.  Let $T$ be a complete $b$-ary tree of height $h$ and let $(\prec,\varphi)$ be an $s$-stack layout of $T$ where $\varphi$ satisfies the conditions of \lemref{layer-colouring}.  
  
  Then there exists an $r$-ary subtree $T'$ of $T$ and a partition $I$ of $[0,1)$ into at most $f(h)$ intervals, such that, for any $d\in\{1,\ldots,h-1\}$, the depth-$d$ nodes of $T'$ are contained in a single interval of $I$.
  
  Furthermore, for each pair $i,j\in\{1,\ldots,h\}$, $i\neq j$ at least one of these two conditions is satisified:
  \begin{compactenum}[(O1)]
    \item $\prec$ reverses $V_i$ and $V_j$; or
    \item $\prec$ is consistent for $V_i$ and
    \begin{compactenum}
      \item If $V_j, V_i\in [x,y)$ for some interval $[x,y)\in I$ then $V_i$ and $V_j$ interleave.
    \end{compactenum}
  \end{compactenum}
  
  TODO: Clean this up.  With only a tiny amount of work, we can ensure that any two levels assigned to the same interval will be consistent.
\end{lem}

\begin{proof}
  The proof is by induction on $h$.  The case $h=1$ is trivial: Take $b\ge b_0:=b'$, let $I:=\{[0,1)\}$ and $f(1)=[0,1)$, and any subtree $T'$ of $T$ with $b'$ leaves.
    
  For $h>1$ apply the inductive hypothesis with a larger value $r=t$, for some integer $t$ to be discussed shortly, to the first $h-1$ levels of $T$ to obtain the partition $I$ and a $t$-ary subtree $T'$.
  
  Now, $T'$ has $t^{h-1}$ leaves, and each of these has $b$ children in $T$.  The partition $I$ partitions these $bt^{h-1}$ leaves of $T$ into $h-1$ sets and one of these sets, $X$, has size at least $bt^{h-1}/(h-1)$. More precisely, there is some interval $[x,y)\in I$ such that $v\in [x,y)$ for at least $bt^{h-1}/(h-1)$ of the children (in $T$) of the leaves of $T'$.
  
  Let $t'>r$ be defined later.  Say that a leaf $v$ of $T'$ is \emph{bad} if less than $t'$ of its children belong to $X$ and $v$ is \emph{good} otherwise.  The set $L$ of bad nodes has size at most 
  \[ |L|\le \frac{bt^{h-1} - bt^{h-1}/(h-1)}{b-t'} = \frac{(1-1/(h-1))}{b-t'}bt^{h-1} \le (t-t')^{h-1}\]
  for $b\gg t\gg (h-1)t'$.  In particular, $\lim_{b\to\infty} b/(b-b')=1$ so, for fixed $t'$,
  \[ \lim_{b\to\infty} |L| = (1-(1/(h-1))t^{h-1} = (t^{h-1}-(t^{h-1}/(h-1)))
     \le (t-t')^{h-1} \]
  for $t\ge 2t'(h-1)$.  Therefore, we can apply \lemref{large-complete-subtree} to find a complete $t'$-ary subtree of $T'$ of height $h-1$ and a this can be completed to a complete $t'$-ary subtree $T''$ of $T$ of height $h$.
  
  Now, the leaves of $T''$ are all contained in the interval $[x,y)\in I$.  The non-leaf nodes of $T''$ are also nodes in $T'$, so all the depth-$d$ nodes of $T''$ belong to a single interval in $I$, for each $d\in\{1,\ldots,h-1\}$. Condition~O1 is satisified for pairs $1\le i< j<h$.  By \lemref{consistent-interleave}, Condition~O2 is satisified for all pairs $1\le i<j< h$.
  
  All that remains is to show how to satisify Condition~1 and 2 for pairs $j,h$.
  Let $V_h''$ be the set of depth-$h$ nodes in $T''$. There are two cases to consider:
  \begin{enumerate}
    \item The set $V_{h-1}''$ of depth-$h-1$ nodes of $T''$ is not contained in $[x,y)$.  In this case $\prec$ reverses $V_h''$ and $V_{h-1}''$. Define $X=V_{h-1}''$.
    
    \item The set $V_{h-1}''$ of depth-$h-1$ nodes of $T''$ is contained in $[x,y)$.  In this case, we apply Dilwerth's Theorem to find a set subset of $X\subseteq V''_h$ of size at least $\sqrt{t'}^{h-1}$ such that $\prec$ reverses $V_{h-1}''$ and $V_{h}''$ or is consistent for $V_{h-1}''$ and $V_{h}''$.  In this case, we thin $T''$ again to get a $t''$-ary tree, which works provided that $t'\gg (t'')^2$.
  \end{enumerate}
  
  Finally, $X$ is the set of leaves in a tree that satisifies all the Conditions of the Lemma except possibly (O2).  To satisfy (O2), find the largest index $j$ such that $V_j''\in [x,y)$ and $\prec$ is consistent for $V_j''$ and $V_h''$.  If not such $j$ exists, then the tree $T''$ already satisfies all the conditions of the lemma. Otherwise, for each node $v$ in $V_j''$, consider the minimal interval that contains $v$ and all of $v$'s descendants in $V_h''$.  Apply Dilwerth's Theorem to the poset defined by these intervals.  If this poset contains a large chain, this corresponds to a large subset of $V_j''$ that interleaves with an appropriately-sized subset of $V_h''$.  Again, we thin $T''$ again to get a tree of smaller arity that satisifies the requirements of the lemma.
  
  If this poset contains a large anti-chain, then this corresponds to a subset of $V_{j}''$ that is separated from its descendants in $V_{h}''$.  Again, we thin $T''$ to get a tree of smaller arity that satisifies the requirements of the lemma.
\end{proof}  

\end{document}








For every $d,h\in\N$, let $T_{d,h}$ denote the complete $d$-ary tree of height $h$ and let $F_{d,h}$ denote the forest containing $d$ complete $d$-ary trees of height $h$.

% The following lemma shows that, for sufficiently large $d'$, every $s$-stack layout of $T_{d',h}$ contains a 1-stack layout of $T_{d,h}$.
% 
% \begin{clm}\clmlabel{one-stack-subtree}
%   For every $d,h,s\in\N$, there exists $d'=d'(d,h,s)$ such that, for any $s$-stack layout $(\prec,\varphi)$ of $T_{d',h}$, $T_{d',h}$ contains a subgraph $T'$ isomorphic to $T_{d,h}$ such that the edges of $T'$ are pairwise non-crossing with respect to $\prec$.  
% \end{clm}
% 
% \begin{proof}
%   Let $(\prec,\varphi)$ be an $s$-stack layout of $T_{d,h}$ and let $r$ be the root of $T_{d,h}$. For each child $v$ of $r$, let $\varphi(v)$ be a colour $c\in\{1,\ldots,s\}$ such that $\varphi(vw)=c$ for at least $d'/s$ children $w$ of $v$.  Now, let $c\in\{1,\ldots,s\}$ be chosen so that $\varphi(v)=c$ for at least $d'/s$ children of $r$.
% 
% 
% 
%   By the pigeonhole principle, there exists some colour $c\in\{1,\ldots,s\}$ and at least $k\ge \lceil d'/s\rceil$ children $v_1,\ldots,v_k$ of $r$ such that $\varphi(rv_i)=c$ for each $i\in\{1,\ldots,k\}$.
% 
% 
% 
%   of the edges $e$ incident to $r$.  Let $v_1,\ldots,v_k$ be the endpoints of these edges.
% 
% 
% 
%   Work down from the root, repeatedly colouring a node with the edge colour that occurs the most frequently at that node and then keeping the most-frequently occurring node colour at each level\ldots.
% \end{proof}

For each $n\in\N$, let $P_n$ denote the path of length $n$ whose vertices are $1,2,3,\ldots,n$ in this order.  

% For a graph $G=H\boxtimes P_n$, and a subgraph $H'\subseteq H$, we let $H'_i$ denote the $i$th copy of $H'$ in $G$.  That is, $V(H'_i)=\{(v,i):v\in V(H')\}$ and $E(H'_i)=\{(v,i)(w,i): vw\in E(H')\}$.

For any total order $(V,\prec)$ and any $x,y\in V$, we use the interval notation $[x,y]_\prec:=\{v\in V(G): x\preceq v\preceq y\}$.  For any subset $X\subseteq V$, $\min_\prec X$ is the unique value in $\ell\in X$ such that $\ell\prec x$ for every $x\in X\setminus\{\ell\}$ and $\max_\prec X$ is the unique value in $r\in X$ such that $x\prec r$ for every $x\in X\setminus\{r\}$.

Let $A$ and $B$ be two disjoint sets with $|A|=|B|$, let $X=A\cup B$, and let $\prec$ be a total order over $X$.  We say that $A$ and $B$ are \emph{separated} with respect to $\prec$ if $[\min_\prec A,\max_prec A]_\prec\cap [\min_\prec B,\max_\prec B]_\prec=\emptyset$. We say that $A$ and $B$ \emph{interleave} with respect to $\prec$ if there exists an orderings $a_1,\ldots,a_n$ and $b_1,\ldots,b_n$ of $A$ and $B$, respectively, such that $a_1\prec b_1\prec a_2\prec b_2\prec\cdots\prec a_n\prec b_n$ or $b_1\prec a_1\prec b_2\prec a_2\prec\cdots\prec b_n\prec a_n$.  

% \begin{clm}\clmlabel{early-interleave}
%   For every $s,d\in\N$, there exists $d',h\in\N$ such that, for any $s$-stack layout $(\prec,\varphi)$ of $G=F_{d',h}\times P_2$, $F_{d',h}$ contains a copy $F'$ of $F_{d,h}$ such that, for some $i\in\{0,\ldots,h\}$, the depth-$i$ nodes of $F'_1$ and $F'_2$ interleave with respect to $\prec$.
% \end{clm}


\begin{lem}\lemlabel{star-forest}
  For every $d\in\N$, there exists $h,d'\in\N$ such that the following is true: Let $(\prec,\varphi)$ be an $s$-stack layout of $G:=F\boxtimes P_2$ where $F:=F_{d',1}$ is a forest of $d'$ $d'$-ary stars.  Then at least one of the following is true:
  \begin{compactenum}[(X1)]
    \item There exists a set $S$ of $d$ roots in $F$ such that $S_1=\{(v,1):v\in S\}$ and $S_2=\{(v,1):v\in S\}$ interleave with respect to $\prec$;
    \alabel{a1}
    
    \item There exists a subgraph $F'$ of $F$ isomorphic to $F_{\delta d',1}$ with leaf set $L$ such that $L_1=\{(v,1):v\in L\}$ and $L_2=\{(v,2):v\in L\}$ interleave with respect to $\prec$;
    \alabel{a2}

    \item There exists a subgraph $F'$ of $F$ isomorphic to $F_{\delta d,1}$
    with root set $R$ and such that, for each node $v$ in $R$ with leaf set $L_v$, $\{(x,1)(x,2):x\in L_v\}$ forms a rainbow with respect to $\prec$ and the vertex sets $C_v=\bigcup_{x\in L_v} \{(x,1)(x,2)\}$ and $C_w=\bigcup_{x\in L_w} \{(x,1)(x,2)\}$ are separated for each $v,w\in R$, $v\neq w$.\label{a3}
        
    \item There exists a subgraph $F''$ of $F$ isomorphic to $T_{d'/(2d),1}$ such that $V_1=\{(v,1):v\in V(F'')\}$ and $V_2=\{(v,2):v\in V(F'')\}$ are separated with respect to $\prec$.
    \alabel{a4}
  \end{compactenum}
\end{lem}


\begin{proof}
  Consider the set $V_0$ of $d'$ roots of trees in $F$.  Select a subset of $V_0'\subseteq V_0$ with the following properties:
  \begin{compactenum}[(C1)]
    \item Consistent ordering: For some $(a,b)\in\{(1,2),(2,1)\}$, $(v,a)\prec (v,b)$ for every $v\in V'_0$; 
    \item Common edge colouring: $\varphi((v,1)(v,2))=\varphi((w,2)(w,2))$ for each $v,w\in V'_0$; 
    \item Large size: $|V'_0|\ge d'/(2s)$.
  \end{compactenum}
  The Pigeonhole Principle guarantees the existence of a set of size $d'/2$ satsifying the first property and a second application of the Pigeonhole Principle ensures the existence of a subset of size $d'/(2d)$ satisfying the second property.
  
  Define the partial order $\prec'$ on $R'_0$ where $v\prec' w$ if and only if $(v,1)\prec (w,1)$ and $(v,2)\prec (w,2)$.  Observe that, if $v\prec' w$, then the edges $e_v =(v,1)(v,2)$ and $e_w=(w,1)(w,2)$ are disjoint with respect to $\prec$.  Indeed, $\varphi(e_v)=\varphi(e_w)$ so $e_v$ and $e_w$ do not cross and $v\prec' w$ so $e_v$ and $e_w$ do not nest.
  
  If $prec'$ contains a chain $S$ of length $d$, then we are done because $S$ satisfies \aref{a1}.  Indeed, the fact that $S$ forms a chain in $\prec'$ implies that the edges $\{(v,1)(v,2):v\in S\}$ are disjoint and (C1) ensures that $(v,1)\prec (v,1)$ for every $v\in S$ or $(v,2)\prec (v,1)$ for every $v\in S$.
    
  Thus, we may assume that $\prec'$ contains no chain of size $d$ so Dilworth's Theorem \cite{dilwerth:decomposition}, implies that $\prec'$ contains an antichain $A_0$ of size at least $|V_0'|/d=d'/(2sd)$.  Let $E_0=\{(v,1)(v,2):v\in V_0'\}$ and notice that $E'$ is a matching that is contained in $G$.  By (C2), no two edges in $E'$ cross with respect to $\prec$ and since $A_0$ is an antichain, no two edges in $E_0$ are disjoint.
  Therefore $E_0$ forms a rainbow with respect to $\prec$.  By (C1) we may assume, without loss of generality, that $(v,1)\prec (v,2)$ for each $v\in A_0$.
  
  Now consider the set $V_1$ of the $d'|A_0|$ (depth-2) children, in $F$, of the (depth-0) nodes in $A_0$.  For each node $x\in V_1$ with parent $v\in A_0$, define the colour-quintuple
  \[  q(x) = (\varphi((x,1)(x,2)),\,\varphi((x,1),(v,1)),\,\varphi((x,2)(v,2)),\,\varphi((x,1)(v,2)),\,\varphi((x,2)(v,1))   ) \enspace . \]
  Observe that $q:V_1\to\{1,\ldots,s\}^5$, so the codomain of $q$ has size $s^5$.  Select a subset $V'_1\subseteq S_1$ with the following properties:
  \begin{enumerate}[(D1)]
    \item For some $(a,b)\in\{(1,2),(2,1)\}$, $(v,a)\prec (v,b)$ for each $v\in V'_1$;
    \item $t(v)=t(w)$ for each $v,w\in V'_1$;
    \item $|V'_1|\ge |S_1|/(2s^5)\ge (d')^2/(4s^6d)$.
  \end{enumerate}
  Again, this set is guaranteed by two application of the Pigeonhole Principle.
  
  Let $v_1,\ldots,v_{k}$ denote the nodes $A_0$, ordered so that 
  \[  (v_1,1)\prec (v_2,1)\prec\cdots \prec (v_k,1)\prec (v_k,2)\prec\cdots\prec (v_2,2)\prec (v_1,2) \enspace . \]
  For each $i\in\{1,\ldots,k\}$, let $C_i$ denote the subset of $V'_1$ consisting of children of $v_i$.  For each $i\in\{1,\ldots,k\}$, let 
  \[ 
    B_i := \{(v,a):v\in \{v_i\}\cup C_i,\, a\in\{1,2\}\} \enspace ,
  \] 
  let $\ell_i=\min_\prec B_i$ and $r_i=\max_\prec B_i$.  Think of $\ell_i$ and $r_i]$ as the left and right endpoints of the smallest interval that contains $B_i$.

  
  \begin{clm}\clmlabel{no-overlap}
    For each $1\le i< j\le k$, $B_i\cap [\ell_j,r_j]_\prec=\emptyset$.
  \end{clm}
  
  \begin{proof}[Proof of \clmref{no-overlap}]
    Define $a,b\in\{1,2\}$ so that $(x,a)\prec (x,b)$ for every $x\in V_1'$ (that this is possible is due to (D1)).
    
    First we rule out the possibility that $(v_i,1)\in [\ell_j,r_j]_\prec$.  For the sake contradiction, suppose that $\ell_j\prec (v_i,1)\prec r_j$ and define $(x,a)=\ell_j$.  By necessity $(x,a)\prec (v_i,1)\prec (v_j,1)\prec (v_j,2)$, so $x\neq v_j$.  Observe that $G$ contains the edge $e_1=(x,a)(v_j,1)$ and that $(x,a)\prec (v_i,1)\prec (v_j,1)$.
    Let $z\in C_i$ be any child of $v_i$ and let $e_2=(v_i,1)(z,a)$. By (D2) $\varphi(e_1)=\varphi(e_2)$ so $e_1$ and $e_2$ do not cross.  Therefore $(x,a)\prec (z,a)\prec (v_j,1)$ otherwise $e_1$ and $e_2$ would cross.
    Now consider the edges $e_3=(x,a)(v_j,2)$ and $e_4=(z,a)(v_i,2)$.  By (D2), $\varphi(e_3)=\varphi(e_4)$.  However,
    \[  (x,a)\prec (y,a)\prec (v_j,2)\prec (v_i,2) \]
    so $e_3$ and $e_4$ cross, a contradiction.  Therefore, $(v_i,1)\not\in[\ell_j,r_j]$.  A symmetric argument rules out the possibility that $(v_i,2)\in[\ell_j,r_j]$.
    
    Thus far, we have shown that
    \[  (v_i,1)\prec \ell_j\preceq (v_j,1) \prec (v_j,2)\preceq r_j \prec (v_i,2) \enspace . \]
    Next we rule out the possibility that $(z,a)\in [\ell_j,r_j]\prec$ for some $z\in C_i$.
    
    
    If $\ell_j\neq (v_j,1)$ then let $(x,a):=\ell_j$, otherwise choose $x$ arbitrarily in $C_j$. Similarly, if $r_j\neq(v_j,2)$, let $(y,b):=r_j$, otherwise set $y:=x$.  Observe that, if $(z,a)\in[(x,a),(v_j,2)]$, then $e_1:=(v_i,1)(z,a)$ crosses $e_2=(v_j,1)(x,a)$ or $e_3=(v_i,2)(z,a)$ crosses $e_4=(v_j,2)(x,a)$.  However, neither of these is possible since $\varphi(e_1)=\varphi(e_2)$ and $\varphi(e_3)=\varphi(e_4)$.  
    
    Therefore, it must be the case that $(z,a)\in[(v_j,2),(y,b)]$.  By convention, $(z,a)\prec (z_b)$.  If $z_b\in[(z,a),(v_j,2)]_\prec$, then the edges $e_5:=(v_j,1)(y,b)$ and $e_6:=(v_i,1)(z,b)$ cross, which is not possible since $\varphi(e_5)=\varphi(e_6)$.  Therefore, $(y,b)\prec (z,b)$.  Next, observe that $(y,a)\prec (z,a)$ since, otherwise $e_7:=(v_j,1)(y,a)$ and $e_8:=(v_i,1)(z,a)$ cross which is also not possible.  Thus, we have
    \[  (v_i,1)\prec (v_j,1)\prec (y,a)\prec (z,a)\prec (y,b)\prec (z,b) \prec (v_i,2) \enspace . \]
    Therefore, the edges $e_9:=(y,a)(y,b)$ and $e_{10}:=(z,a)(z,b)$ cross, which is also not possible since $\varphi(e_8)=\varphi(e_9)$.  Therefore $(z,a)\not\in[\ell_j,r_j]_\prec$ for any $z\in C_i$.  A symmetric argument shows that $(z,b)\not\in[\ell_j,r_j]$ for any $z\in C_i$.  This completes the proof of \clmref{no-overlap}.
  \end{proof}
  We now continue the proof of \lemref{star-forest}.
  
  By \clmref{no-overlap}, there exists
  $\alpha_0\prec\alpha_1\prec\cdots\prec\alpha_k\prec\beta_k\prec\cdots\prec\beta_1\prec\beta_0$
  such that $B_i \subseteq [\alpha_{i-1},\alpha_i]_\prec \cup [\beta_{i},\beta_{i-1}]_\prec$ for each $i\in\{1,\ldots,k\}$.
  
  Now, for each $i\in\{1,\ldots,k\}$ define $h_i$ as the number of vertices $x\in V(G)$ such that $(x,1)\in [\alpha_{i-1},\alpha_i]_\prec$ and $(x,2)\in [\beta_i,\beta_{i-1}]_\prec$ or vice-versa.  We say that $(\prec,\varphi)$ is \emph{$i$-balanced} if 
  
  
  
  
  \[ \min\{|[\alpha_{i-1},\alpha_i]_\prec\cap B_i|, |[\beta_{i},\beta_{i-1}]_\prec\cap B_i|\} \ge B_i/4 \enspace . \]
  
  \begin{clm}
    If $(\prec,\varphi)$ is $i$-balanced then $C_i$ contains a subset $C_i'\subseteq C_i$ of size at least $|C_i|/100$ such that $C_{i,1}':=\{(v,1):v\in C_i'\}$ and $C_{i,2}':=\{(v,2):v\in C_i'\}$ interleave.    
  \end{clm}
  
  
  \begin{clm}
    If $(\prec,\varphi)$ is not $i$-balanced then $C_i$ contains a subset $C_i'\subseteq C_i$ of size at least $|C_i|/100$ such that $C_{i,1}':=\{(v,1):v\in C_i'\}$ and $C_{i,2}':=\{(v,2):v\in C_i'\}$ interleave.
  \end{clm}
  
  \begin{proof}
    Since $(\prec,\varphi)$ is not $i$-balanced, $B_i$ has a subset $B_i'$ of size at least $3|B_i'|/4$ such that $B_i'\subseteq [\alpha_{i-1},\alpha_i]_\prec$ or $B_i'\subseteq [\beta_{i},\beta_{i-1}]_\prec$.  Without loss of generality, assume the former, i.e., that $B_i'\subseteq [\alpha_{i-1},\alpha_i]_\prec$.  Therefore, there exists a subset $C_i'\subset C_i$ of size at least $|C_i|/4$ such that $C_{i,1}'\cup C_{i,2}'\subset [\alpha_{i-1},\alpha_{i}]\cap B_i$.
  \end{proof}
  
\end{proof}
  % \begin{clm}
  %   For any $1\le i<j\le k$, any $x\in C_i$, and any $a\in\{1,2\}$, $(x,a)\prec (v_j,1)$ or $(v_j,2)\prec x$.
  % \end{clm}
  % 
  % \begin{proof}[Proof Sketch]
  %   There's a little-bit of case analysis, but the proof is roughly this:  Consider any child $y$ of $v_j$.  The edges $e_1=(v_j,1)(y_j,1)$ and $e_2=(v_j,2)(y_j,1)$ form a barrier against $(x,1)$ and the edges $e_3=(v_j,1)(y_j,2)$ and $e_4=(v_j,2,y_j,2)$ form a barrier against $(x,2)$. (Note that we really use diagonal strong-product edges here.)
  % \end{proof}
  
  
  % In essence, the vertices in $(C_i,1)$ and $(C_i,2)$ are clustered near $(v_i,1)$ and $(v_i,2)$.  Give up a factor of 2 and keep only the stuff clustered near $(C_i,2)$, say.  If we are able to find $d$ clusters that $d$-interleave then we're done.
  % 
  % Otherwise, by relabelling, we find a $(d')^2/f(d,s)$ rainbow consisting entirely of depth-1 nodes and their horizontal edges in the strong product.
  % Now we continue with the subtrees rooted at these nodes\ldots
  % 
  % Actually,\ldots, this lemma should be restated as a preliminary lemma about $F_{d',1}\boxtimes P_2$.  In this forest at least one of three things must happen:
  % \begin{enumerate}
  %   \item There is a subset of $d$ depth-0 nodes that interleave.
  %   \item There is a copy of $T_{d,1}$ whose depth-1 nodes interleave.
  %   \item There is a copy of $T_{d'',1}$ such that the depth-0 nodes form a rainbow and the depth-1 nodes form a rainbow, and these rainbows "match", where $d''\approx (d')^2/(sd)^2$.
  % \end{enumerate}
  % From this we can get the original lemma by induction on $h$.
  % 
  % 
  
  
  
  
  % \begin{enumerate}
  %   \item For each $a\in\{1,2\}$, each $i,j\in\{1,\ldots,k\}$ with $i\neq j$, $(C_i,a)$ and $(C_i,b)$ are separated.  Indeed, 
  % 
  % It has the following properties:  
  % 
  % 
  % 
  % 
  % By the Pigeonhole Principle, there exists a subset $S'_2$ of size at least $|S_2|/s^3$ such that $t(x)=t(y)$ for each $x,y\in S'_2$.
  % 
  % 
  % 
  % 
  % 
  % 
  %   of size at least $d'/(2s)$ such that $\varphi((v,1)(v,2))=\varphi((w,2)(w,2))$ for each $v,w\in R'$.  
  % 
  % 
  % 
  %  then we claim that $S$ contains a subset $R'$ of size at least $d'/(2ds)$ such that $R_1=\{(v,1):v\in R\}$ and $R_2=\{(v,2):v\in R\}$ are separated. 
  % 
  
% 
% 
%   By choosing the most frequently occuring colour $\varphi(e)$ for all $e\in\{(v,1)(v,2):v\in R\}$, we obtain $R\subset R'$ with $|R|\ge |R'|/s$ such $(v,1)\prec (w,1)$ iff $(w,2)\prec (v,2)$ for every $v,w\in R$. In other words, the edges $\{(v,1)(v,2):v\in R\}$ form a rainbow with respect to $\prec$.
% 
% 
% 
% 
%   Applying (something like) \clmref{one-stack-subtree} to get $F'\subset F$ such that $F'_1$ and $F'_2$ are each 1-stack layouts.  Next, walk down from the root, doing pruning as in the proof of \clmref{one-stack-subtree}, but this time classiflying each level $i$ as interleaving or reversing (this uses Dilwerth's Theorem, and also requires pruning).  Argue that, as long as the levels are reversing, $F'_1$ and $F'_2$ are separated.  If the levels never interleave, the $F'_1$ and $F'_2$ are two $d$-ary forests of height $h$ that are separated.  But $F'$ contains  $R=\{x_iy_i:i\in\{1,\ldots,h\}\}$ of size $h$ such that $R_1$ is a rainbow and, because of cross edges in the strong product, the endpoints of $R_1$ form a twist of size $h>s$ with $R_2$.   
% \end{proof}


\begin{clm}
  For every $s\in\N$, there exists $d,h\in N$ such that $\sn(F_{d,h}\boxtimes P_n)> s$.
\end{clm}

The plan is to show this as follows:


\begin{enumerate}
  \item By \clmref{early-interleave}, $F_1$ and $F_2$ must interleave within their first $s$ levels. Suppose this happens first at level $i>0$.  What happens to $F_3$?  Suppose the depth 0 nodes of $F_1$, $F_2$, and $F_3$ are mutually separated.
  \begin{enumerate}
    \item If $F_3$ interleaves with $F_1$ at some level $i'$, $0<i'<i$, then we get a twist.
    \item If $F_3$ interleaves first with $F_1$ at level $i$ then we get a twist.
    \item If $F_3$ interleaves first with $F_2$ at level $i$, then we get a twist.
    \item If $F_3$ does not interleave with $F_2$ before level $i$, then we get a twist.
  \end{enumerate}
  This only leaves two possibilities:
  \begin{enumerate}
    \item $F_1$ and $F_3$ interleave already at level 0.
    \item $F_3$ interleaves with $F_2$ at some level $i'<i\le i-1$.
  \end{enumerate}
  In the first case, we found $F_1$ and $F_3$ interleaving at level 0, so we move on to consider $F_3$ and $F_4$.  In the second case, we've found that $F_3$ must interleave with $F_2$ by level $i-1$.  We should then be able to argue that $F_4$ must interleave with $F_3$ by level $i-2$, and this can't go on indefinitely, since $i<h<s+1$.  
\end{enumerate}  


\end{document}




\begin{clm}
  
  
\end{clm}










\begin{lem}\lemlabel{bbs}
  For every $k,s\in\N$, there exists $d,h,n\in\N$ such that the following is true.  Let $T$ be a complete $d$-ary tree of height $h$ and $P$ be a path of length $n$.  Then, for any $s$-stack layout $(\prec,\varphi)$ of $T\times P$, there exists an edge $(x,y)\in P$ and a complete binary subtree $T'$ of $T$ of height $h$ such that $(v,x) \prec (w,y)$ for every $v,w\in V(T')$.
\end{lem}

Note: Actually, we only need the $T'_x$ to be separated from the leaves of $T'_y$.

Once we have \lemref{bbs}, the rest of the proof is easy.  By David's argument, the layout of $T'_x$ contains a rainbow of size $\omega_h(1)$.  Now, the fact that $T'_x$ and the leaves of $T'_y$ are separated along with some easy applications of Dilwerth's Theorem (Erd\H{o}s-Szkeres) proves that any $s$-stack layout of $T'\boxtimes (x,y)$ contains a twist of size $\omega_h(1)$.

Now let's work on proving \lemref{bbs}.  

A total order $(\prec,X\times Y)$ is \emph{x-major} if there exists a a total order $(\prec,X)$ such that, for all $x,x'\in X$ and $y,y'\in Y$, $x\prec x'$ implies that $(x,y) \prec (x',y')$.  A total order $(\prec,X\times Y)$ is \emph{y-major} if there exists a a total order $(\prec,Y)$ such that, for all $x,x'\in X$ and $y,y'\in Y$, $y\prec y'$ implies that $(x,y) \prec (x',y')$. 

The following Lemma is incorrect as stated, but let's pretend it's correct for now and see how easy the rest of the proof is:
\begin{lem}\lemlabel{major}
  For every $r,t\in\N$,  $t>r$, there exists $d\in\N$ such that, for every total order $(\prec, [d]\times[r])$, there is a $t$-subset $D\subset[d]$ such that at least one of the following is true:
  \begin{compactenum}
    \item $\prec$ is an x-major ordering of $D\times R$ for some $R\subseteq [r]$ of size at least $t$; or
    \item $\prec$ is a y-major ordering of $D\times[r]$.
  \end{compactenum}
\end{lem}

\lemref{major} has implications for $s$-stack layouts of stars.  For each $d\in\N$, let $S_d$ denote the $d$-leaf star with root $0$ and leaf set $[d]$.  For every $r\in\N$, let $P_r$ denote the $r$-node path $(1,2,3,\ldots,r)$.

\begin{lem}\lemlabel{star}
  For every $s,t,r\in\N$, $r>s$, there exists $d\in\N$ such that, every $s$-stack layout $(\prec,\varphi)$ of $S_d\boxtimes P_r$, there exists a $t$-subset $D\subset[d]$ of the leaves of $S_d$ such that $\prec$ is a y-major ordering of $D\times[r]$.
\end{lem}

\begin{proof}
  By \lemref{major}, the only other alternative is that $\prec$ is an x-major ordering of $D\times R$ for some $t$-element subset $R\subseteq [r]$.  But then it is straightforward to verify that this implies that the edge set $\{(0,y)(x,y):x\in D,y\in R\}$ forms a twist of size $t \ge r > s$.  (Indeed, identifying the vertices $\{(x,y):y\in R)\}$ for each $x\in D$ does not introduce any crossing edges, but gives a layout of $K_{t,t}$.) 
\end{proof}

For each $d,h\in N$, let $T_{d,h}$ denote the complete $d$-ary tree of height $h$ (so that $T_{d,h}$ has $d^h$ leaves).

Consider an $s$-stack layout $(\prec,\varphi)$ of $G:=T_{d,h}\times P_r$.  For convenience, let $T=T_{d,h}$ and $P:=P_r$.  Fix any node non-leaf node $v\in V(T)$.  By \lemref{star} there is a $t$-subset $D\subset V(T)$ consisting of children of $v$ in $T$ such that $\prec$ is a y-major ordering of $D\times[r]$.  Consider the vertices $D\times\{1,2\}$.  Since $\prec$ is y-major over $D$, we may assume without loss of generality $(x,1)\prec (x',2)$ for every $x,x'\in D$.  Since $G$ contains the edges $\{(x,1)(x,2): x\in D\}$ we can conclude, by Dilwerth's Theorem, that these edges can be partitioned into $s$ rainbows, one of which has size at least $t/s$.  That is, $D$ contains a subset $D_1$ of size at least $t/s$ such that $D_1\times\{1,2\}$ is a rainbow.

By iterating the preceding argument we obtain sets $D_2\subseteq D_1\times \{2,3\}$, $D_3\subseteq D_2\times\{3,4\}$,\ldots, $D_{r-1}\subseteq D_{r-2}\times\{r-1,r\}$, where each $D_i$ has size at least $t/s^i$.

The set $D':=D_{r-2}\subseteq D$ has size at least $t'=t/s^{r-2}$ and defines a total order $(\prec,D')$ such that, for every $x,x'\in D'$ and $y\in [r]$, $(x,y)\prec (x',y)$ if and only if
\begin{compactitem} 
  \item $x\prec x'$ and $y$ is even; or 
  \item $x'\prec x$ and $y$ is odd.
\end{compactitem}


Now we focus on $D'\times\{y,y+1\}$ for some $y\in\{1,\ldots,r-1\}$.  Observe that the $s$-stack layout $(\prec,\varphi)$ contains a $t'$-rainbow consisting of edges $\{(x,y)(x,y+1): x\in D'\}$.  Without loss of generality, assume:
\[  (x,y) \prec (x',y+1) \]
for every every $x,x'\in D'$ and give names $x_1,\ldots,x_{t'}$ to the elements of $D'$ so that
\[
    (x_1,y) \prec (x_2,y) \prec \cdots (x_{t'},y) 
    \prec (x_{t'},y+1) \prec \cdots (x_2,y+1) \prec (x_1,y+1) 
\]
Consider some child $z$ of some node $x\in D'$.  The graph $G$ contains the edges $(x,y)(z,y)$ and the edges $(x,y+1)(z,y)$.  We say that $z$ is \emph{$y$-well-behaved} if $(x_1,y)\prec (z,y) \prec (x_{t'},y)$ or if $(x_{t'},y+1)\prec (z,y)\prec (x_1,y+1)$.  Otherwise, $z$ is \emph{$y$-ill-behaved}.

\begin{lem}\lemlabel{ill-behaved}
  In any $s$-stack layout of $G$, for any $y\in\{1,\ldots,r-1\}$ the number of nodes $x\in D'$ such that $x$ has a $y$-ill-behaved child is at most $3s^2$.
\end{lem}

\begin{proof}
  For each $y$-ill-behaved child $z$, $(z,y)$ falls into one of three intervals (before $(x_1,y)$, between $(x_{t'},y)$ and $(x_{t'},y+1)$ or after $(x_1,y+1)$) and, if the result is not true, then one of these intervals contains at least $s^2+1$ $y$-ill-behaved children.  Dilwerth's Theorem (or Erd\H{o}s-Szekeres) implies that these $s^2+1$ $y$-ill-behaved children form a twist of size at least $s+1$ with $(x_1,y),\ldots,(x_{t'},y)$ or with $(x_{t'},y+1),\ldots,(x_1,y+1)$. (Note, the 3 is probably unnecessary here, but I'm feeling lazy.)
\end{proof}

With the same notation as above, we say that a $y$-well-behaved child $z$ is a \emph{$y$-angel} if $(x_1,y) \prec (z,y)\prec (x_{t'},y)$.

\begin{obs}
  For any $y\in\{2,\ldots,r-1\}$ any node $z$ that is not a $y$-angel is $y$-ill-behaved or $(y-1)$-ill-behaved.
\end{obs}

This observation leads to the following corollary of \lemref{ill-behaved}

\begin{cor}\corlabel{devils}
  In any $s$-stack layout of $G$, for any $y\in\{2,\ldots,r-1\}$ the number of nodes $x\in D'$ such that $x$ has a child that is not a $y$-angel is at most $6s^2$.
\end{cor}

Now we're basically done.  Apply \lemref{star} using the star $S$ formed by the root of $T$ and its children.  This yields the set $D'$ of $t'$ depth-1 nodes in $T$ such that $\prec$ is a y-major ordering of $D'\times[r]$.  Delete every node in $D'$ that has a child $z$ that is not a $y$-angel for some $y\in\{2,\ldots,r-1\}$.   By \corref{devils}, this yields a set $D''\subseteq D$, $|D''|\ge|D'|-6rs^2\ge t/s^r - 6rs^2$.   Recall that the only conditions on $s,t,r$ and are that $r>s$ so, for any fixed $s$ we can choose $r=s+1$ and set $t$ arbitrarily large.  At this point we have found a subtree $T'$ of $T$ of height $2$ in which the root has at least $t/s^r - 6rs^2$ children, each depth-1 node has at least $d$ children and such that $\prec$ is a y-major ordering of $V(T')\times[r]$.   In particular, $T'$ contains a binary tree of height $2$ satisfying the conditions of \lemref{bbs}.

Now we can proceed the same way on the depth-1 nodes of $T'$.  Note that, for each such node $v$ whose children (in $T$) are $C_v$, the set $C_v$ satisfies the conditions of the set $D$ we started with. In particular $\prec$ defines a y-major ordering of $C_v\times[r]$.  Thus, we can repeat this argument to find a height-3 subtree, a height-4 subtree, and eventually a height-$h$ subtree $T'_h$ such that $\prec$ is y-major ordering of $T'_h\times[r]$.  In particular, the tree $T'_h$ contains a complete binary tree of height $h$ that satisfies the conditions of \lemref{bbs}.

Unfortunately, \lemref{major} (and therefore \lemref{star}) is not quite correct.  The best we can show is the following:

\begin{lem}\lemlabel{star-ii}
  For every $s,t,r\in\N$, $r>s$, there exists $d\in\N$ such that, every $s$-stack layout $(\prec,\varphi)$ of $S_d\boxtimes P_r$, there exists a $t$-subset $D\subset[d]$ of the leaves of $S_d$ such that for every $y\in\{1,\ldots,r-1\}$ at least one of these holds:
  \begin{compactenum}
    \item $\prec$ is a y-major ordering of $D\times\{y,y+1\}$ and $D\times\{y,y+1\}$ is a monotone rainbow
    \item $\prec$ is a y-major ordering of $D\times\{y+j,y+j+1\}$.
  \end{compactenum}
  Furthermore, every $y\in\{1,\ldots,r-s-1\}$, Property~1 holds for $y+j$ for at least one $j\in\{0,\ldots,s-1\}$.  Furthermore, if Property~2 holds for $y,y+1,\ldots,y_j$ then the permutation of $D$ induced by $\prec$ on $D\times\{y\}$ is the same as that induced by $\prec$ on $D\times\{y+j\}$.
\end{lem}

This is slightly more complicated to work with since it means that there are blocks of $y$ values such that $\prec$ is an x-major ordering of $D\times\{y,y+1,\ldots,y+r\}$.  This makes it trickier to control the placement of the children of $\{x\}\times\{y,y+1,\ldots,y+r\}$. What we can show, though, is that for most $x$, at least one of $(x,y),(x,y+1)\ldots,(x,y+r)$ has no child that is not an angel.  (Note to self: Look out for the possibility that $D\times\{y\}$ and $D\times\{y+2\}$ overlap/interleave even though $D\times\{y\}$ and $D\times\{y+1\}$ are disjoint.)

This (easy) lemma will also be helpful.

\begin{lem}
  Let $G$ be any connected graph with $|V(G)|\ge 2$, let $P=(1,\ldots,r)$ be a path and let $(\prec,\varphi)$ be an $s$-stack layout of $G\boxtimes P$.  Then, for every vertex $v\in V(G)$, the path $P_v=(v,1),\ldots,(v,r)$ does not contain any rainbow of size $s^2+1$ (or twist of size $s$).
\end{lem}

The preceding lemma implies that $P_v$ has bounded bandwidth.


We can prove \lemref{star-ii} using the following lemma:

\begin{lem}\lemlabel{dichotomy}
  Let $k, m\in\N$, let $X$ be an $m$-element set and let $Y$ be 2-element set.
  let $S=X\times Y$, and let $\prec$ be a total order on $S$.  Then at least one of the following is true:
  \begin{compactenum} 
    \item There exists a $\ceil{k/2}$-element subset $D\subseteq X$ such that $\prec$ is a y-major ordering of $D\times Y$.
    \item There exists an $\floor{m/k}$-element subset $D\subseteq X$ such that $\prec$ is a x-major ordering of $D\times Y$.
  \end{compactenum}
\end{lem}

\begin{proof}
  For each $i\in\{0,\ldots,2m\}$ let $S_i$ denote the first $i$ elements of $S$ in $\prec$ order.  Suppose there exists a value $i$ and set $Z\subseteq S_i$, $|Z|=k$ such that, for every $(x,y)\in Z$, $(x,1-y)\not\in S_i$.  Then for some $b\in\{0,1\}$, the set $Z_b=\{(x,y)\in Z: y=b\}$ has size at least $k/2$.  We take $D=\{x : (x,y)\in Z_b\}$ and observe that $\prec$ is a y-major ordering of $D\times\{0,1\}$, so $D$ satisfies Property~1.
  
  If there is no such $i$ and $Z$, then partition the first $k\floor{m/k}$ elements of $S$ into consecutive blocks $B_1,\ldots,B_{\floor{m/k}}$ each of size $2k$ (So $B_j=S_{2kj}\setminus S_{2k(j-1)}$). For each $j\in\{1,\ldots,\floor{m/k}\}$, $A_j:=B_1\cup\cdots\cup B_j$ contains $2jk$ elements and there is a set $X_j\subset X$ of size at least $\tfrac{1}{2}(2jk-(k-1))$ such that $X_j\times Y\subseteq A_j$.  Furthermore, $|X_j\times Y \cap B_j|\ge k+1$.  Since $X_{j-1}\times Y \cap A_{j-1}\ge 2k(j-1)-k+1$, this implies that there is at least one value $x_j\in[m]$ such that $(x_j,0),(x_j,1)\in B_j$. We take $D=\{x_j:j\in\{1,\ldots,\floor{m/k}\}$ and observe that $\prec$ is an x-major ordering of $D\times Y$, so $D$ satisfies Property~2.
\end{proof}


We can now prove \lemref{star-ii}.

\begin{proof}[Proof of \lemref{star-ii}]
  Define $D_0=[d]$, the set of leaves of $S$ and define $X_1\subset D_0$ such that $X_1\times\{1\}$ is a chain or antichain in $\prec$. By Erd\H{o}s-Szekeres, $|X_1|=\Omega(\sqrt{d})$.

  Apply \lemref{dichotomy-ii} iteratively.  During the $i$th iteration, we apply \lemref{dichotomy-ii} with $k=|X_{i}|^{2/3}$, $\prec$, $X=X_{i}$, and $Y=\{i,i+1\}$, yielding a set $D_i$ of size $\Omega(|X_{i}|^{1/3})$ satisfying Property~1 or Property~2.  If $D_i$ satisfies Property~1 then we say that iteration $i$ was a \emph{split} iteration.  Otherwise, iteration $i$ was an \emph{interleaving} iteration.
  
  If iteration $i$ was a split iteration, use Dilwerth's Theorem to find a subset $X_{i+1}\subseteq D_i$ such that $X_{i+1}\times\{i+1\}$ is a chain or antichain in $\prec$.  The fact that $X_i\times\{i\}$ was a chain or antichain in $\prec$ and that $(\prec,\varphi)$ is an $s$-stack layout implies that $|X_{i+1}| \ge |X_i|/s$.
  
  If iteration $i$ was an interleaving iteration, then look back to the smallest value $a$ such that iteration $i-a$ was a split iteration.  Thus, there is a sequence of disjoint intervals $\{(a_x,b_x) : x\in D_i\}$ such that, for every $x\in D_i$ and every $j\in\{i-a+1,\lots,i\}$, $a_x\preceq (x,i)\preceq b_x$ \ldots
  
\end{proof}












\end{document}



Let $V$ be any set, let $S=\{(v,b): v\in V, b\in\{0,1\}$, and let $\prec$ be a total order on $S$.  We say that $\prec$ \emph{separates} a subset $X\subseteq V$ if $(v,b) \prec (v,{1-b})$ for every $v\in X$ and some $b\in\{0,1\}$.  
% We say that $\prec$ \emph{mixes} $X$ if $\prec$ orders $X\times\{0,1\}$ as
% \[
%    (x_1,b)\prec (x_2,1-b)\prec (x_{3},b)\prec \cdots \prec (x_{2|X|},1-b) \enspace .
% \]



We say that $\prec$ \emph{interleaves} $X$ if there exists an ordering $x_1,\ldots,x_{|X|}$ of $X$ and $b_1,\ldots,b_{|X|}\in\{0,1\}^{|X|}$ such that 
\[
  (x_1,b_1)\prec(x_1,1-b_1)\prec(x_2,b_2)\prec(x_2,1-b_2)\prec\cdots\prec (x_{|X|},b_{|X|})\prec(x_{|X|},1-b_{|X|})
  \enspace .
\]
% We say that $\prec$ \emph{splits} $X$ if there exists an ordering $x_1,\ldots,x_{|X|}$ of $X$ and $b\in\{0,1\}$ if
% \[
%   (x_1,b)\prec (x_2,1-b)\prec\cdots\prec (x_{|X|},?)\prec(x_{|X|},1-?)\prec\cdots\prec (x_2,b)\prec(x_1,1-b) 
% \]







\begin{lem}\lemlabel{dichotomy}
  Let $V$ be any set, $|V|=m$, let $S=\{(v,b): v\in V, b\in\{0,1\}$, and let $\prec$ be a total order on $S$.  Then there exists a $\Omega(\sqrt{m})$-element subset $X\subseteq V$ such that $\prec$ separates $X$ or $\prec$ interleaves $X$.   
\end{lem}

\begin{proof}
  Scan $S$ in $\prec$ order.  If at the $i$'th iteration you see $i/2 + k$ distinct $v$ values, then you've found a set $X$ of size $k/2$ that $\prec$ separates.  If that never happens then, among the first $2k$ values, at least $k+1$ of them are matched.  Among the next $2k$ values $k+1$ of them are matched. At most $k-1$ of these matching edges go back to unmatched edges in previous blocks, so at least one matching edge is in the second block.  Proceed the same way to find $m/(2k)$ matched pairs.
\end{proof}


\begin{lem}\lemlabel{star-separation}
  For every $k,s\in\N$, there exists $d,n\in\N$ such that the following is true.
  Let $S$ be a $d$-leaf star and $P$ be an $n$-vertex path.  Then, in any $s$-stack layout $(\prec,\varphi)$ of $S\times P$, there exists an edge $xy\in E(P)$ and a $t$-vertex subset $S'\subset V(S)$ such that $(v,x) \prec (w,y)$ for every $v,w\in S'$.
\end{lem}

\begin{proof}
  Apply \lemref{dichotomy} iteratively to the copies of the leaves of $S$ generated by the path $P$.  If at any point we get a set $X$ such that $\prec$ splits $X$, then we're done.  Otherwise we have $n$ copies of a large star with their leaves interleaved.  This gives a large twist, which can't happen in an $s$-stack layout.  
\end{proof}


I think we're done.  

\begin{thm}
  For every $k\in\N$, there exists $d,h,n\in N$ such that the following is true.
  Let $T$ be a complete $d$-regular tree of height $h$ and let $P$ be a path of length $n$.  Then $\sn(T\boxtimes P)\ge k$.
\end{thm}

\begin{proof}
  Apply \lemref{star-separation} to the root of $T$ and its and its children to obtain an edge $xy\in E(P)$ and a $t$-element set $X$ of depth-$1$ nodes in $T$ such that $(v,x)\prec (w,y)$ for every $v,w\in X$.  Therefore, the layout contains a matching $M=\{(v,x)(v,y):v\in X\}$ of size $t$ whose left and right endpoints are separated.  An easy application of Dilwerth's Theorem implies that $X$ contains a subset $X'$ of size at least $\sqrt{t}$ that defines a rainbow, i.e., for every $v,w\in X'$, $(v,x)<(w,x)$ if and only if $(w,y)<(v,y)$.

  Now we use strong product.  For each node $x'\in X'$ with child $x''$, $T\boxtimes P$ contains the edges $(x',0)(x'',0)$ and $(x',0)(x'',1)$.  The only way to avoid a big twist is to interleave the $(x',0)$s and $(x'',0)$s.  The same thing is true about the $(x',1)$s and $(x'',1)$s.  But now we've found a big height-2 subtree. that is $\prec$ separated...... Hmmmmmm.
    % 
    % 
    % Another couple of applications of Dilwerth shows that this 
  % 
  % 
  % 
  % 
  % Let $Y$ be the children of the nodes in $X'$.
  % For each node $x  The graph $T\boxtimes P$ contains the edges $(v,x)
  % 
  % 
  % 
  %  set consisting of exactly one child from each element of $X$.  
  % 
\end{proof}





% Now, how to do we get from here to a large binary subtree, i.e., \lemref{bbs}?  I'm not sure yet, but we're getting close.  Here's an idea.  Find an enormous set of leaves at the root that are separated by $(x,y)$.   Now consider the (enormous) number of children of these leaves.  
% 






















% Let $S$ denote a star with root $s_0$ and $n$ leaves $s_1,\ldots,s_n$.  Let $P$ denote the path $p_1,\ldots,p_n$ of length $n$.  We are interested in stack layouts of $S\boxtimes P$ (and also $S\times P$).
% 
% For a total order $<$ over some set $X$, we say that two subsets $A,B\subseteq X$ are \emph{separated} with respect to $<$ if $a<b$ for all $a\in A$ and $b\in B$ or if $b<a$ for all $a\in A$ and $b\in B$.  In the former case, we write $A<B$ and in the latter case $B<A$.  A sequence of subsets $A_1,\ldots,A_r\subseteq X$ are \emph{separated} (with respect to $<$) if $A_i$ and $A_j$ are separated for every distinct $i,j\in\{1,\ldots,r\}$.
% 
% Let $G=S\boxtimes P$ and, for each $i\in\{0,\ldots,n\}$ and $j\in\{1,\ldots,n\}$, $v_{i,j}=(s_i,p_j)$.  For index sets $I\subseteq\{0,\ldots,n\}$ and $J\subseteq\{1,\ldots,n\}$, we define
% $v_{I,J}=\{v_{i,j}: i\in I,\, j\in J\}$.
% 
% \begin{lem}\lemlabel{twister}
%   Let $<$ denote a total order over $V(G)$ and suppose there exists $I,J\subseteq\{1,\ldots,n\}$ such that
%   \begin{compactenum} 
%     \item $|I|=|J|=k$; 
%     \item $v_{I,\{j\}}$ and $v_{I,\{j'\}}$ are separated for all distinct $j,j'\in J$.  
%   \end{compactenum}
%   Then the layout $(G,<)$ has a twist of size $k$.
% \end{lem}
% 
% \begin{proof}
%   Let $j_1,\ldots,j_{k}$ be the ordering of $J$ such that $v_{0,j_1}< v_{0,j_2}<\cdots< v_{0,j_{k}}$ and let $i_1,\ldots,i_{k}$ be the ordering of $I$ such that $v_{i_1,J}<v_{i_2,J}<\cdots<v_{i_k,J}$.  Then the edge set
%   \[   
%     \left\{ v_{0,j_a}v_{a,j_{a}} : j\in\{1,\ldots,k\} \right\}
%   \]
%   form a twist of size $|J|$.  Indeed, by definition,
%   \[
%       v_{0,j_1} < \cdots < v_{0,j_k}
%   \]
%   and, since $v_{i_1,J}<\cdots< v_{i_k,J}$, 
%   \[
%       v_{i_1,j_1} < v_{i_2,j_2} < \cdots < v_{i_{k},j_k} \enspace . \qedhere
%   \]   
% \end{proof}
% 
% % This next one is wrong:
% % \begin{cor}\corlabel{twister}
% %   Let $<$, $G$, $I$, $J$ satisfy Conditions 1 and 2 of \lemref{twister} (but not necessarily Condition~3).   Then the layout $(G,<)$ has a twist of size at least $\floor{k/2}$.
% % \end{cor}
% % 
% % \begin{proof}
% %   Let $j_1,\ldots,j_{k}$ be the ordering of $J$ such that $v_{0,j_1}< v_{0,j_2}<\cdots< v_{0,j_{k}}$ and consider the median element $v_{0,j_{\floor{k/2}}}$.  Then,  
% %   \begin{enumerate}
% %     \item There is a set $I'\subseteq I$, $|I'|\ge \floor{k/2}$ such that $v_{\{0\},\{j_{\floor{k/2}}\}} < v_{\{i\},J}$ for all $i\in I'$; or
% %     \item There is a set $I'\subseteq I$, $|I'|\ge \floor{k/2}$ such that $v_{\{i\},J} < v_{\{0\},\{j_{\floor{k/2}}\}}$ for all $i\in I'$.
% %   \end{enumerate}
% %   In the first case we can apply \lemref{twister} directly to the sets $I'$ and $J'=\{j_1,\ldots,j_\floor{k/2}\}$.  In the latter case we can apply \lemref{twister} (with condition 3 replaced by $\{v_{\{0\},J}\}>v_{I,J}$) to the sets $I'$ and $J'=\{j_{\floor{k/2}+1},\ldots,j_k\}$.  
% % \end{proof}
% 
% The preceding lemma dispenses with the possibility that $S\times P$ has an $S$-major layout.  The next lemma shows that, any layout of $S\times P$ that is not $S$-major has a large amount of interleaving.
% 
% \begin{lem}\lemlabel{blocks-or-interleave}
%    There exists a function $k:\N\to\N$ with $k(n)\in\omega_n(1)$ such that, for any ordering $<$ over $V(G)$,  at least one of the following is true:
%   \begin{enumerate}
%     \item The conditions of \corref{twister} hold with $|I|=|J|=k$.
%     \item There exists $I,J\subseteq\{1,\ldots,n\}$ with $|I|=|J|=k$ and an ordering $j_1,\ldots,j_k$ of $J$ such that $v_{I,j_1}< v_{I,j_2}<\cdots<v_{I,j_k}$.
%   \end{enumerate}
% \end{lem}
% 
% \begin{proof}
%   Consider the set $S$ of $tn$ leftmost vertices for some value of $t$ to be decided later.
%   \begin{enumerate}
%     \item If the vertices in $S$ take on at most $a$ different $J$-values then there is some $j$ and $I$, $|I|\ge\ceil{t/a}$ such that $S$ contains $v_{I,j}$ Save $I_1=I$ as an $I$-block and continue on $v_{I,[n]\setminus\{j\}}$. The point is that this gets us an $I$-block of size $\ceil{a/t}$ and lets us recurse on a set of size $\ceil{t/a}$.
% 
%     \item If the vertices in $S$ take on at most $a$ different $I$ values then there is some $i$ and $J$, $|J|\ge\ceil{t/a}$ such that $S$ contains $v_{i,J}$.  Save $J_0=J$ as a $J$-block and continue on $v_{[n]\setminus\{i\},J}$.  Here we get a $J$-block of size $\ceil{n/t}$ and get to recurse on a set of size $t/a$.
%   \end{enumerate}
%   We could take $t=n^2/2$ and $a=n/\log n$.  Then we can repeat this procedure $\log n-\log\log n$ times until we eventually get $\log n$ sets, the last of which has size $\log n$.  We can probably do better if we parameterize by $n$, the number of leaves in $S$ and $\ell$, the length of $P$. Maybe $\sqrt{n}$ is the right answer.
% \end{proof}
% 
% \begin{lem}\lemlabel{tiger}
%   For any constant $s$, and any $s$-stack layout of $G=S\times P$, there exists $I,J\subseteq\{1,\ldots,n\}$, $|I|=|J|\in\omega_n(1)$ and an ordering $j_1,\ldots,j_k$ of $J$ such that $v_{I\cup\{0\},j_1}< v_{I\cup\{0\},j_2}<\cdots<v_{I\cup\{0\},j_k}$.   
% \end{lem}
% 
% \begin{proof}
%   Use the previous lemma and argue that, in the second case, if we find large sets $I'\subseteq I$ and $J'\subseteq J$ such that $v_{0,J}<v_{I',J'}$ then we get a large rainbow (same in as in \lemref{twister}).  If we can't do that, then the same line of reasoning used in \lemref{blocks-or-interleave} proves the result.
% \end{proof}
% 
% Actually, what we've shown so far is that any $s$-stack layout of $n$ disjoint copies $S$ will contain $k$ separated copies of a subgraph of $S$.
% 
% Next step, extend \lemref{tiger} to show that any $s$-stack layout of the Cartesian product of the $n$-ary tree of height $2$ with $P$ has to have a large subtree that is interleaved.  
% 
% Finally, this should kill the possibility of an $s$-stack layout of the Cartesian product of the $n$-ary tree of height $2$ with shortcuts from the root to the leaves with a path.   


\end{document}
